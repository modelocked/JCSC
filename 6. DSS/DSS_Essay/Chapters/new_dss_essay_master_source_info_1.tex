
DETERRENCE
Deterrence is one power attempting to influence the decision making of another
Diplomacy which Aims to resolve a conflict without full scale war. “Coercive diplomacy” is a disruptive form of strategy.
Consider North Korea who have resisted coercive diplomacy as an example of where it hasn’t worked. You must understand the society you’re dealing with.
Often based on punishment or denial or both
Thomas Schelling is the key author for this
1.	Threat must be sufficiently potent
2.	Threat must be credible in the mind of the adversary
3.	Corer must assure adversary that compliance will not lead to more demands
4.	Conflict must not be perceived as a zero sum gain
Jakobsen’s (1998)
1.	Threat to defeat the opponent quickly and with little cost
2.	A deadline for compliance
3.	Assurance to the adversary against future demands
4.	An offer of inducement for compliance




Ayiotis (2023) Joint Irish–British military planning and operations: a historical perspective Historical analysis using Military Archives and secondary sources Irish–UK joint planning recurred since 1922; peaked in WWII; continued via Operation Sandstone (pp. 12–20). Plan W; 18th Military Mission; pragmatic neutrality; elite liaison; post-war Sandstone survey (pp. 14–20). Draws on named letters, mission files, concrete tasks; situates debate after Ukraine shock (pp. 12–20). Scope selective; little dissent analysis; some reliance on press for modern context (pp. 12–13). Aligns with O’Halpin on implicit reliance; Kinsella on reactive capability (pp. 14–16, 20). Contrasts Churchill’s rhetoric with discreet cooperation; jointness shown as norm not exception (pp. 16–18). Irish lens: neutrality as expedience; cross-border planning; memory dulled by Troubles (pp. 18–20). Historically grounded case: structured UK joint planning remains realistic if framed for consent (pp. 12–20).

DIMERS card (LaTeX)

\section*{Source Analysis — \textit{Ayiotis 2023}, Joint Irish–British Military Planning and Operations: A Historical Perspective}
\textbf{Describe:} Explains how Irish–UK joint planning has been recurrent since 1922, peaking under Plan W and the 18th Military Mission, then resuming with Operation Sandstone (pp. 12–20).
\textbf{Interpret:} Directly relevant to whether contemporary Irish defence should formalise UK alignment; excludes deep public-opinion modelling but notes post-Troubles sensitivities (pp. 19–20).
\textbf{Methodology:} Archival correspondence, liaison records, ORBAT notes, plus contextual journalism; valid for historical inference, limited for forecasting (pp. 15–19).
\textbf{Evaluate:} Contribution is precise granularity on how joint planning worked; it bites by normalising cooperation; contradiction arises against rhetorical neutrality (pp. 16–18).
\textbf{Author:} Commissioned in Defence Forces Review; institutional lens aims at policy relevance; counter-voices flagged via Kinsella and forum debates (pp. 12–13, 20).
\textbf{Synthesis:} Aligns with O’Halpin on reliance and with Kinsella on reactivity; diverges from purist neutrality readings by showing cooperation as norm (pp. 14–16, 20).
\textbf{Limit.} Focuses on planning more than outcomes and public legitimacy.
\textbf{Implication:} Irish DF can pursue scoped, transparent UK joint planning while protecting political legitimacy through clear mandates.

Method weight

3 — Observational and doctrinal history using archival evidence, not experimental or quasi-experimental.

Claims-cluster seeds

Claim: Since 1922, Irish–UK joint planning has been the rule rather than the exception.
Best line with page: Abstract states recurrence from 1922, peaking WWII, extending to Sandstone (p. 12–13).
Rival reading: Apparent cooperation was episodic and contingent.
Condition: Holds when archives show sustained liaison, not ad hoc.
Irish DF implication: Build formal mechanisms that acknowledge historic practice.

Claim: Plan W and the 18th Military Mission enabled executable joint defence on Irish soil.
Best line with page: Plan W scope and mission liaison detailed (p. 16, 14–15).
Rival reading: Plans were insurance only, never politically saleable.
Condition: Triggers defined by invasion or request.
Irish DF implication: Pre-agreed triggers and liaison cut delays.

Claim: Aligning with Britain could strengthen neutrality credibility against Germany.
Best line with page: Col James Flynn memo arguing removal of British aggression risk helps neutrality (p. 18).
Rival reading: Any alignment corrodes neutrality.
Condition: Requires diplomatic assurance against UK unilateralism.
Irish DF implication: Pair plans with state-to-state assurances.

Claim: Post-war Operation Sandstone shows resumed cooperation despite non-NATO stance.
Best line with page: Aerial coastal survey 1948–55, purpose and NATO refusal noted (p. 19).
Rival reading: Technical survey, not defence alignment.
Condition: Cooperation limited to specific tasks.
Irish DF implication: Use task-bounded projects to manage politics.

Claim: Public memory dulled the record of cooperation, complicating consent today.
Best line with page: Liaison depth unknown for decades; Troubles reframed sensitivities (p. 19–20).
Rival reading: Current debate is informed enough.
Condition: Requires transparent communication.
Irish DF implication: Pair policy with civic briefings.

PEEL-C drafting — Two paragraphs (for and against)

Point: Ireland should pre-agree joint defence planning with the UK for constrained contingencies, because the historic record shows it works under pressure.
Evidence: Plan W and the 18th Military Mission established practical cross-border tasks, liaison, and even airfield siting for RAF operations (p. 14–16).
Explain: These mechanisms reduce decision latency, align triggers, and translate neutrality into credible defence capacity by removing ambiguity over help on request.
Limit. Cooperation must be request-based, time-bounded, and audited to prevent mandate creep.
Consequent: The Irish DF should draft trigger matrices and liaison SOPs with UK counterparts to protect subsea cables and airspace within Irish political red lines.

Point: Yet formalising UK alignment risks re-politicising neutrality and re-opening post-Troubles sensitivities, so scope must be tightly framed.
Evidence: Public acknowledgement of wartime liaison only emerged decades later, and memory of cooperation was dulled, shaping perceptions (p. 19–20).
Explain: Hidden practice undermines consent; today’s legitimacy needs transparency and parliamentary oversight, or alignment could erode trust more than it adds security.
Limit. A purely domestic posture cannot offset capability gaps in air and sea without partners.
Consequent: For a small state, publish a short, specific joint-planning note with triggers, governance, and sunset clauses to reassure the Oireachtas and public.
% Evidence–Implication Log (LaTeX)
\begin{tabular}{p{3.2cm}p{4.2cm}p{3.6cm}p{3.2cm}p{4.2cm}}
	\textbf{Claim} \& \textbf{Best source (page)} \& \textbf{Rival source/reading} \& \textbf{Condition} \& \textbf{Implication for Irish DF}\\\hline
	Joint planning recurrent since 1922 \& Ayiotis abstract (pp. 12–13) \& Episodic only \& Sustained liaison exists \& Normalise scoped UK planning \\
	Plan W enabled executable defence \& Plan W, 18th MM (pp. 14–16) \& Political fiction \& Clear invasion/request triggers \& Pre-agree triggers, SOPs \\
	Alignment can bolster neutrality \& Flynn memo (p. 18) \& Alignment erodes neutrality \& Diplomatic assurances \& Pair plans with guarantees \\
	Sandstone shows post-war coop \& Operation Sandstone (p. 19) \& Technical survey only \& Task-bounded scope \& Use project-based cooperation \\
	Memory complicates consent \& Dull public awareness (pp. 19–20) \& Debate well informed \& Transparent comms \& Publish joint-planning note \\
\end{tabular}

\bigskip

\noindent\textbf{Gaps}

\noindent Chase: Primary 18th MM files on ROE, command chains, and exact triggers.\\
Park: Broader EU–NATO alignment debate until Irish–UK bilateral guardrails are specified.

\noindent Limit. If clustering or method weighting is skipped, prose slides into description.\\
Implication: Enforce clusters and weights or criticality will miss JCSC expectations.

\parencite{BAILES_2012}

% MATRIX_TSV_ROW
% (leave as-is if intended to paste into Excel)

Bailes \& Thorhallsson (2013)	Instrumentalizing the European Union in Small State Strategies (p.99)	Conceptual analysis using literature and observed examples; strategy texts as lens (pp.101,105)	EU offers soft-security ‘shelter’; existential benefits with sovereignty costs (pp.109–112)	Supranationalism; soft-security tools; Europeanization pressure; ‘escape from smallness’ (pp.105,111)	Integrates small-state theory with EU governance; clarifies non-military security (pp.105–111)	Limited case detail; implementation pathways thin [NO SOURCE]	Builds on Katzenstein’s shelter logic; extends beyond economics (pp.100,111)	Shifts from alliances to law-based multilateral shelter; existential over military (pp.108–110)	Table maps EU coverage; Lisbon clauses relevant to solidarity (pp.105,107)	Exploit EU soft-security shelter while managing identity risks through safeguards (pp.110–112)

\section*{Source Analysis — \textit{Bailes \& Thorhallsson 2013}, Instrumentalizing the European Union in Small State Strategies}

\textbf{Describe:} Argues the EU gives small European states soft-security options and broader ‘shelter’, altering agendas and discourse (abstract, p.99). 

\textbf{Interpret:} Directly addresses whether small states can instrumentalise EU institutions to offset vulnerabilities relevant to an Irish DSS brief. 

\textbf{Methodology:} Conceptual synthesis using strategy texts, secondary literature, observed examples; no new dataset, doctrinal in tone (pp.101,105). 

\textbf{Evaluate:} Shows EU’s soft-security depth and existential effects; notes costs in sovereignty pooling identity pressures (pp.109–111).  

\textbf{Author:} Small-states scholars in Icelandic context; stance foregrounds EU institutional logics with balanced caution on costs. [NO SOURCE]

\textbf{Synthesis:} Aligns with Katzenstein on shelter effects; diverges from classic alliance views by stressing non-zero-sum, law-based integration (pp.100,108–110).  

\textbf{Limit.} Lacks granular Irish cases or operational guidance for defence actors. [NO SOURCE]\\
\textbf{Implication:} Irish DF should treat EU soft-security regimes as primary shelter while guarding autonomy in identity-sensitive areas.

\textbf{Method weight:} 3 — Observational/doctrinal synthesis; theory-led, no new empirical dataset.

\textbf{Claims-cluster seeds}
\begin{enumerate}
	\item \textit{Claim:} EU membership offers small states existential soft-security shelter. \textit{Best line with page:} “unique soft security features… ‘escape from smallness’” (p.111). \textit{Rival reading:} Shelter overstated where NATO dominates. \textit{Condition:} Low imminent military threat. \textit{Irish DF implication:} Prioritise EU civil protection, health, border regimes.
	\item \textit{Claim:} Costs include sovereignty pooling and identity erosion risks. \textit{Best line with page:} “unprecedented degree of pooling… identity erosion” (pp.109–110). \textit{Rival reading:} Europeanisation strengthens resilience. \textit{Condition:} High domestic contestation. \textit{Irish DF implication:} Safeguard niches, maintain consent.
	\item \textit{Claim:} Europeanisation pressures alignment even pre-accession. \textit{Best line with page:} Accession “puts pressure… to accept norms and goals” (p.112). \textit{Rival reading:} Opt-outs preserve choice. \textit{Condition:} Treaty flexibility. \textit{Irish DF implication:} Use opt-outs sparingly, shape norms early.
	\item \textit{Claim:} Soft-security coordination beats bilateral ‘big-power’ shelter. \textit{Best line with page:} Collective solutions more efficient than ad hoc aid (p.111). \textit{Rival reading:} Bilateralism is faster. \textit{Condition:} Transboundary risks. \textit{Irish DF implication:} Invest in EU mechanisms over one-off deals.
\end{enumerate}

\textbf{PEEL-C (for)}\\
\textit{Point:} The EU functions as Ireland’s principal soft-security shelter, reducing vulnerability across borders, health and economic shocks.\\
\textit{Evidence:} The authors argue EU pooled assets, regulation and emergency aid disproportionately benefit small states, enabling impact on climate and crises (p.111).\\ 
\textit{Explain:} A rules-based single market, harmonised standards and funds compress transaction costs, amplify voice and stabilise expectations for small actors, improving national resilience.\\
\textit{Limit.} Shelter relies on compliance capacity and domestic legitimacy.\\
\textit{Consequent:} Irish DF should deepen EU civil protection links, stress test interdependencies, and use Brussels to convene partners quickly.

\textbf{PEEL-C (against)}\\
\textit{Point:} EU shelter can dilute small-state identity and constrain bespoke responses when crises trigger one-size-fits-all measures.\\
\textit{Evidence:} Sovereignty pooling and exposure to EU-wide policies raise cost–benefit tensions for states without prior experience of certain threats (pp.109–110, 112).\\  
\textit{Explain:} Centralised norms may override valued liberties or crowd national bandwidth, reducing agility in niche areas that matter to Dublin.\\
\textit{Limit.} Opt-outs and consensus in strategic areas preserve room for manoeuvre.\\
\textit{Consequent:} Irish DF should ring-fence niche capabilities, pre-negotiate caveats, and pair EU compliance with national red-team reviews.

% Second Evidence and Implication Log
\begin{tabular}{p{3.2cm}p{4.2cm}p{3.6cm}p{3.2cm}p{4.2cm}}
	\textbf{Claim} \& \textbf{Best source (page)} \& \textbf{Rival source/reading} \& \textbf{Condition} \& \textbf{Implication for Irish DF}\\\hline
	EU soft-security shelter matters most \& Bailes \& Thorhallsson (p.111) \& NATO focus downplays EU \& Low military threat \& Prioritise EU civil protection, frontier tech \\
	Costs: sovereignty pooling, identity risk \& Bailes \& Thorhallsson (pp.109–110) \& Europeanisation as resilience \& High contestation \& Maintain public consent, protect niches \\
	Europeanisation pressures alignment \& Bailes \& Thorhallsson (p.112) \& Opt-out politics \& Flexible treaty areas \& Engage early in norm-setting \\
	Collective beats bilateral on transboundary risks \& Bailes \& Thorhallsson (p.111) \& Bilateral agility \& Urgent episodic crises \& Use EU hubs, pre-arrange surge MOUs \\
\end{tabular}

\textbf{Gaps}\\
Chase: recent Irish case studies applying EU civil protection, health security, cyber response.\\
Park: deep military integration debates unless mandate widens.

\parencite{BAILES_2013}

\section*{Source Analysis — \textit{Bailes, Thorhallsson \& Johnstone 2013}, Scotland as an Independent Small State: Where would it seek shelter?}
\textbf{Describe:} Study argues an independent Scotland would require strategic, political, economic and societal shelter, with options in EU, NATO, rUK, Nordics and the US (p.1).\\
\textbf{Interpret:} Directly relevant to a DSS brief on small-state security by mapping shelter mixes, costs and neighbour dependencies for a proximate case.\\
\textbf{Methodology:} IR-led conceptual synthesis; frames options via small-state theory, not advocacy on the referendum; descriptive scenario analysis (pp.3–4).\\
\textbf{Evaluate:} Persuasive on EU/NATO logic, and on rUK/US centrality; Arctic and Nordic lenses add texture; empirical granularity is thin (pp.11–17).\\
\textbf{Author:} Senior scholars in Iceland and Akureyri; small-state and Nordic expertise shapes focus on multilevel shelter.\\
\textbf{Synthesis:} Aligns with classic claims that small states seek alliances and institutional protection; extends to soft-security and societal shelter (pp.3–6).\\
\textbf{Limit.} Lacks post-2014 developments or Irish cases to test transferability. [NO SOURCE]\\
\textbf{Implication:} Ireland should treat EU soft-security and NATO partnerships as core shelter while managing UK interdependence politically and operationally.

\textbf{Method weight:} 3 — Observational/doctrinal synthesis without new dataset; comparative but non-experimental.

\textbf{Claims-cluster seeds}
\begin{enumerate}
	\item \textit{Claim:} EU and NATO are logical shelters for a new small state. \textit{Best line with page:} Conclusion stresses EU/NATO as mainstream post-1990 strategy (p.17). \textit{Rival reading:} Neutrality with ad hoc coalitions. \textit{Condition:} Low direct threat; rules-based order. \textit{Irish DF implication:} Prioritise EU civil protection and NATO partnerships.
	\item \textit{Claim:} rUK and US would be pivotal to viability. \textit{Best line with page:} rUK as primary shelter; US underwriting and brokerage (pp.13, 17). \textit{Rival reading:} Diverse partners hedge UK risk. \textit{Condition:} Cooperative London–Edinburgh ties. \textit{Irish DF implication:} Sustain UK interoperability while deepening EU–NATO links.
	\item \textit{Claim:} Shelter has autonomy costs. \textit{Best line with page:} Small states ‘pay’ in reduced freedom of manoeuvre (p.7). \textit{Rival reading:} Europeanisation builds resilience. \textit{Condition:} Domestic consent. \textit{Irish DF implication:} Ring-fence niches, maintain legitimacy.
	\item \textit{Claim:} Nordic ties add soft-security and identity shelter. \textit{Best line with page:} Nordic cooperation offers political cover, soft-security gains, NORDEFCO learning (p.14). \textit{Rival reading:} Limited strategic heft. \textit{Condition:} Shared Arctic and societal agendas. \textit{Irish DF implication:} Build Nordic working groups on SAR and resilience.
\end{enumerate}

\textbf{PEEL-C (for)}\\
\textit{Point:} EU and NATO compose the most efficient shelter mix for a small European state seeking continuity and capacity in hard and soft security.\\
\textit{Evidence:} The authors conclude EU/NATO membership is the logical, mainstream path for new small states in Europe since 1990 (p.17).\\
\textit{Explain:} EU rules, funds and operational regimes reduce exposure to shocks, while NATO’s collective defence and specialisation let limited forces contribute credibly. This integrated mix spreads risk and amplifies voice without heavy duplication.\\
\textit{Limit.} Benefits rely on implementation capacity and steady public consent.\\
\textit{Consequent:} Irish DF should deepen EU civil protection and NATO partnership projects to hedge transboundary risks with scarce resources.

\textbf{PEEL-C (against)}\\
\textit{Point:} Shelter can constrain bespoke responses and tighten dependence on larger neighbours, creating identity and autonomy frictions in crisis.\\
\textit{Evidence:} Shelter ‘price’ includes reduced freedom of manoeuvre, mirrored in EU and NATO participation costs (p.7).\\
\textit{Explain:} Standardised norms, budget ceilings and pooled capabilities may dull domestic priorities or crowd out niches, especially when London’s preferences dominate the neighbourhood ecosystem.\\
\textit{Limit.} Opt-outs, caveats and niche specialisation maintain space to manoeuvre.\\
\textit{Consequent:} Irish DF should protect niche capabilities, pre-negotiate caveats, and pair EU/NATO commitments with national red-team reviews.

\textbf{Evidence and Implication Log}
\begin{tabular}{p{3.2cm}p{4.2cm}p{3.6cm}p{3.2cm}p{4.2cm}}
	\textbf{Claim} \& \textbf{Best source (page)} \& \textbf{Rival source/reading} \& \textbf{Condition} \& \textbf{Implication for Irish DF}\\\hline
	EU–NATO shelter is mainstream \& Bailes et al. (p.17) \& Neutrality plus ad hoc coalitions \& Low direct threat \& Prioritise EU soft-security, NATO partnerships\\
	rUK/US pivotal \& Bailes et al. (p.13) \& Diversify partners \& Cooperative neighbours \& Maintain UK ties, hedge via EU–NATO\\
	Shelter has autonomy costs \& Bailes et al. (p.7) \& Europeanisation builds resilience \& Domestic consent \& Communicate trade-offs, protect niches\\
	Nordic soft-security value \& Bailes et al. (p.14) \& Limited strategic heft \& Shared agendas \& Leverage Nordic fora for SAR, resilience\\
\end{tabular}

\textbf{Gaps}\\
Chase: Concrete Irish cases linking EU civil protection, NATO PfP, and UK coordination in recent crises.\\
Park: Deep Arctic Council status debates unless Irish remit widens.

\parencite{BESSNER_2015}

\section*{Source Analysis — \textit{Bessner \& Guilhot 2015}, How Realism Waltzed Off}
\textbf{Describe:} The article explains why Waltz moved from classical realism to neorealism, arguing that neorealism reconceives realism in liberal form by excluding decisionmaking (p. 88).\\
\textbf{Interpret:} It is directly relevant to a question about agency in IR theory, showing how neorealism sidelines policy choice to align with liberal democratic norms (pp. 106–107, 117).\\
\textbf{Methodology:} Intellectual history and close reading, situating Waltz among cybernetics \& system theory, contrasting him with Morgenthau and Lippmann (pp. 108–109, 104–106).\\
\textbf{Evaluate:} Strong on genealogy and concepts; contribution is the cybernetic reading of neorealism; weaker on archival corroboration; the democracy defence is judged thin (p. 117).\\
\textbf{Author:} Both authors read Waltz as liberal in outlook; they counter classical realist elitism and highlight U.S. political science context (pp. 88–90, 104–106).\\
\textbf{Synthesis:} Aligns with Shklar on decisionism’s tie to systems thinking; diverges from Morgenthau’s decisionist pedagogy centred on statesmen (pp. 107–109, 90–94).\\
\textbf{Limit.} The core claims rest on interpretive synthesis more than primary archives.\\
\textbf{Implication:} For a small state or the Irish Defence Forces, structural reading helps orientation, but operational doctrine must reinsert decisionmaking.

\textbf{Method weight:} 3 — Observational/doctrinal synthesis using textual evidence, no triangulated empirical test.

\textbf{Claims–cluster seeds}
\begin{enumerate}
	\item \textit{Claim:} Neorealism removes decisionmaking to reconcile realism with liberalism. \textit{Best line:} “Waltz … circumvented entirely the problem of decisionmaking … a self-contained system” (p. 106). \textit{Rival reading:} Pure theoretical maturation. \textit{Condition:} When systemic adaptation dominates outcomes. \textit{Irish DF implication:} Keep structural scanning, but retain commander’s decision doctrine.
	\item \textit{Claim:} Cybernetics \& system theory underpin neorealism over rational choice. \textit{Best line:} Systems “moved away from notions of decision and choice … alternative to rational choice” (p. 109). \textit{Rival reading:} Microeconomics is the model. \textit{Condition:} High organisational complexity. \textit{Implication:} Use systems-informed red-teaming, but preserve decision cells.
	\item \textit{Claim:} Democracies can do foreign policy as effectively as authoritarian regimes. \textit{Best line:} “Democratic governments … are well able to compete” (p. 106, citing Waltz 1967). \textit{Rival reading:} Classical realist scepticism of publics. \textit{Condition:} Mature bureaucratic routines. \textit{Implication:} Whole-of-government processes can deliver strategic coherence.
	\item \textit{Claim:} Bipolarity stabilises by simplifying balancing. \textit{Best line:} “Bipolar systems … are more stable … simplify balancing” (p. 115). \textit{Rival reading:} Multipolar finesse prevents war. \textit{Condition:} Two dominant poles. \textit{Implication:} Small states should anchor in alliances, calibrate autonomy prudently.
\end{enumerate}

\textbf{PEEL-C (for)}\\
\textit{Point:} Neorealism’s power is to discipline analysis by treating international outcomes as systemic adaptations, limiting overconfidence in leaders’ agency.\\
\textit{Evidence:} Bessner \& Guilhot show Waltz “developed a model … entirely detached from foreign policy” to bypass decisionmaking (p. 106).\\
\textit{Explain:} That move renders many policy errors less about individual misjudgement than about positional constraints in the structure, a useful corrective for small states.\\
\textit{Limit.} System focus can obscure windows where deft decision can bend constraints.\\
\textit{Consequent:} Irish planners should fuse structural indicators with decision-focused exercises, ensuring commanders rehearse choices within, and occasionally against, structural pressure.

\textbf{PEEL-C (against)}\\
\textit{Point:} Removing decisionmaking risks blinding practitioners to the levers that remain under human control, especially for small states leveraging niches.\\
\textit{Evidence:} The authors concede neorealism “saved democracy by making it inconsequential … the system would take care of itself” (p. 117).\\
\textit{Explain:} If democracy is operationally irrelevant, doctrine may neglect political mobilisation, lawfare, or coalition entrepreneurship that alter payoffs even in tight structures.\\
\textit{Limit.} The critique targets a stylised neorealism, not all structural analysis in practice.\\
\textit{Consequent:} The Irish DF should institutionalise decision wargames alongside structural assessments, preserving agility to exploit fleeting opportunities in alliance politics.

\textbf{Evidence \& Implication Log}
\begin{tabular}{p{3.2cm}p{4.2cm}p{3.6cm}p{3.2cm}p{4.2cm}}
	\textbf{Claim} \& \textbf{Best source (page)} \& \textbf{Rival source/reading} \& \textbf{Condition} \& \textbf{Implication for Irish DF}\\\hline
	Neorealism removes decisionmaking \& Bessner \& Guilhot (p. 106) \& Theory matured, not ideological \& System dominates \& Blend structural intel with decision drills\\
	Cybernetics underpins Waltz \& Bessner \& Guilhot (p. 109) \& Microeconomics model \& High complexity \& Use systems thinking, keep command judgement\\
	Democracies can perform \& Bessner \& Guilhot (p. 106) \& Classical realist scepticism \& Mature bureaucracy \& Whole-of-govt mechanisms for strategy\\
	Bipolarity stabilises \& Bessner \& Guilhot (p. 115) \& Morgenthau’s finesse view \& Two clear poles \& Anchor in alliances, manage autonomy\\
\end{tabular}

\textbf{Gaps}\\
Chase: Primary archival evidence on Waltz’s adoption of cybernetic sources beyond textual markers.\\
Park: Fine-grained econometric tests of neorealism’s systemic predictions for Cold War dyads.

\parencite{BETTS_1996}

\section*{Source Analysis — \textit{Betts 1996}, The downside of the cutting edge}
\textbf{Describe:} Betts argues an RMA benefits the United States yet breeds overconfidence, professional complacency and heightens risks of miscalculation and escalation.\\
\textbf{Interpret:} This frames how small states should temper tech enthusiasm with strategy, budgeting and escalation control, not gadgeteering alone.\\
\textbf{Methodology:} Policy essay using Gulf War imagery, Vietnam learning failures and NATO first-use dilemmas to reason about strategy–technology fit.\\
\textbf{Evaluate:} Contribution is a crisp triad—expectations, complacency, instability—and a warning about “tactical clarity, strategic obscurity.” Evidence is illustrative, not systematic.\\
\textbf{Author:} Betts, a Columbia security scholar writing in \textit{The National Interest}, brings a U.S. vantage point that prizes strategic prudence.\\
\textbf{Synthesis:} Aligns with Cohen’s obscurity caution and Krepinevich’s critique of the Army Concept, diverging from triumphalist RMA narratives.\\
\textbf{Limit.} The piece generalises from cases without triangulated testing.\\
\textbf{Implication:} Irish defence should pair precision investment with mass, resilience and escalation management, and retain competence for messy low-tech fights.

\textbf{Method weight:} 3 — Observational doctrinal analysis from cases and strategic reasoning, no empirical triangulation.

\textbf{Claims–cluster seeds}
\begin{enumerate}
	\item \textit{Claim:} Public “bloodless war” expectations rise after precision spectacles. \textit{Best line with page:} “Laser-guided bombs… belief that war can be bloodless” [n.p.]. \textit{Rival reading:} RMA deters so expectations don’t matter. \textit{Condition:} After highly curated combat imagery. \textit{Irish DF implication:} Manage narratives, protect readiness funding.
	\item \textit{Claim:} RMA can entrench big-war orthodoxy and ill-suit unconventional conflicts. \textit{Best line with page:} “Commitment to high-tech operations may prove unsuitable… unpleasant choices” [n.p.]. \textit{Rival reading:} Tech lifts all missions. \textit{Condition:} Urban, irregular or hybrid contexts. \textit{Implication:} Train low-tech skills alongside precision.
	\item \textit{Claim:} Conventional superiority can push great-power adversaries toward WMD escalation. \textit{Best line with page:} “Decided technical advantage… loser desperate… unconventional weapons” [n.p.]. \textit{Rival reading:} Advantage deters escalation. \textit{Condition:} Adversary sees vital stakes. \textit{Implication:} Embed escalation control in planning.
	\item \textit{Claim:} Adversaries will develop asymmetric counters; last-move fallacy misleads planners. \textit{Best line with page:} “Asymmetrical solutions… counters to American technological prowess” [n.p.]. \textit{Rival reading:} Overmatch nullifies adaptation. \textit{Condition:} Cheap countermeasures proliferate. \textit{Implication:} Invest in deception resilience and rapid adaptation.
\end{enumerate}

\textbf{PEEL-C (for)}\\
Point: Treat RMA as a strategic asset only if paired with prudence about adversary incentives and domestic politics.\\
Evidence: Betts warns decisive conventional overmatch can push a weaker great power toward nuclear or biological escalation (n.p.).\\
Explain: Precision dominance raises an opponent’s temptation to escape defeat by upping the ante, which alters thresholds for coercion and compellence.\\
Limit. This logic bites hardest when the adversary’s stakes are local and vital.\\
Consequent: Ireland should privilege escalation control, dispersal and hardened C2 alongside modest precision buys.

\textbf{PEEL-C (against)}\\
Point: Overstating RMA downsides risks underinvesting in tools that deter and shorten wars.\\
Evidence: Betts concedes an RMA offers an “important net advantage” if integrated with strategy (n.p.).\\
Explain: Superior ISR–strike can coerce early, limit attrition and avert prolonged contests that favour larger powers.\\
Limit. Capability without readiness, stockpiles and doctrine yields brittle superiority.\\
Consequent: A small state should buy cheap precision enablers, but hedge with magazines, mobilisation and allied interoperability.

\textbf{Evidence \& Implication Log}
\begin{tabular}{p{3.2cm}p{4.2cm}p{3.6cm}p{3.2cm}p{4.2cm}}
	\textbf{Claim} \& \textbf{Best source (page)} \& \textbf{Rival source/reading} \& \textbf{Condition} \& \textbf{Implication for Irish DF}\\\hline
	Bloodless-war expectations follow precision imagery \& Betts 1996 [n.p.] \& RMA deterrence makes this moot \& Post-spectacle politics \& Protect readiness budgets, message costs\\
	RMA can misfit irregular wars \& Betts 1996 [n.p.] \& Tech applies across missions \& Urban/hybrid fights \& Drill low-tech, civil–military integration\\
	Overmatch may spur WMD escalation \& Betts 1996 [n.p.] \& Advantage deters escalation \& Adversary sees vital stakes \& Build escalation ladders, harden C2\\
	Asymmetric counters erode advantage \& Betts 1996 [n.p.] \& Overmatch prevents adaptation \& Cheap countermeasures \& Invest in deception resilience, rapid adaptation\\
\end{tabular}

\textbf{Gaps}\\
Chase: Cases quantifying RMA-driven escalation risks in great-power crises since 1991.\\
Park: Detailed cost–effectiveness of specific Irish precision platforms pending requirements.

\section*{Source Analysis — \textit{Commission on the Defence Forces 2022}, Report of the Commission on the Defence Forces}
\textbf{Describe:} Establishes three Levels of Ambition, identifies a resource–ambition gap, and specifies 2030 outputs: complete RAP, nine ships with double crewing, ~2,000 patrol days (pp. 36–40).\\
\textbf{Interpret:} Directly frames Ireland’s 2030 choice; lists are indicative, not prescriptive, and omit precise costings (pp. 39–40, 53).\\
\textbf{Methodology:} Commission synthesis of submissions and comparative assessment; observational, doctrinal validity within institutional bounds.\\
\textbf{Evaluate:} Strong on concrete naval outputs and RAP urgency; weaker on fiscal phasing and conventional sea warfighting at LOA 2 (pp. 36–37).\\
\textbf{Author:} Reformist institutional stance seeking persistent capability uplift with external oversight.\\
\textbf{Synthesis:} Aligns with White Paper on sovereignty needs; diverges by formalising LOA tiers and quantifying outputs (pp. 36–40).\\
\textbf{Limit.} Recommendations are indicative; delivery routes may vary.\\
\textbf{Implication:} Sequence RAP, fleet renewal and double crewing to convert presence into persistent awareness by 2030 for a small state.

\section*{Source Analysis — \textit{Cottey 2022}, A Celtic Zeitenwende? Continuity and Change in Irish National Security Policy}
\textbf{Describe:} Argues continuity will dominate despite Ukraine; four drivers persist—low threat, free-riding, domestic constraints, EU good citizenship (pp. 2–3, 7–9).\\
\textbf{Interpret:} Sets realistic DSS expectations to 2030; informs pacing for LOA 2 under political caution (pp. 2–3).\\
\textbf{Methodology:} Peer-reviewed analytic synthesis using budgets, polls and official texts; observational rather than experimental.\\
\textbf{Evaluate:} Persuasive on political economy and neutrality; lighter on small-state comparators and fresh empirical data (pp. 5–9).\\
\textbf{Author:} Academic voice, cautious on prospects for rapid change; no declared funding.\\
\textbf{Synthesis:} Converges with CODF on low conventional risk and cyber salience; diverges by tempering expectations of transformation (pp. 2–5, 7–9).\\
\textbf{Limit.} Single-author synthesis anchored in 2022 evidence.\\
\textbf{Implication:} Prioritise RAP, crewing and cyber first to bank steady gains without breaching neutrality norms.

\section*{Source Analysis — \textit{Cohen 2002}, Supreme Command in the 21st Century}
\textbf{Describe:} Critiques “normal theory” and advances active civilian control as an unequal dialogue; Churchill’s injunction to probe guides oversight (n.p.).\\
\textbf{Interpret:} Provides behavioural rules for ministers to steer capability delivery and test advice during CODF implementation (n.p.).\\
\textbf{Methodology:} Doctrinal essay with historical vignettes; argumentative, not empirical.\\
\textbf{Evaluate:} Useful for oversight behaviours and candid dialogue; US-centric and light on small-state specifics.\\
\textbf{Author:} U.S. strategic studies scholar advocating assertive civilian leadership; no small-state institutional lens.\\
\textbf{Synthesis:} Supports probing civilian oversight of LOA 2 delivery; diverges by focusing on wartime command rather than peacetime planning.\\
\textbf{Limit.} Sparse small-state tailoring and operational metrics.\\
\textbf{Implication:} Institutionalise unequal dialogue in Irish defence governance—probe hard in private, decide, then support in public.

\parencite{DUMAN_2025} 

\section*{Source Analysis — \textit{Duman \& Rakipoğlu 2025}, The Structural Paralysis of the UN Security Council: Great Power Politics and the Gaza Crisis}
\textbf{Describe:} Examines UNSC response to Gaza, Oct 2023–Jan 2025. Thirteen drafts, four adopted. Core claim: veto power, especially US use, paralysed decisive action and weakened adopted texts (pp. 46, 53–56).\\
\textbf{Interpret:} Relevant to DSS on institutions under strain. Shows humanitarian imperatives subordinated to alliances and P5 narratives; E10 solidarity could not overcome structural veto (pp. 66–68).\\
\textbf{Methodology:} Qualitative content analysis of P5 statements across twelve meetings, coding UN drafts, votes, and transcripts in the Oct 2023–Jan 2025 window; interpretivist, record-based validity with clear scope conditions (pp. 48–49).\\
\textbf{Evaluate:} Contribution: resolution-by-resolution tracing that links veto justifications to outcomes; documents how adopted texts were diluted by negotiation and legal reinterpretation (pp. 46, 62–63).\\
\textbf{Author:} Authors are Turkish academics; the narrative references Ankara’s critique that the world is “bigger than five,” signalling a reformist lens (p. 53).\\
\textbf{Synthesis:} Aligns with long-running calls for curbing veto use in mass atrocity contexts; diverges from great-power management logics that normalise veto as stabiliser (pp. 66–68).\\
\textbf{Limit.} Single case, discourse-heavy evidence, limited assessment of ground effects beyond texts.\\
\textbf{Implication:} Irish DF should coalition with E10 on text-crafting, back veto-restraint codes in atrocity cases, and resource humanitarian access diplomacy.

\textbf{Method weight:} 3/5. Solid primary records and clear design, yet single-case scope, interpretivist bias risk, limited outcome validation against field effects.

\textbf{Claims-cluster seeds}
\begin{itemize}
	\item \textbf{Claim:} US vetoes were the principal barrier to decisive Council action.\\
	Best line: “The United States’ veto power stands at the centre of this deadlock” (pp. 46–47).\\
	Rival reading: Veto preserved leverage for hostage diplomacy and de-escalation sequencing.\\
	Condition: Holds when ceasefire text omits language demanded by Washington.\\
	Irish DF implication: Work E10–US bridges early; pre-consult to avoid veto triggers.
	\item \textbf{Claim:} Even adopted resolutions were substantially weakened by negotiation and reinterpretation.\\
	Best line: “Even resolutions that passed were substantially weakened through political compromises and legal reinterpretations” (p. 46).\\
	Rival reading: Dilution was necessary to build minimum consensus for any operational effect.\\
	Condition: Late-stage text where hostage and ceasefire linkages are unresolved.\\
	Irish DF implication: Insert safeguard clauses and reporting mandates that survive dilution.
	\item \textbf{Claim:} E10 agency was limited despite unity, given P5 narrative dominance.\\
	Best line: E10 unity on 2728 could not offset P5 privileges and narrative control (pp. 66–68).\\
	Rival reading: E10 succeeded tactically with 2728; incrementalism is the rational path.\\
	Condition: When a P5’s ally is a direct belligerent.\\
	Irish DF implication: Build cross-regional E10 blocs early; circulate joint interpretive statements.
	\item \textbf{Claim:} Curtailing veto use in mass atrocity cases is central to meaningful reform.\\
	Best line: Reform must curb veto abuse in atrocity situations to restore legitimacy (p. 68).\\
	Rival reading: Veto curtailment is utopian; procedural tweaks and working methods matter more.\\
	Condition: When casualty thresholds and humanitarian access metrics cross agreed triggers.\\
	Irish DF implication: Back veto-restraint code commitments and atrocity-triggered explanations.
\end{itemize}

\textbf{Evidence \& Implication Log}
\begin{tabular}{p{3.2cm}p{4.2cm}p{3.6cm}p{3.2cm}p{4.2cm}}
	\textbf{Claim} \& \textbf{Best source (page)} \& \textbf{Rival source/reading} \& \textbf{Condition} \& \textbf{Implication for Irish DF}\\\hline
	US veto central \& Duman \& Rakipoğlu (pp. 46–47) \& Veto preserves negotiation leverage \& US red-lines unmet \& Pre-consult with US; sequence text to hostage tracks\\
	Adopted texts weakened \& Duman \& Rakipoğlu (p. 46) \& Dilution as consensus-building \& Late-stage bargaining \& Insert durable humanitarian clauses and reporting\\
	E10 agency constrained \& Duman \& Rakipoğlu (pp. 66–68) \& Incrementalism succeeds via 2728 \& P5 ally engaged \& Build cross-regional E10 coalitions; joint statements\\
	Veto curtailment needed \& Duman \& Rakipoğlu (p. 68) \& Working-methods reform preferable \& Atrocity thresholds crossed \& Support veto-restraint codes and atrocity triggers\\
\end{tabular}

\textbf{Gaps}\\
• What to chase: Triangulate Council discourse with operational impact metrics on aid flows and civilian harm.\\
• What to park: Broader UN reform literature beyond Gaza until core argument is drafted.


EU defence instruments consolidate strategic autonomy.\\
Best line (p.07–08): EDF “endowed with EUR 13 billion… enhance EU strategic autonomy.”\\
Rival: Budgets signal intent more than capability.\\
Condition: Coherence across PESCO, EDIDP, EDF, EPF.\\
Irish DF implication: Use PESCO mobility and co-fund capability gaps.

Military mobility makes EU–NATO complementarity practical.\\
Best line (p.07): Mobility plan eases cross-border movement and “will also benefit… NATO.”\\
Rival: Legal and infrastructure barriers persist.\\
Condition: National implementation and Schengen–defence coordination.\\
Irish DF implication: Prioritise dual-use infrastructure and movement procedures.

Integrated approach links internal and external for resilience.\\
Best line (pp.06, 16): Joined-up work on migration, cyber, counterterrorism; external budget +30 percent.\\
Rival: Risks securitising development.\\
Condition: Safeguards on rights and humanitarian principles.\\
Irish DF implication: Embed justice and policing linkages in mission design.

Civilian CSDP compact strengthens crisis management.\\
Best line (p.08): Compact parameters due by year-end; focus on police, rule of law, civil admin.\\
Rival: Staffing shortfalls limit delivery.\\
Condition: Member State rosters and training pipelines.\\
Irish DF implication: Expand civilian deployments and training cadres.

External budget reform boosts strategic flexibility (NDICI).\\
Best line (p.16): NDICI proposal to match funding with priorities and react swiftly.\\
Rival: Centralisation may dilute accountability.\\
Condition: Clear conditionality and evaluation.\\
Irish DF implication: Align DF engagement with NDICI windows.

\textbf{PEEL-C (for)}\\
\textit{Point:} EU defence instruments now operationalise strategic autonomy.\\
\textit{Evidence:} PESCO launched with 25 Member States and 17 projects, backed by EDIDP €500m, EDF €13bn and a proposed EPF €10.5bn. Military mobility enables rapid cross-border movement and benefits NATO.\\
\textit{Explain:} Instruments, funding and planning coherence shift the Union from declarations to capability pathways.\\
\textit{Limit:} Reporting lists inputs not outcomes.\\
\textit{Consequent:} Irish DF should target mobility, pooled procurement and mission enablers.

\textbf{PEEL-C (against)}\\
\textit{Point:} Without coherence, new tools risk symbolism.\\
\textit{Evidence:} The report itself warns delivery must follow and initiatives require coherence across pillars.\\
\textit{Explain:} Dispersed projects can fragment capacity if evaluation, staffing and legal fixes lag.\\
\textit{Limit:} Some coherence steps are in train via NDICI and internal–external linkages.\\
\textit{Consequent:} Irish DF should hardwire evaluation, legal clearances and civil–military pipelines into participation.

\textbf{Evidence \& Implication Log}
\begin{tabular}{p{3.2cm}p{4.2cm}p{3.6cm}p{3.2cm}p{4.2cm}}
	\textbf{Claim} \& \textbf{Best source (page)} \& \textbf{Rival source/reading} \& \textbf{Condition} \& \textbf{Implication for Irish DF}\\\hline
	EU instruments drive strategic autonomy \& EDF €13bn; EDIDP €500m; EPF €10.5bn (pp.07–08) \& Budgets without delivery \& Coherence across PESCO–EDF–EPF \& Focus on enablers, pooled buys\\
	Mobility enables EU–NATO complementarity \& Mobility plan aids NATO (p.07) \& Infrastructure, legal bottlenecks \& National implementation \& Invest in dual-use routes, SOPs\\
	Integrated approach builds resilience \& Link internal–external; budget +30\% (pp.06,16) \& Securitisation risk \& Rights safeguards \& Pair policing with governance aid\\
	Civilian CSDP compact matters \& Focus on police, rule of law; compact due (p.08) \& Staffing shortfalls \& MS rosters \& training \& Expand civilian expert pools\\
	External budget reform boosts flexibility \& NDICI proposal (p.16) \& Centralisation may dilute accountability \& Clear conditionality/evaluation \& Align DF engagement with NDICI windows\\
\end{tabular}

\textbf{Gaps}\\
• Independent effectiveness evaluations of PESCO, EDIDP and EDF outcomes since 2018 [NO SOURCE].\\
• Irish DF participation detail in specific PESCO projects and mobility corridors [NO SOURCE].

\textbf{Two PEEL-C Paragraphs}

\textbf{Strongest claim —} \textit{Point:} States that control hubs can surveil or throttle network flows.\\
\textit{Evidence:} The article defines panopticon and chokepoint tied to hubs, then shows SWIFT enabling both after 9/11, with U.S. Treasury access via the Virginia mirror (pp. 55, 65–66).\\
\textit{Explain:} Hub centrality and jurisdiction turn efficiency into leverage. Information extraction exposes networks; exclusion compels policy shifts.\\
\textit{Limit:} The piece probes plausibility rather than testing scope conditions.\\
\textit{Consequent:} Irish Defence Forces should audit hub exposure in finance and data, then design operational bypasses.

\textbf{Counter —} \textit{Point:} Interdependence does not always yield coercion because not all sectors rest on asymmetric networks.\\
\textit{Evidence:} The authors note broader limits where markets are liquid or not network-centred, reducing control points (e.g., oil) (p. 74).\\
\textit{Explain:} Without hubs, leverage dissipates. Domestic institutions can also curb chokepoint use online.\\
\textit{Limit:} Liquidity can change if infrastructure recentralises.\\
\textit{Consequent:} Treat coercion risk as sector-specific and update routes as topology shifts.

\section*{Evidence \& Implication Log}
\begin{tabular}{p{3.2cm}p{4.2cm}p{3.6cm}p{3.2cm}p{4.2cm}}
	\textbf{Claim} \& \textbf{Best source (page)} \& \textbf{Rival source/reading} \& \textbf{Condition} \& \textbf{Implication for Irish DF}\\\hline
	Hubs enable panopticon and chokepoint \& Farrell \& Newman, p. 55 \& Liberal reciprocity limits coercion \& Few dominant hubs \& Map finance and data hub dependencies\\
	U.S. weaponises SWIFT for both effects \& Farrell \& Newman, pp. 65–66 \& EU privacy blocks sustained access \& Jurisdiction or allied consent \& Build sanctions-resilient payment options\\
	Internet centralisation fuels panopticon \& Farrell \& Newman, p. 73 \& Platform self-regulation substitutes chokepoints \& Legal authority over platforms \& Encrypt, segment, diversify routing\\
	Not all sectors are chokepointable \& Farrell \& Newman, p. 74 \& Future recentralisation \& Liquidity, low network reliance \& Prioritise sector-specific risk scans\\\hline
\end{tabular}

\textbf{Gaps}\\
• What to chase: Quantify Ireland’s exposure to SWIFT nodes, cable landings, and EU IXPs with route diversity options.\\
• What to park: Historical breadth on non-Western hub formation unless a DSS scenario demands it.

\parencite{EU_2019}

\section*{Source Analysis — \textit{EEAS 2019}, The European Union’s Global Strategy: Three Years On, Looking Forward}
\textbf{Describe:} The report takes stock of the 2016 EUGS, placing the security of the Union as the first priority, and details delivery on defence initiatives, missions, and multilateral action (pp. 10–12, 15).\\
\textbf{Interpret:} It is directly relevant to questions of small-state defence posture in Europe; it argues multilateralism is an existential interest and links EU credibility to collective capacity to act (pp. 9–10).\\
\textbf{Methodology:} An institutional self-assessment based on stakeholder consultations, programme launches, and operational summaries; validity is constrained by authorial stake and limited external counterfactuals (pp. 9–10).\\
\textbf{Evaluate:} Substantive delivery is evidenced through PESCO, the EDF, and mission data; EU–NATO cooperation is framed as mutually reinforcing, yet impact metrics are uneven across lines of effort (pp. 11–12).\\
\textbf{Author:} Produced under the High Representative/EEAS, it reflects a pro-integration, multilateral lens that favours strategic autonomy.\\
\textbf{Synthesis:} It aligns with the EU’s commitment to multilateral global governance and cooperative regional orders, positioning the Union as a trusted point of reference (p. 15).\\
\textbf{Limit.} Internal review with selective indicators and little independent evaluation.\\
\textbf{Implication:} Irish DF should leverage PESCO, military mobility, and maritime cooperation to scale effect through the Union.

\textbf{Method weight:} 3 — Credible institutional review with concrete programme evidence, but self-report bias and sparse external validation reduce robustness.

\textbf{Claims-cluster seeds}
\begin{itemize}
	\item \textbf{Claim:} EU has become a credible maritime security provider; piracy incidents fell and cooperation deepened (pp. 11–12).  
	Best line: Atalanta incidents fell from 176 to four failed attacks in 2018.  
	Rival: Credibility rests on NATO enablers.  
	Condition: Sustained C2, ISR, and logistics.  
	Irish DF implication: Prioritise naval interoperability, MSA, and joint training.
	\item \textbf{Claim:} PESCO and the EDF operationalise strategic autonomy.  
	Best line: PESCO provides a binding framework; EDF incentivises cooperative development (pp. 11–12).  
	Rival: Risk of duplication and fragmentation [NO SOURCE].  
	Condition: Interoperable, deployable capabilities.  
	Irish DF implication: Target value-adding PESCO projects.
	\item \textbf{Claim:} Multilateralism is an existential interest for the Union.  
	Best line: The rules-based order is an existential interest; multilateralism is acutely needed (p. 9).  
	Rival: Sovereigntist hedging limits EU clout [NO SOURCE].  
	Condition: Coalitions of the willing by issue.  
	Irish DF implication: Anchor DF roles in UN/EU frameworks.
	\item \textbf{Claim:} EU–NATO cooperation is mutually reinforcing, not zero-sum.  
	Best line: 74 common actions and hand-in-hand posture (p. 12).  
	Rival: Autonomy weakens NATO [NO SOURCE].  
	Condition: Mobility, cyber, maritime deliverables.  
	Irish DF implication: Plan for dual-use mobility upgrades.
\end{itemize}

\textbf{PEEL-C paragraphs}\\
\textit{Point.} The EU has converted the EUGS into tangible defence delivery that small states can leverage.\\
\textit{Evidence.} PESCO offers a binding framework for joint investment and readiness, while the EDF stimulates cooperative capability development (pp. 11–12).\\
\textit{Explain.} For Ireland, pooling in PESCO and tapping EDF lowers cost, raises interoperability, and channels influence.\\
\textit{Limit.} Institutional self-reporting and uneven metrics warrant caution.\\
\textit{Consequent:} Use PESCO and EDF selectively to amplify DF capability without strategic overreach.

\textit{Point.} EU credibility as a maritime security provider strengthens collective sea-lane security relevant to a trading island state.\\
\textit{Evidence.} Atalanta correlates with a drop from 176 piracy incidents to four failed attacks in 2018 and deeper EU–NATO maritime cooperation (pp. 11–12).\\
\textit{Explain.} Irish DF gains by focusing on maritime situational awareness, ISR sharing, and exercises.\\
\textit{Limit.} Capability gaps persist without sustained investment and mobility fixes.\\
\textit{Consequent:} Back EU maritime and mobility initiatives to secure trade routes at scale.

\section*{Evidence \& Implication Log}
\begin{tabular}{p{3.2cm}p{4.2cm}p{3.6cm}p{3.2cm}p{4.2cm}}
	\textbf{Claim} \& \textbf{Best source (page)} \& \textbf{Rival source/reading} \& \textbf{Condition} \& \textbf{Implication for Irish DF}\\\hline
	EU maritime provider \& Atalanta incidents fell to four (pp. 11–12) \& NATO enablers central [NO SOURCE] \& Sustained C2/ISR/logistics \& Invest in MSA and naval interoperability\\
	PESCO/EDF build autonomy \& PESCO binding framework; EDF incentives (pp. 11–12, 15) \& Duplication risk [NO SOURCE] \& Interoperable, deployable outputs \& Select high-value PESCO projects\\
	Multilateralism is existential \& Rules-based order is existential (p. 9) \& Sovereigntist hedge [NO SOURCE] \& Issue-based coalitions \& Anchor DF roles in UN/EU missions\\
	EU–NATO synergy \& 74 common actions; hand-in-hand (p. 12) \& Autonomy weakens NATO [NO SOURCE] \& Mobility, cyber, maritime delivery \& Plan dual-use mobility upgrades\\\hline
\end{tabular}

\textbf{Gaps}\\
• Independent evaluations of EU missions’ impact and cost-effectiveness; external audits [NO SOURCE].\\
• Park detailed annex metrics beyond immediate DF relevance.

PEEL-C drafting

Paragraph 1 — strongest claim:  
\textbf{Point.} Institutions rarely cause peace because they mirror underlying power.  
\textbf{Evidence.} Mearsheimer defines institutions as negotiated rule-sets and concludes they have minimal independent effect on state behaviour.  
\textbf{Explain.} If rules arise from great-power bargains, stability comes from the distribution of capabilities, not institutional design.  
\textbf{Limit.} Some regimes may still bite where verification is cheap.  
\textbf{Consequent.} Irish DF should prioritise credible niche forces and coalition interoperability over institutional faith.  

Paragraph 2 — counter:  
\textbf{Point.} Institutional design can still matter when trust exists and attribution is clear.  
\textbf{Evidence.} Mearsheimer shows collective security collapses without trust, but that exposes a conditional rather than absolute limit.  
\textbf{Explain.} Where a lonely aggressor is identifiable and preponderant power can mobilise quickly, institutions may coordinate action that deterrence alone would not.  
\textbf{Limit.} Such conditions are rare and fragile.  
\textbf{Consequent.} Irish DF should engage institutions to coordinate legitimacy, yet keep national readiness for gaps.  

\section*{Evidence \& Implication Log}  
\begin{tabular}{p{3.2cm}p{4.2cm}p{3.6cm}p{3.2cm}p{4.2cm}}  
	\textbf{Claim} \& \textbf{Best source (page)} \& \textbf{Rival source/reading} \& \textbf{Condition} \& \textbf{Implication for Irish DF}\\\hline  
	Institutions mirror power more than they shape it \& Mearsheimer 1994 (p.~[NO SOURCE]) \& Institutionalist reading [NO SOURCE] \& High relative-gains salience \& Build capability then use institutions to coordinate \\  
	Relative gains blunt cooperation \& Mearsheimer 1994 (p.~[NO SOURCE]) \& Iteration lowers cheating fears [NO SOURCE] \& Economic heft maps to military potential \& Invest in niches that convert resources to influence \\  
	Collective security is trust-dependent \& Mearsheimer 1994 (p.~[NO SOURCE]) \& Trust grows with practice [NO SOURCE] \& Lonely aggressor, rapid preponderance \& Keep bilateral hedges with UN mandates \\  
\end{tabular}  

Two PEEL-C paragraphs  

\textbf{Point:} Institutions mainly mirror power, so peace depends on capabilities and coalitions, not IO membership.  
\textbf{Evidence:} He defines institutions as rules yet concludes they have “minimal influence on state behavior,” reflecting distributions of power.  
\textbf{Explain:} If rules track power, then policy should weight state interests and relative capability over regime density.  
\textbf{Limit:} Discrete institutional effects short of “causing peace” may still cumulate.  
\textbf{Consequent:} Irish DF should use IOs to organise and legitimise efforts while investing in movement, lift and C3I enablers.  

\textbf{Point:} Cooperation falters on relative gains more than cheating, so distributive safeguards beat surveillance fixes.  
\textbf{Evidence:} The neat economy–security divide collapses under relative gains; OECD cases show distribution, not defection, driving outcomes.  
\textbf{Explain:} Where gains translate to power, partners measure splits, not just totals, limiting deep commitments.  
\textbf{Limit:} In low-threat settings, iteration and linkage can still unlock deals.  
\textbf{Consequent:} Irish DF should hard-code fair-share formulas and capability-for-access swaps into EU–UN engagements.  

\textbf{Gaps}\\  
• Verify original page numbers from \textit{International Security} pagination; add precise cites.\\  
• Park broader post-1994 EU–NATO practice until scoping DSS learning outcomes.\\  
• Post-1994 empirical tests of incremental institutional effects relevant to EU–UN missions [NO SOURCE].  

\parencite{MEARSHEIMER_2019}  

\section*{Source Analysis — \textit{Mearsheimer 2019}, Bound to Fail: The Rise and Fall of the Liberal International Order}  
\textbf{Describe:} Mearsheimer explains why the post–Cold War liberal order faltered then failed, arguing it was structurally doomed and would give way to realist orders in a multipolar system (pp.~7–8, 42–44).  
\textbf{Interpret:} The piece matters for DSS because it frames strategy for small states in renewed great-power rivalry and decoupling. It sidelines granular EU defence policy and Irish instruments, which we must supply.  
\textbf{Methodology:} A conceptual typology distinguishes international, bounded, agnostic, and ideological orders, supported by historical illustrations, not formal tests (pp.~11–15). Validity rests on parsimonious structure rather than exhaustive data.  
\textbf{Evaluate:} Strongest contribution is the clear distinction between bounded and international orders and why liberal orders require unipolarity (pp.~11–15). This clarifies policy claims that otherwise blur institutions with power.  
\textbf{Author:} A leading US realist at Chicago, sceptical of liberal hegemony’s feasibility and desirability, writing within the American strategic debate.  
\textbf{Synthesis:} Aligns with Walt’s caution on liberal overreach and Rosato’s power-politics reading of European integration, diverges from Ikenberry on liberal order durability and restraint (pp.~21–23).  
\textbf{Limit.} Assumes nationalism invariably trumps liberalism and that NATO enlargement was chiefly liberal integration, a contested reading (pp.~23–24).  
\textbf{Implication:} For Ireland, plan for dual bounded orders led by the US and China, use EU capability and UN legitimacy pragmatically, and avoid democratisation wars (pp.~42–44, 24). This aligns with module outcomes on assessing structure and evaluating policy.  

Method weight: 3 — Conceptual typology with illustrative cases offers clarity, yet limited falsifiability and small-state agency under-specification.  

PEEL-C drafting

Strongest claim paragraph.  
\textbf{Point:} The second phase of the RMA reshapes war by coupling cyber, robotics and precision against elites.  
\textbf{Evidence:} Metz argues that future conflict will target infrastructure and leadership while mass forces recede (p. 40).  
\textbf{Explain:} This shifts advantage to agile actors that fuse sensors, autonomy and legal framing, altering campaign design.  
\textbf{Limit:} The timeline for reliable precision and attribution is uncertain.  
\textbf{Consequent:} The DF should invest first in resilience and federated sensing to avoid strategic paralysis.  

Counter paragraph.  
\textbf{Point:} Conservative continuity may prevail, with exquisite platforms and classic phasing enduring.  
\textbf{Evidence:} Joint Vision era planning assumes technology enhances, rather than replaces, manoeuvre and fires.  
\textbf{Explain:} If stealth and speed outpace proliferation, small swarms remain ancillary and the operational level persists.  
\textbf{Limit:} This discounts adversary adaptation and infrastructure vulnerabilities.  
\textbf{Consequent:} The DF should hedge by pairing traditional readiness with low-cost cyber and drone capabilities.  

\textbf{Gaps}\\  
• Check precise pagination and key lines in Metz 2000 for quotation accuracy.\\  
• Park deeper comparative testing against post-2000 case studies until scoping approves SOURCES=VERIFY.  

\parencite{MINIHAN_2018}

\section*{Source Analysis — \textit{Minihan 2018}, Is there a future for United Nations Peacekeeping as presently constituted?}  
\textbf{Describe:} Minihan surveys UN peacekeeping’s trajectory and argues for time-bounded, evaluated missions, clearer mandates, and institutional reform, including Security Council veto discipline and Irish parliamentary oversight; he highlights UNIFIL’s constrained mandate and endorses a two–three-year horizon (pp.~121, 125, 128–129).  

\textbf{Interpret:} For Irish Defence Forces, this frames a pragmatic test: deploy where political strategy is credible, mandate is executable, and exit metrics exist; it brackets hard budget lines and legal constraints on reform.  

\textbf{Methodology:} Policy essay drawing on secondary sources and practitioner judgement; validity rests on transparent proposals, cited UN statements, and mission vignettes rather than systematic data (pp.~122–129).  

\textbf{Evaluate:} Most persuasive where he targets mission “shelf life,” lack of basic evaluation, and Guterres’ limit claim that peacekeeping only creates space for political solutions, justifying exit criteria (p.~128). Less developed are costs, force design, and law–policy frictions.  

\textbf{Author:} Capt (Retd) John Minihan, Irish practitioner voice in \textit{Defence Forces Review} (p.~121).  

\textbf{Synthesis:} Converges with standing critiques of vague mandates and Security Council paralysis; his UNIFIL reading complements “tripwire stability” but warns of mandate–means gaps (pp.~125, 129).  

\textbf{Limit.} Opinion-led, limited original empirics, Irish-centric focus.  

\textbf{Implication:} Ireland should codify two–three-year exit metrics, formalise Oireachtas oversight, and champion veto-discipline norms to match mandates with means.  

Method weight: 3 — Reasoned practitioner policy critique with cited UN positions, but limited original evidence and quantification.  

\section*{Claims–cluster seeds}  

Missions need two–three-year horizons with basic evaluation.\\
• Best line: “Peacekeeping… should… depart… all in a timeframe of two to three years” (p.~128).\\
• Rival reading: Longer deployments embed stability and learning (p.~125).\\
• Condition: Clear political strategy underpins mandate execution (p.~128).\\
• Irish DF implication: Build mission exit metrics into planning and cabinet submissions.\\

Security Council veto misuse hollows mandates; reform is required.\\
• Best line: Veto produces stalemate; reform menu offered (p.~129).\\
• Rival reading: Veto stabilises crises by forcing consensus.\\
• Condition: Code of conduct for veto in atrocity/mandate-setting cases (p.~129).\\
• Irish DF implication: Use UNSC bids to press veto-discipline tied to executable mandates.\\

UNIFIL demonstrates ‘tripwire’ stability but mandate–means gaps persist.\\
• Best line: No capability or will to confront armed actors; mandates ‘interpreted’ (p.~125).\\
• Rival reading: Relative calm evidences success of stabilisation (p.~125).\\
• Condition: External distractions reduce spoiler activity (p.~125).\\
• Irish DF implication: Prioritise force protection and liaison while advocating mandate clarity.\\

Political-military oversight must be strengthened at home.\\
• Best line: Need for competent, politically astute General Staff; routine Oireachtas debate (pp.~127–128).\\
• Rival reading: Existing structures suffice; risks politicisation.\\
• Condition: Bi-annual committee hearings with expert input (p.~128).\\
• Irish DF implication: Institutionalise pre-deployment scrutiny and after-action reviews.\\

\section*{PEEL-C drafting}  

Paragraph 1 — strongest claim\\
\textbf{Point.} UN missions should be time-bounded and routinely evaluated.  
\textbf{Evidence.} Minihan argues that peacekeeping must set clear objectives and “depart… in a timeframe of two to three years,” noting a “lack of basic evaluation” (p.~128).  
\textbf{Explain.} Exit metrics align military activity with political strategy and guard against mandate drift.  
\textbf{Limit.} Some theatres require longer stabilisation.  
\textbf{Consequent.} Irish DF should embed exit criteria and evaluation baselines in all mission plans.  

Paragraph 2 — counter\\
\textbf{Point.} Durable stability can emerge even under long, imperfect mandates.  
\textbf{Evidence.} Minihan concedes UNIFIL’s presence provides “tripwire” stability and a peaceful environment despite mandate constraints (p.~125).  
\textbf{Explain.} Extended presence can deter escalation while political tracks mature.  
\textbf{Limit.} Calm may rely on exogenous factors, not institutional design (p.~125).  
\textbf{Consequent.} Irish DF should balance time limits with conditions-based reviews rather than rigid deadlines.  

\section*{Evidence \& Implication Log}  
\begin{tabular}{p{3.2cm}p{4.2cm}p{3.6cm}p{3.2cm}p{4.2cm}}  
	\textbf{Claim} \& \textbf{Best source (page)} \& \textbf{Rival source/reading} \& \textbf{Condition} \& \textbf{Implication for Irish DF}\\\hline  
	Time-bounded missions with evaluation \& Minihan 2018 (p.~128) — two–three-year horizon \& Long deployments embed stability (UNIFIL calm) \& Clear political strategy and exit metrics \& Tie deployment to measurable political milestones\\  
	Veto reform for executable mandates \& Minihan 2018 (p.~129) — reform menu, code of conduct \& Veto forces consensus, prevents rash action \& Atrocity/mandate-setting code applies \& Use UNSC bids to press veto-discipline norms\\  
	UNIFIL shows tripwire stability but mandate–means gap \& Minihan 2018 (p.~125) — mandates ‘interpreted’ \& Stability evidences success (p.~125) \& Spoilers distracted externally (p.~125) \& Prioritise liaison, force protection, and push for clarity\\  
	Home oversight and strategy integration \& Need for politically astute General Staff; routine Oireachtas scrutiny (pp.~127–128) \& Existing structures adequate \& Bi-annual hearings with expert inputs \& Institutionalise pre-deployment reviews and AARs\\  
\end{tabular}  

\textbf{Gaps}\\  
• Chase empirical tests for two–three-year horizon and evaluation effects; map costs and force design trade-offs.\\  
• Park detailed China trajectory post-2018 and deep UNSC reform pathways for a separate brief.  

\parencite{NYE_2008}

\textbf{PEEL-C paragraphs}  
\textit{Point.} Organisational innovation, not kit alone, determines who gains from a military revolution.  
\textit{Evidence.} Krepinevich shows that similar platforms in 1940 produced divergent outcomes because Germany integrated doctrine, C2, and organisation coherently (pp.~36--37).  
\textit{Explain.} For Ireland, investing in doctrine, mission command, and joint training will yield disproportionate returns in coalitions.  
\textit{Limit.} The cases are historical and Western.  
\textit{Consequent:} Prioritise doctrinal agility and C2 experimentation over platform counting.  

\textit{Point.} Advantages from new military systems are brief, so small states must design for rapid adaptation.  
\textit{Evidence.} The British Dreadnought lead vanished as rivals built copy fleets; monopolies proved fleeting (pp.~36--37).  
\textit{Explain.} Irish DF should treat capability as a learning pipeline, not a one-off purchase.  
\textit{Limit.} Naval arms-race dynamics may not map perfectly to land or cyber.  
\textit{Consequent:} Build fast-learning, upgradeable capabilities with coalition pathways.  


\section*{Evidence \& Implication Log}  
\begin{tabular}{p{3.2cm}p{4.2cm}p{3.6cm}p{3.2cm}p{4.2cm}}  
	\textbf{Claim} \& \textbf{Best source (page)} \& \textbf{Rival source/reading} \& \textbf{Condition} \& \textbf{Implication for Irish DF}\\\hline  
	Org. innovation decides gains \& 1940 concepts beat similar kit (pp.~36--37) \& Technology alone suffices \& Competitive adversary \& Invest in doctrine, C2, design agility.\\  
	Advantages are brief \& Dreadnought lead evaporated (pp.~36--37) \& First-mover retains edge \& Rapid diffusion \& Build adaptability and rapid learning.\\  
	ISR--precision rise \& Detect, track, engage faster; sims matter (pp.~40--41) \& Attrition persists \& Integrated sensors and C2 \& Prioritise MSA, data sharing, sims.\\  
	Small powers can leap \& Ideas substitute for mass (pp.~38--39) \& Scale always wins \& Focused niche \& Develop specialist EU mission capabilities.\\\hline  
\end{tabular}  

\textbf{Gaps}\\  
• Fresh small-state cases applying ISR--simulation niches in EU missions.\\  
• Park deep cost data; date-bound US-centric debates.\\  

\textit{Notes on evidence:} Definition and four-element framework; Gulf War not a completed revolution; and ISR--precision--simulation trajectory are drawn directly from the uploaded article.  

\parencite{KEOHANE_1988}
