\chapter{Personal Development Goals and Learning Objectives}

Success on the \gls{jcsc} cannot be measured by academic performance alone. My career to date has shown that I can complete demanding tasks with determination. I communicate effectively, think analytically and work with candour. Yet I also recognise weaknesses that may hinder progress. I am prone to anger, I spread myself too thin and I remain disorganised. My assertiveness, often a strength, can become overbearing. I have a low social battery and can be cynical, distrusting institutions as well as individuals. These traits may be perceived by colleagues as intimidating or disengaged, even when my intent is otherwise, which underlines the importance of perspective-taking.  

Family is the central value shaping my goals. In the past I placed career before home life. That is no longer acceptable. My foremost objective is to ensure that the course does not come at the expense of family. Alongside this I aim to maintain health and fitness by exercising three times per week and sustaining my body weight. These are not secondary commitments but essential. They preserve resilience and ensure that I can complete the course without undermining personal priorities. In leadership terms this reflects the Defence Forces doctrine that credibility rests on both professional and personal integrity \parencite{DFDOCTRINE_2023}.  

Academically I intend to achieve results that reflect my education and input. I already hold two Master’s degrees and do not need to prove myself again in that regard. The challenge is to balance ambition with restraint, recognising that at times “good enough” may be sufficient when the alternative is sacrificing health or family. However, lowering standards does not come easily. My habit has been to set the bar as high as possible, and the adjustment required to protect family time may itself be the most difficult learning objective.  

My learning objectives are threefold. First, to practise reflection throughout the course, acknowledging both strengths and blind spots. Second, to refine my leadership style, especially in balancing assertiveness with patience and measured silence with necessary intervention. Third, to maintain equilibrium between professional development and family commitments. These objectives correspond with recommended approaches to reflective practice in higher education, which emphasise the integration of personal experience with theoretical insight \parencite{MOON_2004}.  

These goals are ambitious yet realistic. They recognise that the greatest challenge of the course is not mastering theory but managing myself. If I emerge with integrity intact, family relationships protected and professional competence enhanced, the course will have been worthwhile.  

\textbf{Thesis:} The most important outcome of this course will be sustaining family and resilience while refining leadership practice through critical reflection and managed ambition.  

\begin{itemize}
	\item My case narrative is subjective; I will validate it through peer and mentor feedback.  
	\item Balancing ambition with family commitment is difficult; I will set explicit boundaries for study time.  
	\item Cynicism towards institutions risks disengagement; I will reframe it as a driver for constructive critique.  
\end{itemize}

\textbf{Next step:} I will construct a weekly plan that protects family and fitness commitments before allocating study, ensuring that ambition remains contained within sustainable boundaries.  
