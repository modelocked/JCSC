\chapter{For the proposal that small states can influence internationally - provided they have credibility}

\textcite{FARRELL_2019} speaks of ``weaponized interdependence'', where in an interdependent international order \footnote{Such as current the liberal rules-based order.}, the connections are not symmetric. That is to say that some actors (states) become hubs/nodes for particular types of connections. For example, Ireland is a hub for international technology firms, Singapore is a note for [], Panama's canal is a node for international shipping. States controlling such nodes can leverage interdependence to coerce others. However, such power is fragile. It could be employed in a hard power (coercion) or soft (influential). Soft power is that which is not coercive. To many (perhaps most), soft power is a weaker capability than hard power (which is coercive). It seems that the most powerful is in fact the use of soft power when you already have hard power. Said differently, having hard power and not using it is the highest form of power. I suggest that there is an analogue here with leadership. Command is a legal authority bestowed on an individual. However, authority is given to a person by their subordinates, where they get what they want through influence rather than coercion. Hence, small states who would consider weaponised interdependence must, in a neo-realist manner, understand the system in which they live. They can exercise influence \textit{to an extent} bounded by tolerance of the great powers.

So, for small states whose only meaningful tools are soft power, then they are bounded because their use of soft power isn't necessarily virtuous. They're not employing hard power because they choose not to, in favour of soft power. They're not employing hard power because they don't have it to begin with. Covnersely, a large power is virtuous by not using hard power.



NORTH KOREA IS A NEXCELLENT EXAMPLE OF DETERRENCE. JUST BY POSSESSING THE NUCLEAR WEAPONS MEANS NOWBODY WANTS TO ATTACK YOU BECAUSE IT BROADEDNED THE SPECTRUM OF WAR – TO GENERAL AND THEN THERMONUCLEAR WAR. GENERALLY, MILITARIES AND POLITICS HAS TO NEVER HAS BEEN TO GENERAL WAR BECAUSE OF THE THEAT OF THERMONUCLEAR. By way of an example of lacking credibility. Putin regularly threatens to use nuclear power. Indeed the soviet union has said so for decades. Does this reduce his credibility. This echoes Obama’s red line WRT the use of WMDs  in Syria.

HENCE QATAR’S USE AL JEZERRS IS VERY USERFUL. CONSIDER NORTH KOREA, PERHAPS SOFT POWER THROUGH “GETTING INFORMATION IN” THORUGH INFORMATION OPS COULD BE VERY INFLUENTIAL. 

For the irish army the softpwoar comes a little bit from peackepeepigin. Gives a little credibility on the international stage.. Soft power is very important: education, IHL, etc., Ireland offers a lot of prestige this way, if you can go about affecting change that way, then great because you don’t 


ISRAEL used coercive deterrence for many decades called the “Begin doctrine”. Tjeu have used defensive “denial” and “punishment” (targeted assassinations, attack the opoponent’s key military sites; engage in allowing a large amount of collateral damage). \parencite{HIRST_2010} descriptions of Israeli conduct towards Lebanon clearly paint a reaslist perspective, where power was wielded primarily by the military. He states that ``a nation born of the sword was forever going ot live by it''. Indeed, the suggestion by Ben Gurion that the foreign ministry's job is to explain the actions of the defence ministry to the west of hte world underpins this. \parencite[p. 53]{HIRST_2010}. It is of note that such an aggressive perspective is at odds with considerations that Israel is a small state (read small power).

\textcite{DUMAN_2025} demonstrate that the UNSC’s structural dysfunction is starkly evident in its handling of the Gaza crisis (2023–2025).  Across thirteen draft resolutions, only four were adopted, while nine failed, primarily due to six U.S. vetoes blocking ceasefire initiatives.  Even adopted texts were hollowed out, with Washington demanding the removal of terms such as ``permanent'' ceasefire and recasting Resolution 2728 as ``non-binding''.  The authors argue that the veto has shifted from a mechanism for consensus to an instrument of strategic obstruction, paralysing the Council’s ability to address humanitarian crises.  This paralysis has profound implications: it enables Israel to continue projecting military power despite growing international condemnation, because its actions are structurally protected by U.S. vetoes within the Security Council.  In realist terms, Israel’s capacity to ignore normative pressure is not a function of its own legitimacy but of American shelter.  For small states like Ireland, the lesson is sobering: reliance on the UN for legitimacy is increasingly untenable when great-power sponsorship, rather than institutional consensus, dictates outcomes. \textcite{DUMAN_2025} The authors suggests peacekeeping’s obsolescence, pushing Ireland toward structural-realist adaptation.


\parencite{ROTHSTEIN_1966} notes that ``Intervention was also bound to cause trouble between the rest of the nonaligned states and the two superpowers; and  as their votes in the UN and their political support on a wide range of issues became increasingly more important than their military contribution, that too became a consideration''. This highlights another aspect of  Thorhallsson's ``perceptual size'' \nocite{THORHALLSSON_2006}, where a small state may have elevated influence due to their votes within an international institution such as the UN. It is of note therefore that the increasing logjammed nature of hte UNSCR diminishes the influence of small states who leverage that multilateral organisation. Data from \parencite{HELLMUELLER_2024} demonstrate that the \textit{establishment of new UN peace operations has collapsed since the 1990s}. During the immediate post-Cold War decade, the Security Council authorised on average five to six new missions per year (1991--1995). This rate declined to roughly two to three new missions per year through the 2000s.Since 2012, however, the creation of new peacekeeping operations (PKOs) has virtually ceased, with the few new mandates established being overwhelmingly \textit{special political missions} (SPMs) or special envoys/advisors rather than robust peacekeeping deployments. The empirical trend is stark: whereas the 1990s represented the high-water mark of UN activism, the 2010s and 2020s have seen the Security Council \textit{almost entirely abandon the authorisation of new PKOs}.This pattern provides strong evidence for the contention that peacekeeping is in structural decline.For small states such as Ireland, which have historically derived disproportionate legitimacy and influence from participation in UN peacekeeping, the erosion of new mandates signals the waning of a core foreign-policy instrument. Where once Ireland could reliably translate its neutrality and peacekeeping contributions into international credibility, the paralysis of the Security Council now undermines this niche. In strategic terms, the \textit{ends} (influence and legitimacy) no longer align with the available \textit{ways and means} (credible peacekeeping deployments). This decline therefore marks not only the twilight of UN peacekeeping but also the need for small states to reconsider alternative avenues for sustaining international relevance.


\nocite{COTTEY_2022}Ireland’s security posture is best interpreted through a neo-realist lens of hedging. COTTEY (2022) underscores that despite the systemic shock of the Ukraine war, Irish national security continues to rest on very low defence spending, limited combat capability, and only cautious EU engagement. This continuity reflects not idealist neutrality but a pragmatic calculation: small states cannot afford unilateralism and therefore hedge by balancing autonomy with tacit alignment to stronger powers. \parencite{FANNING_2015,AYIOTIS_2023} demonstrate that this logic is longstanding—since de Valera’s realist manoeuvres in WWII, Ireland has relied upon covert security dependence on Britain, cloaked in the language of neutrality. While COTTEY does not explicitly frame this as hedging, the evidence of structural free-riding and reliance on great power guarantees illustrates precisely that dynamic. The limit is that neutrality remains politically potent at home, constraining acknowledgement of dependence; the implication is that Irish policy is best understood as a form of security hedging, combining symbolic neutrality with material reliance on others.


Flynn (2019)\nocite{FLYNN_2019} cautions that small states must avoid tokenism if they wish to remain relevant in peacekeeping: only reinforced company-level contributions, with credible air and maritime assets, cross the threshold of meaningful influence. Ireland’s shrinking footprint—complete withdrawal from UNDOF in 2024, near-total exit from African missions (save for minimal support in Uganda), and the planned phaseout of its UNIFIL deployment under UNSCR 1701 by 2027—suggests it has now fallen below that threshold, with only around 300 troops remaining in Lebanon as of August 2025 amid the mission's final mandate extension to December 2026. The realist implication is stark: peacekeeping, once Ireland’s primary niche and source of legitimacy, is entering twilight, exacerbated by UN Security Council dysfunction, with only about 25 resolutions adopted by September 26, 2025, and no new peacekeeping missions established since 2014. \nocite{QUINN_2018} Quinn (2018) frames peacekeeping as an expression of Irish identity and a liberal foreign-policy tool, but this is increasingly untenable in a multipolar world where consensus on mandates wanes. As de Valera’s wartime neutrality revealed \parencite{FANNING_2015,AYIOTIS_2023}, Ireland’s posture has always been realist: a veil of moralism masking strategic hedging. Today, performative liberalism sustains the illusion of influence, but beneath it lies structural realism and material weakness, with defense spending stagnant at around 0.3 percent of GDP. My own experience of recent debates about Gaza illustrates this tension: while public figures argued for symbolic gestures such as escorting aid ships or deploying civilian observers, the reality is that Israel, as a hard realist state, is unmoved by such symbolism. Gestures may satisfy domestic expectations, but they do not alter outcomes. This underscores the risk that Ireland’s legitimacy, long derived from peacekeeping, is collapsing into symbolic performance without strategic effect. Quinn \nocite{QUINN_2018} reports on perceptions regarding Ireland's participation in peacekeeping as values-based rather than transactional. I suggest that it was naive of him not to consider the realist or neo-realist aspect of Ireland's participation. I assess that while Ireland has a while the performative libaral aspect of peacekeeping is real (and certainly believed by the public), its origins have been realist in nature. Given \parencite{WALTZ_1979}'s concepts of structual realism, I assess that Ireland's IR engagement is structiural-realist in nature.  The government does engage in the liberal international order but is distrustful of it, or uses it cynically for its own benefit. Ireland seeks to 
increase its security through multilateral security cooperation initiatives rather than by force of  arms'' \parencite{TONRA_2007}.