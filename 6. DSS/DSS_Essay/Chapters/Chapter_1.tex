\chapter{Introduction}


\section{Introduction}This chapter introduces the debate regarding how small states can influence international security outcomes. It defines what it means to be small state, explains why legitimacy emerges as the \gls{cog} and explores how myths \& national history shape strategic identity. This essay shall consider Ireland\index{Ireland}, \gls{dprk}\index{Democratic Peoples Republic of Korea}\index{DPRK|see{Democratic People's Republic of Korea}} and Israel\index{Israel} as `problem cases' for the small state category. It employs a five-effects framework handrail throughout: niche specialisation\index{Niche Specialisation}; \index{Organisational Agility}organisational agility; hybrid leverage\index{Hybrid Leverage}; \index{Soft Power Synergy}soft power synergy; \index{Legitimacy}legitimacy. This framework is intended to act as a boundary for the argument and to establish the following hypothesis: That small states are limited in their ability to dictate outcomes but may still shape them conditionally through legitimacy and institutional engagement.

\section{Defining the Small State} The first task is to define what constitutes a small state.\parencite{KEOHANE_1969,BROOKS_2007,BIDDLE_1996}
 