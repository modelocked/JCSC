\chapter{MASTER PARTIALLY-ANALYSED SOURCE MATERIAL FOR ESSAY}

\section*{Essay Skeleton: Small States and Security Outcomes}

\subsection*{Introduction}
This essay addresses the research question: \textit{To what extent can small states influence international security outcomes through military and non-military means?} It employs a five-effects framework---Niche Specialisation, Organisational Agility, Hybrid Leverage, Soft Power Synergy, and Legitimacy---to analyse how small states, and Ireland in particular, navigate structural constraints. The sub-questions interrogate whether innovation, institutions, and legitimacy provide meaningful influence, or whether continuity and great-power dominance render small states peripheral. The analysis draws on both theoretical and policy sources, ranging from \textcite{KREPINEVICH_1994} to Irish government documents, to ensure conceptual and practical grounding.

\subsection*{1. Niche Specialisation Effect}
\textbf{DIMER.} \textcite{KREPINEVICH_1994} contends that military revolutions occur when new technologies are embedded in doctrine and organisations, creating scope for small states to ``steal a march'' on larger powers. By contrast, \textcite{GRAY_2005} insists that continuity and political context outweigh technological rupture. Ireland’s peacekeeping and cyber niches exemplify attempts at relevance, but their influence depends on political coherence.

\textbf{PEEL-C.} Point: Small states gain visibility by cultivating niche roles. Evidence: Irish UN peacekeeping and EU crisis-management training missions. Explain: These projects demonstrate credibility disproportionate to size. Limit: As Gray warns, technological or operational novelty cannot compensate for absent political ends. Consequent: Niche specialisation is viable, but conditional on legitimacy. 

\textbf{Comparative.} Krepinevich’s optimism contrasts with Gray’s scepticism. For Ireland, the synthesis suggests niches can amplify voice, but only when embedded in coherent political strategies.

\subsection*{2. Organisational Agility Effect}
\textbf{DIMER.} Agility derives from structural reform and relational legitimacy. The HLAP (2022) commits to new command structures and Reserve revitalisation \parencite{HLAP_2022}, yet the 2019 Defence White Paper update exposes enduring gaps \parencite{WHITE_2019}. Cohen’s concept of an ``unequal dialogue'' stresses that agility is as much about political–military interaction as about bureaucratic reform \parencite{COHEN_2002}. 

\textbf{PEEL-C.} Point: Organisational agility is essential for small-state credibility. Evidence: HLAP’s structural reforms, Irish participation in PESCO. Explain: Adaptation demonstrates commitment to multilateral engagement. Limit: Recruitment and retention crises undermine delivery. Consequent: Agility is necessary but insufficient without material credibility. 

\textbf{Comparative.} Cohen complements Gray: structural reforms are hollow without political dialogue. For Ireland, agility requires both institutional adaptation and genuine civil–military engagement.

\subsection*{3. Hybrid Leverage Effect}
\textbf{DIMER.} Farrell and Newman show how interdependence creates both vulnerabilities and coercive opportunities through chokepoint and panopticon effects \parencite{FARRELL_2019}. For small states, this duality means exposure to great-power coercion but also scope to shape regulatory norms. Gray’s scepticism tempers optimism, warning that adversaries adapt to offset hybrid advantages \parencite{GRAY_2005}.

\textbf{PEEL-C.} Point: Hybrid tools both constrain and empower small states. Evidence: EU’s GDPR illustrates how even small states, through EU platforms, can shape global digital standards. Explain: Ireland’s embeddedness in EU networks converts vulnerability into regulatory leverage. Limit: Agency is collective, not unilateral; dependence on EU structures persists. Consequent: Hybrid leverage functions through institutional coalitions. 

\textbf{Comparative.} Nye’s optimistic account of soft power contrasts with Farrell and Newman’s structural pessimism. The synthesis suggests that small states must mitigate coercion through regulatory multilateralism, turning vulnerability into legitimacy.

\subsection*{4. Soft Power Synergy Effect}
\textbf{DIMER.} Nye defines soft power as the ability to co-opt through attraction, stressing that credibility is the scarcest resource \parencite{NYE_2008}. Keohane highlights that institutions provide both rationalist and reflective leverage, multiplying small-state voice \parencite{KEOHANE_1988}. Bailes and Thorhallsson argue that EU shelter enhances small-state legitimacy but risks dependence \parencite{BAILES_2013}. 

\textbf{PEEL-C.} Point: Soft power amplifies small-state influence when embedded in institutions. Evidence: Ireland’s neutrality and peacekeeping framed within UN and EU contexts. Explain: These practices magnify diplomatic visibility and legitimacy. Limit: Institutional shelter can erode autonomy and trigger domestic contestation. Consequent: Soft power synergy is effective but contingent. 

\textbf{Comparative.} Thorhallsson’s optimism about shelter contrasts with Gray’s warning that symbolic gestures collapse without coherent ends. For Ireland, synergy requires embedding neutrality in substantive political purpose, avoiding hollow symbolism.

\subsection*{5. Legitimacy Effect}
\textbf{DIMER.} Gray insists that strategy is defined by political consequences, not operational brilliance \parencite{GRAY_2018}. Cohen highlights that legitimacy derives from active, frictional civil–military dialogue \parencite{COHEN_2002}. Irish policy documents (HLAP 2022; Strategy Statement 2025–2028) emphasise cultural reform, oversight, and transparency as responses to legitimacy deficits \parencite{DOD_2025}. 

\textbf{PEEL-C.} Point: Legitimacy is the decisive multiplier for small states. Evidence: Ireland’s neutrality and peacekeeping, coupled with cultural reforms in defence. Explain: External legitimacy relies on neutrality; internal legitimacy requires reform and dialogue. Limit: Neutrality without delivery risks hollowing credibility. Consequent: Legitimacy anchors all other effects. 

\textbf{Comparative.} Nye links legitimacy to credibility and attraction, while Gray ties it to political coherence. The synthesis underscores that both external credibility and internal political grounding are indispensable.

\subsection*{Conclusion}
Small states cannot unilaterally determine international security outcomes, but they can shape norms, institutions, and perceptions in ways disproportionate to material resources. Niche specialisation and hybrid leverage offer tactical openings, while organisational agility ensures adaptability. Yet these effects are fragile without the synergistic power of institutions and, above all, the anchor of legitimacy. For Ireland, neutrality, peacekeeping, and EU membership provide platforms of influence, but credibility hinges on aligning limited means with coherent political ends. Influence is thus conditional, relational, and legitimacy-dependent.



\section*{Cavalry to Computer: The Pattern of Military Revolutions (Krepinevich, 1994)}

\parencite{KREPINEVICH_1994}Speaks of the changes in military tactics through the use of the ``epoch''. This is at odds with Gray (2018), who notes the arbitrary nature of ages during historical studies clouds the evolutionary nature of mankind.  

Krepinevich’s assessment following the advent of nuclear weapons, that ``here was a shift so radical it convinced nearly all observers that a fundamental change in the character of warfare was at hand,'' might actually be in agreement with Gray, who expressly excludes the nuclear sphere from his discussion of grand strategy.  

First, and to reiterate a point made earlier, emerging technologies only make military revolutions possible. To realize their full potential, these technologies typically must be incorporated within new processes and executed by new organizational structures. He highlights that the Wehrmacht's intellectual breakthroughs are what really unlocked the technological breakthroughs. Gray (2018) discussed this, noting that while Germany was technologically and intellectually superior, its morally bankrupt political system meant no tegy could succeed.  

His statement that ``an American tank designed in the 1920s was adapted by the Soviets in the process of developing the T-34, one of the most effective tanks to emerge during World War II. The U.S. Army, on the other hand, was equipped during the war primarily with the inferior Sherman tank'' is a technocentric view. The Sherman was more plentiful and easier to maintain and repair.  

But a confirming war is not essential for military organizations to seize opportunities. This is my concern with Ireland and drones and AI. Indeed, this is how I feel: military revolutions may offer major opportunities for relatively small or medium-sized powers to steal a march on greater powers.  

\section*{DIMER Analysis — Krepinevich (1994)}

\textbf{Describe} \\
Krepinevich argues that military revolutions occur when new technologies are combined with novel operational concepts and organisational adaptation, producing order-of-magnitude gains in combat effectiveness.  

\textbf{Interpret} \\
Relevant to small states: revolutions may offer major opportunities for them to steal a march on greater powers. Aligns with Niche Specialisation and Hybrid Leverage Effects. Risk: arbitrary epochs overstating discontinuity.  

\textbf{Methodology} \\
Historical/theoretical synthesis, comparative historical sweep, but selection bias.  

\textbf{Evaluate} \\
Contribution: foundational model for RMA debates. Weakness: technocentric, overlooks logistics and legitimacy.  

\textbf{(Autho)R} \\
Andrew F. Krepinevich, U.S. defence analyst, influenced post-Gulf War debates.  

\textbf{Limit → Implication} \\
Limit: ex post epoch framing, technocentric bias.  
Implication: small states should focus on organisational adaptation and niche exploitation.  

\section*{Mapping Krepinevich into Effects}

\begin{itemize}
	\item \textbf{Niche Specialisation}: small states can steal a march on greater powers.  
	\item \textbf{Organisational Agility}: four elements stress doctrine/structure as decisive.  
	\item \textbf{Hybrid Leverage}: confirms war not essential; parallels modern cyber/drones.  
	\item \textbf{Soft Power Synergy}: minimal, but notes societal roots of revolutions.  
	\item \textbf{Legitimacy}: underplays moral context; contrast with Gray.  
\end{itemize}

\section*{Comparative PEEL-C (Krepinevich vs Gray)}

Point: The literature diverges on whether small states can exploit revolutions or whether continuity/legitimacy dominate.  
Evidence: Krepinevich highlights organisational breakthroughs; Gray warns innovation cannot overcome illegitimate politics.  
Explain: For Ireland, innovation (drones) is possible, but risks collapse without legitimacy.  
Limit: Krepinevich overstates rupture; Gray underplays innovation.  
Consequent: Balanced approach needed — innovation plus legitimacy.  

\section*{Lecture Notes on New Realism (Ireland)}

Ireland’s policy is basically unchanged: good relations with UK, Europe and USA, neutrality as cost-saving and security-enhancing. The Defence Forces exist largely “just to stay alive.” Ireland has played the neorealist game well: not taking strong stances, not obstructing, e.g. Bertie Ahern’s ambiguity on GWOT while allowing U.S. use of Shannon. Gray (2018) reminds us that Ireland’s military is not central to political policy, save for token peacekeeping.  

\section*{Integrating New Realism into Effects}

\begin{itemize}
	\item \textbf{Niche Specialisation}: token peacekeeping sold as valuable.  
	\item \textbf{Organisational Agility}: DF survivalist mindset reflects inertia.  
	\item \textbf{Hybrid Leverage}: parity matters; hybrid tools more realistic than balancing.  
	\item \textbf{Soft Power Synergy}: neutrality and consensus diplomacy amplify influence.  
	\item \textbf{Legitimacy}: external legitimacy strong, but domestic questions persist.  
\end{itemize}

\section*{PEEL-C on Ireland and Legitimacy}

Point: Ireland’s influence relies on neutrality and legitimacy rather than effectiveness.  
Evidence: Neutrality, DF “just staying alive,” Shannon ambiguity.  
Explain: Plays neorealist game effectively.  
Limit: Gray (2018) warns disconnection between means and ends.  
Consequent: Overreliance risks hollowing out legitimacy.  

\section*{Comparative Analysis (All Four)}

Keohane (1969) and Thorhallsson (2006) highlight institutions and perceptions as multipliers. Krepinevich (1994) offers optimism about innovation. Gray (2018) stresses legitimacy and continuity. Ireland illustrates the balance: institutions, agility, but constrained by legitimacy gaps.  

\section*{Concluding PEEL-C}

Point: Small states shape outcomes by balancing institutions, innovation, and legitimacy.  
Evidence: Keohane (institutions), Thorhallsson (perceptions), Krepinevich (innovation), Gray (legitimacy).  
Explain: Ireland can leverage neutrality, institutions, and innovation, but legitimacy is decisive.  
Limit: Each incomplete alone.  
Consequent: Synthesis of all four is most robust.  

\section*{DIMER Analysis — Keohane (1988)}

\textbf{Describe} \\
Keohane distinguishes between two approaches to analysing international institutions: the \textit{rationalistic}, grounded in rational choice and transaction cost theory, and the \textit{reflective}, focused on norms, intersubjective meanings, and practices such as sovereignty. Rationalists argue that institutions reduce uncertainty and facilitate cooperation by lowering transaction costs, while reflectivists emphasise how institutions constitute actors’ identities and preferences. Keohane calls for a synthesis, urging empirical work that integrates rationalist and reflective insights \parencite{KEOHANE_1988}.

\textbf{Interpret} \\
For small states, this distinction matters because it highlights both instrumental and normative dimensions of institutional participation. Rationally, institutions provide low-cost platforms for influence (e.g., UN, EU, NATO). Reflectively, they embed legitimacy and normative credibility (e.g., neutrality, peacekeeping). For Ireland, neutrality functions as both a rational cost-saving and institutionally beneficial strategy, while also serving as a reflective practice reinforcing sovereign identity and legitimacy.

\textbf{Methodology} \\
This article is a conceptual and theoretical essay (ISA Presidential Address). It synthesises ongoing debates between rationalists (e.g., utility maximisation, regime theory) and reflectivists (e.g., sociological and normative institutionalists). Its strength lies in its typological clarity and influence on subsequent IR theory; its weakness is abstraction and lack of empirical cases.

\textbf{Evaluate} \\
Contribution: This article bridges rational institutionalism (Keohane’s \textit{After Hegemony}, 1984) with emerging reflective critiques, foreshadowing the rationalist–constructivist synthesis of the 1990s. It provides a theoretical rationale for why institutions matter to both powerful and weak states. Weakness: the reflective side is underdeveloped, with little detail on how small states operationalise legitimacy through institutions.citep

\textbf{(Autho)R} \\
Robert O. Keohane, a leading institutionalist scholar, was at the forefront of debates on regimes and interdependence. His bias is strongly pro-institutionalist, but here he shows openness to reflective critiques. Context: late Cold War, IR theory pluralisation. Influence: highly cited, foundational for rationalist vs. constructivist debates.

\textbf{Limit → Implication} \\
\textit{Limit}: The article is overly abstract and does not directly address the agency of small states.  
\textit{Implication}: For this essay, Keohane provides the theoretical justification for combining rationalist (institutional leverage) and reflective (legitimacy and norms) dimensions. This aligns with the Soft Power Synergy and Legitimacy Effects, showing how Ireland leverages neutrality and peacekeeping both instrumentally and normatively.

\section*{PEEL-C Paragraph (Application to Soft Power Synergy)}

\textbf{Point} \\
Small states amplify their influence by embedding themselves in institutions that provide both rational and normative leverage.  

\textbf{Evidence} \\
Keohane (1988) demonstrates that institutions operate as rationalistic mechanisms reducing transaction costs and as reflective practices embedding norms such as sovereignty. He stresses that cooperation always occurs within some institutional context that shapes behaviour and expectations \parencite{KEOHANE_1988}.  

\textbf{Explain} \\
For Ireland, neutrality and peacekeeping within UN and EU frameworks illustrate this dual role: rationally, they provide low-cost platforms for voice and security benefits; reflectively, they sustain Ireland’s legitimacy as a principled small state.  

\textbf{Limit} \\
However, Keohane’s analysis is abstract and does not explain how small states prevent marginalisation within power-dominated institutions.  

\textbf{Consequent} \\
Thus, Ireland’s ability to convert neutrality into influence relies not just on rationalist cost–benefit, but also on sustaining reflective legitimacy across institutional contexts, making institutional credibility as important as material capacity.  

\section*{DIMER Analysis — Farrell \& Newman (2019)}

\textbf{Describe} \\
Farrell and Newman (2019) argue that globalisation has produced asymmetric networks of finance and information which can be exploited by powerful states. They identify two mechanisms of ``weaponised interdependence'': the \textit{panopticon effect}, whereby states with jurisdiction over central network nodes extract information, and the \textit{chokepoint effect}, whereby those same states exclude adversaries from critical flows. Their cases (SWIFT financial messaging and internet communications) demonstrate how the United States leverages network centrality to coerce others \parencite{FARRELL_2019}.

\textbf{Interpret} \\
This framework reframes interdependence from mutual gains (Keohane \& , 1977) to asymmetric vulnerabilities. For small states, the analysis is double-edged: embeddedness in global networks amplifies exposure to coercion by great powers, yet it also creates opportunities to participate in collective norm-building around network governance. For Ireland, reliance on EU and US-centric financial and digital networks constrains autonomy, but EU membership provides shelter and a collective platform to influence regulatory standards.

\textbf{Methodology} \\
The article is a conceptual–theoretical synthesis drawing on network theory and detailed analytic narratives of SWIFT and internet governance. Evidence is qualitative: case histories, policy documents, and elite interviews. Its strength lies in theoretical innovation (structural power via networks); its weakness is reliance on two US-centric cases, with limited generalisability to peripheral or small-state actors.

\textbf{Evaluate} \\
Contribution: Introduces a novel structuralist account of economic coercion, showing how infrastructures of cooperation double as infrastructures of coercion. It challenges liberal interdependence theory by exposing how globalisation entrenches hierarchies. Weakness: the account privileges great powers, underplaying the agency of middle and small states. Ireland, for instance, is framed mainly as vulnerable rather than as a potential shaper of EU-level regulatory responses.

\textbf{(Autho)R} \\
Henry Farrell (George Washington University) and Abraham Newman (Georgetown) are leading scholars of international political economy and regulatory politics. Their prior work on privacy and financial regulation situates them as institutionalist–structuralist thinkers. Bias: they foreground US dominance, downplaying middle and small-state adaptation. Influence: considerable — this article is central to contemporary debates on geoeconomics and coercion.

\textbf{Limit → Implication} \\
\textit{Limit}: The framework privileges US and EU structural dominance, sidelining how small states adapt or resist network coercion. \\
\textit{Implication}: For this essay, Farrell \& Newman provide theoretical depth for the \textbf{Hybrid Leverage Effect}: small states such as Ireland can exploit multilateral institutions (EU data regulation, GDPR, financial governance) to constrain weaponisation by great powers and amplify their legitimacy as norm entrepreneurs.

\section*{PEEL-C Paragraph (Application to Hybrid Leverage)}

\textbf{Point} \\
Small states are vulnerable in asymmetric networks but can also leverage institutional coalitions to mitigate coercion.  

\textbf{Evidence} \\
Farrell and Newman (2019) argue that states with jurisdiction over network hubs exploit \textit{panopticon} and \textit{chokepoint} effects, turning infrastructures such as SWIFT into instruments of coercion. They show how US pressure compelled SWIFT to disconnect Iranian banks, crippling Iran’s financial system \parencite{FARRELL_2019}.  

\textbf{Explain} \\
For Ireland, deeply embedded in EU financial and digital networks, this illustrates both vulnerability and opportunity. Vulnerability stems from reliance on US-centric infrastructures (for example, dollar clearing, American cloud providers). Opportunity arises through EU collective regulation: initiatives such as GDPR demonstrate how smaller states, acting through the EU, can set global standards that constrain US or Chinese firms, thereby converting embeddedness into leverage.  

\textbf{Limit} \\
However, Farrell and Newman underplay small-state agency. Their framework assumes structural dominance by great powers, leaving limited space for Ireland’s diplomatic entrepreneurship within the EU and UN.  

\textbf{Consequent} \\
Thus, weaponised interdependence highlights constraints on Irish autonomy, but also justifies the Hybrid Leverage Effect: Ireland’s influence lies not in counterbalancing great powers directly but in shaping collective regulatory frameworks that discipline network coercion.  

\section*{DIMER Analysis: Thorhallsson (2006)}

\subsection*{Describe}
Thorhallsson’s article explores how the size of states in the European Union (EU) shapes their political influence and strategic choices. It advances a conceptual framework distinguishing between small, middle, and large states, with small states analysed through dimensions of administrative capacity, bargaining power, and reliance on institutional frameworks. A central claim is that size conditions states’ ability to pursue autonomy versus shelter, with small states often leveraging the EU for influence that would be unattainable in bilateral relations \parencite{THORHALLSSON_2006}.

\subsection*{Interpret}
The argument directly applies to small EU member states, such as Ireland, Denmark, or the Baltic states, and helps explain how they navigate constraints in international politics. However, the framework is less applicable to small states outside strong institutional settings (e.g., Qatar or Singapore), where shelter is sought through bilateral alignments rather than supranational institutions. Importantly, the article illuminates the “so what” for Irish defence policy: Ireland’s influence in security debates is mediated not by raw power but by institutional access and normative legitimacy. The lacuna lies in the limited treatment of non-EU small states and in insufficient consideration of hybrid threats and new technologies.

\subsection*{Methodology}
The study is conceptual-theoretical rather than empirical. Thorhallsson synthesises existing theories of size and applies them to the EU context, drawing on institutionalist and constructivist approaches. While this provides analytical clarity, it sits low on the hierarchy of evidence: it offers a heuristic rather than data-driven validation. Reliability is moderate, but external applicability is limited given its EU-centric scope.

\subsection*{Evaluate}
The article contributes a valuable typology of small-state behaviour within multilateral institutions and clarifies how EU structures condition influence. Its strengths are theoretical precision and relevance to European integration studies. However, compared to realist accounts (e.g., Waltz \parencite{WALTZ_1979}) that stress structural constraints, Thorhallsson arguably overstates institutional empowerment. Similarly, contrasted with Mearsheimer’s scepticism about institutional efficacy \parencite{MEARSHEIMER_1994}, Thorhallsson’s optimism appears conditional on EU coherence, which may not hold in crisis situations. The article remains well-cited and continues to shape small-state scholarship, but it underplays hard power limitations.

\subsection*{(Autho)R}
Baldur Thorhallsson is a leading scholar of small states and European integration, with extensive publications on “shelter theory.” His institutionalist orientation and Icelandic perspective bias him toward highlighting the utility of multilateral institutions for small states. While authentic and credible, his stance may downplay scenarios where institutions fragment or fail to provide real protection.

\subsection*{Limit → Implication}
Limit: The EU-centric framework excludes small states operating outside strong institutional shelters.  
Implication: Application to Ireland is strong, but broader conclusions must be adapted when assessing comparative cases like Qatar or Singapore.

\section*{Effects Mapping}
\begin{itemize}
	\item \textbf{Niche Specialisation:} Supports the claim that small states can specialise in EU niches (e.g., peacekeeping, norm entrepreneurship) to amplify influence.
	\item \textbf{Organisational Agility:} Implies small states can adapt institutional strategies more quickly than larger powers constrained by consensus-building.
	\item \textbf{Hybrid Leverage:} Underdeveloped in this article; institutionalism underplays asymmetric or hybrid tools.
	\item \textbf{Soft Power Synergy:} Strongly aligned—EU membership enhances small states’ normative appeal and diplomatic credibility.
	\item \textbf{Legitimacy:} Central to Thorhallsson’s thesis: small states gain legitimacy through institutional participation, turning weakness into credibility.
\end{itemize}

\section*{PEEL-C Paragraph}
Thorhallsson argues that small states can amplify their influence through EU institutional mechanisms, leveraging legitimacy and niche specialisation \parencite{}. For example, Ireland’s engagement in Common Security and Defence Policy (CSDP) missions exemplifies how small states convert limited resources into disproportionate diplomatic credibility. This demonstrates that organisational agility and soft power synergy can offset material constraints by embedding national preferences in multilateral decisions. However, Thorhallsson’s EU-centric analysis excludes non-European small states, limiting its transferability. Consequently, while legitimacy and institutional shelter provide Ireland with leverage, broader applicability requires integrating perspectives from small states outside the EU framework.

\section*{DIMER Analysis — Bailes \& Thorhallsson (2012/2013)}

\textbf{Describe} \\
Bailes and Thorhallsson (2012/2013) argue that membership of regional institutions, especially the European Union, is central to small states’ security strategies. They contend that the EU provides not just economic shelter but also unique forms of ``soft'' security, such as resilience against pandemics, environmental degradation, energy vulnerability, cyber threats and terrorism. The EU’s supranational features, including the Commission and Court of Justice, provide smaller states with mechanisms of influence and protection that differ from traditional alliance politics \parencite{BAILES_2012}.  

\textbf{Interpret} \\
For small states, the EU represents a strategic protector that allows them to ``escape smallness’’ by pooling sovereignty and aligning national strategies with EU norms. The EU’s broad agenda means that small states’ declared strategies are often instrumental: signalling harmlessness, projecting loyalty, and importing EU priorities to gain legitimacy. For Ireland, this translates into framing neutrality, peacekeeping and EU membership as part of a coherent legitimacy-seeking strategy that magnifies its international voice despite limited military capability.

\textbf{Methodology} \\
The article is conceptual and comparative, drawing on qualitative evidence from European small states’ security strategies, EU governance structures, and case examples such as the Nordic states, Cyprus, and the Western Balkans. Its strength lies in providing a systematic framework for linking small-state strategy to EU shelter; its weakness is Eurocentrism and limited applicability beyond the European context.

\textbf{Evaluate} \\
Contribution: The paper advances small-state studies by integrating EU membership into strategic analysis, shifting attention away from ``hard’’ military threats to the existential and ``soft’’ security concerns shaping small-state behaviour. Weakness: it may exaggerate the EU’s protective role, underestimating the risks of sovereignty loss, identity erosion, and domestic elite–public divides.  

\textbf{(Autho)R} \\
Alyson Bailes, a former British diplomat and visiting professor in Iceland, and Baldur Thorhallsson, founder of the Centre for Small State Studies, are leading authorities on small-state theory. Bias: both are sympathetic to EU integration and institutionalist perspectives, downplaying realist constraints. Influence: the article is widely cited in European small-state literature.  

\textbf{Limit → Implication} \\
\textit{Limit}: The framework is highly Eurocentric and assumes the EU is a benign shelter, neglecting cases where integration amplifies vulnerability or triggers domestic resistance.  
\textit{Implication}: For this essay, Bailes \& Thorhallsson underpin the \textbf{Soft Power Synergy} and \textbf{Legitimacy Effects}: they show how small states such as Ireland instrumentalise EU membership to project legitimacy and gain influence, but they also reveal the strategic costs of dependence and loss of autonomy.

\section*{PEEL-C Paragraph (Application to Soft Power Synergy and Legitimacy)}

\textbf{Point} \\
Small states amplify their influence by instrumentalising the European Union as both a security shelter and a source of legitimacy.  

\textbf{Evidence} \\
Bailes and Thorhallsson (2012/2013) argue that EU membership allows small states to address vulnerabilities that no bilateral protector could manage, from energy resilience to border management and climate change. They note that declared strategies often import EU norms to signal harmlessness and loyalty, thereby gaining legitimacy and access to collective resources \parencite{BAILES_2012}.  

\textbf{Explain} \\
For Ireland, this analysis explains how neutrality and peacekeeping are embedded in EU frameworks, enabling Dublin to project itself as a responsible and principled actor. The EU provides Ireland with a platform for soft power synergy — enhancing diplomatic visibility and normative credibility — while also reinforcing its legitimacy at home and abroad.  

\textbf{Limit} \\
However, the article underestimates the costs: pooling sovereignty can expose small states to identity erosion, elite–public gaps, and obligations disproportionate to their capabilities. Ireland’s referenda on EU treaties reveal these tensions, where neutrality and sovereignty anxieties limit enthusiasm for deeper integration.  

\textbf{Consequent} \\
Thus, instrumentalising the EU offers Ireland significant legitimacy and visibility, but this influence is contingent and vulnerable to domestic scepticism. For small states, EU shelter enhances voice, yet overreliance risks hollowing out independent strategic agency.  


\section*{DIMER Analysis — Bailes, Thorhallsson \& Johnstone (2013)}

\textbf{Describe} \\
Bailes, Thorhallsson and Johnstone (2013) apply small state theory to Scotland’s possible independence, asking where it might seek ``shelter''. They argue that small states require shelter across multiple dimensions: strategic, political, economic and societal. Options include alliances with larger states, regional institutions such as NATO and the EU, and cultural–societal ties (e.g. Nordic cooperation). The study concludes that an independent Scotland would inevitably need multi-dimensional external shelter, but such protection always comes at a cost in autonomy \parencite{BAILES_2013}.

\textbf{Interpret} \\
The article extends the ``shelter theory’’ in small state studies: security is not confined to military alliances but includes economic stability, societal resilience and cultural legitimacy. For Ireland, the implication is clear: neutrality and peacekeeping are not stand-alone strategies but part of a broader need for economic and political shelter, provided primarily through EU membership and close relations with the US and UK. Shelter theory therefore underpins both the Legitimacy and Soft Power Synergy Effects in your framework.

\textbf{Methodology} \\
The paper is conceptual and comparative, drawing on EU small-state cases, Nordic precedents and Scotland’s contemporary debate. It uses qualitative analysis of policy options and historical analogies. Strength: integrates multi-dimensional notions of shelter. Weakness: Eurocentric and context-specific; limited applicability to small states outside Europe.

\textbf{Evaluate} \\
Contribution: Advances small-state theory by broadening security to include economic and societal shelter, not just hard power. This makes the framework highly relevant to contemporary non-traditional threats. Weakness: assumes shelter is always attainable through institutions; underestimates the fragility of great-power guarantees or domestic legitimacy.  

\textbf{(Autho)R} \\
Alyson Bailes (diplomat-scholar), Baldur Thorhallsson (Iceland, leading small-state theorist), and Rachael Lorna Johnstone (law scholar). Bias: pro-institutionalist and sympathetic to Nordic/EU integration. Influence: adds to Thorhallsson’s wider ``shelter theory’’ corpus.

\textbf{Limit → Implication} \\
\textit{Limit}: Eurocentric and institutionalist bias; assumes NATO/EU shelters are accessible and reliable, downplaying autonomy or domestic contestation. \\
\textit{Implication}: For this essay, the shelter framework clarifies that small states such as Ireland derive security primarily through embedding in multilateral and societal shelters. This reinforces the Soft Power and Legitimacy Effects but warns that influence is contingent on external guarantees and internal legitimacy.

\section*{PEEL-C Paragraph (Application to Legitimacy and Shelter)}

\textbf{Point} \\
Small states rely on multi-dimensional shelter to maintain legitimacy and influence, but such protection comes at the cost of autonomy.  

\textbf{Evidence} \\
Bailes, Thorhallsson and Johnstone (2013) argue that an independent Scotland would require strategic, political, economic and societal shelter from NATO, the EU, its neighbours and the US. They stress that shelter reduces vulnerability, absorbs shocks and aids recovery, but always entails concessions of sovereignty and freedom of manoeuvre \parencite{BAILES_2013}.  

\textbf{Explain} \\
For Ireland, the logic is similar: neutrality and peacekeeping operate only because of EU membership, transatlantic ties and economic integration. Ireland presents these as principled choices, but in practice they are embedded in external shelters that provide resilience and legitimacy.  

\textbf{Limit} \\
However, the framework underplays domestic legitimacy. Irish public scepticism towards EU defence integration, like Scottish divisions on NATO, shows that shelter strategies may trigger internal political costs even if externally beneficial.  

\textbf{Consequent} \\
Thus, shelter theory underpins the Legitimacy Effect: Ireland’s international credibility depends on its ability to balance external shelter with internal political acceptance. Over-reliance on institutional shelters without domestic support risks hollowing out legitimacy.  

\section*{DIMER Analysis — Bailes \& Thorhallsson (2012/2013)}

\textbf{Describe} \\
Bailes and Thorhallsson (2012/2013) argue that membership of regional institutions, especially the European Union, is central to small states’ security strategies. They contend that the EU provides not just economic shelter but also unique forms of ``soft'' security, such as resilience against pandemics, environmental degradation, energy vulnerability, cyber threats and terrorism. The EU’s supranational features, including the Commission and Court of Justice, provide smaller states with mechanisms of influence and protection that differ from traditional alliance politics \parencite{BAILES_2012}.  

\textbf{Interpret} \\
For small states, the EU represents a strategic protector that allows them to ``escape smallness’’ by pooling sovereignty and aligning national strategies with EU norms. The EU’s broad agenda means that small states’ declared strategies are often instrumental: signalling harmlessness, projecting loyalty, and importing EU priorities to gain legitimacy. For Ireland, this translates into framing neutrality, peacekeeping and EU membership as part of a coherent legitimacy-seeking strategy that magnifies its international voice despite limited military capability.

\textbf{Methodology} \\
The article is conceptual and comparative, drawing on qualitative evidence from European small states’ security strategies, EU governance structures, and case examples such as the Nordic states, Cyprus, and the Western Balkans. Its strength lies in providing a systematic framework for linking small-state strategy to EU shelter; its weakness is Eurocentrism and limited applicability beyond the European context.

\textbf{Evaluate} \\
Contribution: The paper advances small-state studies by integrating EU membership into strategic analysis, shifting attention away from ``hard’’ military threats to the existential and ``soft’’ security concerns shaping small-state behaviour. Weakness: it may exaggerate the EU’s protective role, underestimating the risks of sovereignty loss, identity erosion, and domestic elite–public divides.  

\textbf{(Autho)R} \\
Alyson Bailes, a former British diplomat and visiting professor in Iceland, and Baldur Thorhallsson, founder of the Centre for Small State Studies, are leading authorities on small-state theory. Bias: both are sympathetic to EU integration and institutionalist perspectives, downplaying realist constraints. Influence: the article is widely cited in European small-state literature.  

\textbf{Limit → Implication} \\
\textit{Limit}: The framework is highly Eurocentric and assumes the EU is a benign shelter, neglecting cases where integration amplifies vulnerability or triggers domestic resistance.  
\textit{Implication}: For this essay, Bailes \& Thorhallsson underpin the \textbf{Soft Power Synergy} and \textbf{Legitimacy Effects}: they show how small states such as Ireland instrumentalise EU membership to project legitimacy and gain influence, but they also reveal the strategic costs of dependence and loss of autonomy.

\section*{PEEL-C Paragraph (Application to Soft Power Synergy and Legitimacy)}

\textbf{Point} \\
Small states amplify their influence by instrumentalising the European Union as both a security shelter and a source of legitimacy.  

\textbf{Evidence} \\
Bailes and Thorhallsson (2012/2013) argue that EU membership allows small states to address vulnerabilities that no bilateral protector could manage, from energy resilience to border management and climate change. They note that declared strategies often import EU norms to signal harmlessness and loyalty, thereby gaining legitimacy and access to collective resources \parencite{BAILES_2012}.  

\textbf{Explain} \\
For Ireland, this analysis explains how neutrality and peacekeeping are embedded in EU frameworks, enabling Dublin to project itself as a responsible and principled actor. The EU provides Ireland with a platform for soft power synergy — enhancing diplomatic visibility and normative credibility — while also reinforcing its legitimacy at home and abroad.  

\textbf{Limit} \\
However, the article underestimates the costs: pooling sovereignty can expose small states to identity erosion, elite–public gaps, and obligations disproportionate to their capabilities. Ireland’s referenda on EU treaties reveal these tensions, where neutrality and sovereignty anxieties limit enthusiasm for deeper integration.  

\textbf{Consequent} \\
Thus, instrumentalising the EU offers Ireland significant legitimacy and visibility, but this influence is contingent and vulnerable to domestic scepticism. For small states, EU shelter enhances voice, yet overreliance risks hollowing out independent strategic agency.  


\section*{DIMER Analysis — EU Global Strategy (2016–2019)}

\textbf{Describe} \\  
The 2016 EU Global Strategy (EUGS) set out a vision of a ``stronger Europe’’ through five priorities: security and defence, resilience of states and societies, cooperative regional orders, support for multilateralism, and an integrated approach to crises \parencite{EU_2016}. Annual reports (2017–2019) show progressive institutionalisation: new defence structures (MPCC, PESCO, CARD), stronger EU–UN cooperation, closer NATO complementarity, and investment in strategic communication and public diplomacy \parencite{EU_2017,EU_2018,EU_2019} . The EUGS positioned the EU as a comprehensive security actor, blending military and non-military instruments with development, climate, and migration policies.  

\textbf{Interpret} \\  
For small states such as Ireland, the EUGS is highly relevant: it embeds neutrality and peacekeeping within broader EU frameworks and allows niche contributions (training missions, CSDP operations) to gain collective visibility. The emphasis on resilience and multilateralism reinforces Ireland’s reliance on institutions rather than unilateral military power. However, the drive towards ``strategic autonomy’’ creates tensions: small states may benefit from EU shelter but also face pressure to align with defence integration beyond their traditional comfort zones.  

\textbf{Methodology} \\  
The EUGS and its reports are policy documents—programmatic, declaratory, and integrative. They rely on consensus-building among member states, not empirical testing. Strength: provide authoritative statements of EU collective intent. Weakness: aspirational language may outpace actual implementation, particularly in military capacity.  

\textbf{Evaluate} \\  
Contribution: These documents show how the EU institutionalises soft and hard power into a hybrid toolbox, providing small states with platforms for influence. They underpin the \textbf{Soft Power Synergy Effect} (public diplomacy, normative leadership), \textbf{Organisational Agility Effect} (EU structures enabling niche contributions), and \textbf{Legitimacy Effect} (multilateral shelter legitimising small-state choices). Weakness: Eurocentric scope, limited attention to great-power pushback, and potential overreach in claims of strategic autonomy.  

\textbf{(Autho)R} \\  
The documents are authored under the EEAS and HR/VP Federica Mogherini, reflecting institutional and political consensus. Bias: aspirational integrationist framing, limited acknowledgement of member state reluctance or EU hard-power weaknesses. Influence: high—EUGS has structured EU foreign and security policy debates since 2016.  

\textbf{Limit → Implication} \\  
\textit{Limit}: The EUGS overstates EU capacity to act autonomously, and assumes member state alignment.  
\textit{Implication}: For this essay, the EUGS illustrates how small states can amplify influence through EU frameworks, but also how vulnerability arises if EU ambitions clash with neutrality traditions or domestic legitimacy.  

\section*{PEEL-C Application}

\textbf{Point} \\  
The EU Global Strategy strengthens the ability of small states like Ireland to project influence by embedding their niche roles in a collective framework.  

\textbf{Evidence} \\  
The 2017 report notes that ``more has been achieved in the last ten months [in defence] than in the last ten years,’’ citing PESCO, MPCC, and CARD as mechanisms for pooling capacity (EU\_2017). The 2019 review stresses that the EU acts as a global security provider through sixteen civilian and military missions and is moving towards ``strategic autonomy’’ \parencite{EU_2019}.  

\textbf{Explain} \\  
For Ireland, these developments offer legitimacy and visibility for peacekeeping, training missions, and crisis-management contributions, allowing small states to ``escape smallness’’ by acting through EU structures. This directly supports the \textbf{Soft Power Synergy} and \textbf{Organisational Agility Effects}.  

\textbf{Limit} \\  
However, deeper defence integration (e.g., PESCO commitments) risks clashing with neutrality narratives, raising domestic legitimacy questions. The EU’s aspirational tone may also mask limited operational capability.  

\textbf{Consequent} \\  
Thus, the EUGS illustrates both opportunity and constraint: small states can leverage EU shelter for influence, but must carefully balance external commitments with domestic legitimacy.  

\section*{DIMER Analysis — Gray (2005)}

\textbf{Describe} \
Gray (2005) asks whether war has changed since the Cold War. He argues that while the character of warfare shifts, its nature remains constant: war is always political, shaped by context, and full of danger, uncertainty, and friction. He provides four caveats against futurology: never divorce war from context; beware preparing for the wrong kind of conflict; trend analysis is unreliable; and history shows poor predictive records. He then analyses post–1989 warfare, identifying eight themes: (1) war’s nature is unchanging; (2) the US has mainly fought third-rate enemies; (3) so-called “new wars” are not actually new; (4) the US temporarily is the balance of power; (5) military transformation is overhyped; (6) interstate war is down but not out; (7) terrorism has changed but will not endure; and (8) cultural delegitimisation of war is fragile. He concludes that Clausewitz endures: war is political behaviour for political ends, and continuity outweighs change \parencite{GRAY_2005}.

\textbf{Interpret} \
For your essay, Gray provides the sceptical counterweight to technological or institutional optimism. Unlike Krepinevich’s “military revolutions,” Gray insists that continuity, politics, and legitimacy dominate. For small states, this implies caution: neutrality, peacekeeping, or drones will not transform their strategic situation if political purpose and legitimacy are absent. Ireland’s tendency to treat neutrality as cost-saving and peacekeeping as symbolic fits Gray’s warning: without serious political ends, military means cannot generate lasting influence.

\textbf{Methodology} \
Conceptual–historical analysis rooted in Clausewitzian theory and strategic history. Evidence is interpretive, not empirical: examples range from Germany’s failures in 1914/41 to US interventions in Panama, Somalia, Kosovo, Afghanistan, and Iraq. Strength: sharp theoretical clarity; Weakness: pessimistic bias and focus on Western cases.

\textbf{Evaluate} \
Contribution: Reinforces the enduring relevance of Clausewitz and warns against overstating novelty in war. Offers a corrective to RMA and “new wars” narratives. Weakness: downplays how innovation or institutions might grant small or medium states agency. His framework risks pessimism: small states may appear doomed to irrelevance, unless they embed their choices in legitimate political ends.

\textbf{(Autho)R} \
Colin S. Gray was professor at the University of Reading and one of the leading strategic theorists of his generation. Bias: strongly Clausewitzian, sceptical of fads (e.g., RMA, post-heroic warfare). Influence: high — a major figure in debates on continuity vs. change in war.

\textbf{Limit → Implication} \
\textit{Limit}: Gray underestimates the potential of innovation, institutions, and norms to create niches for small states.
\textit{Implication}: In your essay, use Gray to ground the \textbf{Legitimacy Effect}: Ireland’s military posture is strategically hollow unless tied to coherent political purpose. This tempers Krepinevich’s optimism and Thorhallsson’s institutional agency.

\section*{PEEL-C Paragraph (Application to Legitimacy Effect)}

\textbf{Point} \
Gray shows that small states cannot rely on technology, institutions, or symbolic gestures unless their military activity is tied to legitimate political ends.

\textbf{Evidence} \
Gray (2005) insists that “whatever about warfare is changing, it is not, and cannot be, warfare’s very nature,” stressing that politics, uncertainty, and legitimacy define outcomes. He critiques overconfidence in “new wars,” military transformation, and US hegemony, concluding that continuity prevails \parencite{GRAY_2005}.

\textbf{Explain} \
For Ireland, this reinforces that neutrality and peacekeeping only matter if they are embedded in political strategy. Otherwise, they risk being symbolic rather than substantive contributions. This complements your lecture note observation that the Defence Forces exist “just to stay alive.”

\textbf{Limit} \
Gray risks excessive pessimism: his dismissal of innovation and institutions leaves little room for small-state agency, which other authors (Krepinevich, Keohane, Thorhallsson) highlight.

\textbf{Consequent} \
Nonetheless, his argument grounds your Legitimacy Effect: Ireland’s influence depends less on drones, EU shelters, or peacekeeping symbolism, and more on aligning military means with coherent, legitimate political ends.

\section*{Mapping Gray (2005) into the Five Effects}

\textbf{Niche Specialisation Effect} \
Fit: Weak. Gray dismisses the idea that novelty or “new wars” create revolutionary opportunities. He stresses continuity, not rupture, implying that small states cannot “escape smallness” through narrow niches.
Use in Essay: Contrast with KREPINEVICH’s optimism about revolutions — Gray shows that niches (e.g., drones) will not matter unless tied to coherent political ends \parencite{GRAY_2005}.

\textbf{Organisational Agility Effect} \
Fit: Moderate. Gray accepts that militaries can adapt operationally (e.g., Germany’s ability to adjust in WWI/WWII), but insists that adaptability cannot overcome flawed political purpose.
Use in Essay: Supports your argument that Ireland’s Defence Forces’ “survivalist mindset” limits real agility. Organisational changes are insufficient if unconnected to political legitimacy \parencite{GRAY_2005}.

\textbf{Hybrid Leverage Effect} \
Fit: Indirect. Gray cautions against overhyping transformations like the RMA, warning that adversaries adapt asymmetrically and exploit unintended consequences.
Use in Essay: Useful to temper the idea that cyber, AI, or economic coercion grant small states leverage. Hybrid tools only matter within the political context and may provoke counter-adaptation \parencite{GRAY_2005}.

\textbf{Soft Power Synergy Effect} \
Fit: Weak. Gray underplays institutional or normative influence, focusing on the political and strategic logic of war. However, his scepticism can sharpen your use of KEoHANE and THORHALLSSON — by showing that soft power narratives risk being hollow without political-strategic substance \parencite{GRAY_2005}.

\textbf{Legitimacy Effect} \
Fit: Strongest. Gray insists that strategy is about the political consequences of military action, not military action itself. Without legitimate political ends, even operational and technological superiority collapses (e.g., Nazi Germany).
Use in Essay: This underpins your Legitimacy Effect: Ireland’s peacekeeping and neutrality only enhance influence if perceived as coherent political strategies, not symbolic survival \parencite{GRAY_2005}.

\section*{Comparative PEEL-C (GRAY 2005 vs THORHALLSSON 2006)}

\textbf{Point} \
Debate on small-state influence divides between sceptics who emphasise continuity and constraint, and optimists who highlight institutional and perceptual agency.

\textbf{Evidence} \
GRAY (2005) argues that the nature of war is unchanging and that technological or institutional innovations cannot overcome flawed political purposes; strategy is always about political consequences rather than operational brilliance \parencite{GRAY_2005}. By contrast, THORHALLSSON (2006) contends that small states amplify influence by leveraging perceptual and preference size within institutions such as the EU, showing that legitimacy and reputation can magnify their voice despite limited material capabilities \parencite{THORHALLSSON_2006}.

\textbf{Explain} \
Together, these perspectives illuminate the conditionality of small-state influence. THORHALLSSON shows how Ireland can project legitimacy through neutrality and peacekeeping, but GRAY cautions that without coherent political ends, such strategies risk being symbolic rather than substantive. Influence depends not just on institutions or perception but on the strategic purpose underpinning them.

\textbf{Limit} \
GRAY risks excessive pessimism, underplaying how institutions offer small states real agency, while THORHALLSSON risks over-optimism, assuming perceptions translate automatically into durable influence.

\textbf{Consequent} \
For Ireland, the synthesis implies that neutrality and EU engagement can enhance voice, but only if they are embedded in a legitimate political strategy. Otherwise, institutional participation remains hollow, confirming GRAY’s warning that military or diplomatic gestures are meaningless without political ends.

\section*{Comparative PEEL-C (GRAY 2005 vs KEOHANE 1969)}

\textbf{Point} \
At the systemic level, sceptics like GRAY emphasise structural continuity in war, while institutionalists such as KEOHANE argue that small states gain influence by embedding themselves in international organisations.

\textbf{Evidence} \
GRAY (2005) insists that the fundamental nature of war is eternal, warning that technological change, military transformation, or “new wars” cannot alter the primacy of politics and legitimacy \parencite{GRAY_2005}. By contrast, KEOHANE (1969) categorises small states as “system-affecting” actors that influence outcomes by shaping norms and behaviours through alliances and international institutions, even if they cannot unilaterally determine systemic outcomes \parencite{KEOHANE_1969}.

\textbf{Explain} \
Applied to Ireland, KEOHANE’s framework explains how neutrality and UN/EU membership allow Dublin to “punch above its weight” institutionally, while GRAY’s analysis warns that such influence remains fragile if disconnected from coherent political ends. Ireland’s peacekeeping and EU activism only translate into real security outcomes when embedded in legitimate political strategies, not as stand-alone gestures.

\textbf{Limit} \
GRAY risks excessive determinism by downplaying agency, while KEOHANE risks overestimating institutional voice, neglecting the constraints imposed by great-power dominance and strategic continuity.

\textbf{Consequent} \
For Ireland, the synthesis implies that international institutions can amplify influence, but only if their use is politically coherent and legitimate. Otherwise, symbolic neutrality and peacekeeping risk collapsing into irrelevance, as GRAY cautions.

\section*{DIMER Analysis — Cohen (2002)}

\textbf{Describe} \\
Cohen (2002) argues that civil--military relations cannot be understood through the ``normal theory,'' where civilians set goals and militaries simply execute. Instead, effective supreme command is both a \textit{process} of bureaucratic coordination (committees, staff, NSC) and a \textit{relationship} of ``unequal dialogue,'' where political leaders constantly probe and challenge military subordinates. This friction, although uncomfortable, produces better strategic outcomes \parencite{COHEN_2002}.

\textbf{Interpret} \\
For small states, Cohen’s emphasis underscores that legitimacy and influence are inseparable from political--military integration. Where militaries are excluded from strategy, as in Ireland, neutrality and peacekeeping risk becoming symbolic rather than strategic. His idea of supreme command as dialogue aligns with the \textbf{Legitimacy Effect} in the analytical framework.

\textbf{Methodology} \\
Conceptual--historical analysis, drawing on memoirs (Hankey, Livy), case studies (WWI, WWII, Kosovo), and theoretical critique. Strength: historically rich and conceptually sharp. Weakness: mostly Western great-power cases, with little on small states.

\textbf{Evaluate} \\
\textit{Contribution}: Cohen challenges the myth of non-interference, showing that civilian meddling is not only inevitable but desirable for sound strategy. \\
\textit{Weakness}: focus on US/UK/France makes application to small states indirect. Still, his framing of legitimacy and dialogue complements Gray (2005, 2018) by reinforcing that politics always dominates strategy.

\textbf{(Autho)R} \\
Eliot A. Cohen, professor at Johns Hopkins SAIS, influential in civil--military debates and US grand strategy. Bias: rooted in American great-power perspective, privileging Western democratic models. Influence: high --- the ``unequal dialogue'' is a widely cited concept.

\textbf{Limit $\rightarrow$ Implication} \\
\textit{Limit}: Geared toward great powers with globally relevant militaries. \\
\textit{Implication}: For this essay, Cohen bolsters the \textbf{Legitimacy Effect} --- Ireland’s marginalisation of its Defence Forces from political dialogue undermines credible strategy. His framework suggests that legitimacy requires active, even frictional, civil--military engagement.

\section*{PEEL-C Application (Legitimacy Effect)}

\textbf{Point} \\
Civil--military legitimacy depends not on harmony but on friction within an unequal dialogue.  

\textbf{Evidence} \\
Cohen (2002) shows that democratic leaders from Churchill to Clemenceau constantly prodded and clashed with generals, producing better strategy. He calls this the ``unequal dialogue'' at the heart of supreme command \parencite{COHEN_2002}.  

\textbf{Explain} \\
Applied to Ireland: excluding Defence Forces from strategic debate makes neutrality and peacekeeping symbolic gestures rather than strategic tools.  

\textbf{Limit} \\
Cohen’s model presumes militaries with weight in decision-making, unlike Ireland’s marginalised force.  

\textbf{Consequent} \\
Still, the lesson holds: legitimacy stems from active dialogue, not deference or exclusion. Ireland’s neglect of this dialogue undermines its credibility.  

\section*{Mapping Cohen (2002) into Effects}

\begin{itemize}
	\item \textbf{Niche Specialisation}: Minimal. Cohen’s focus is not on capabilities or innovation, but on the political context in which strategy is shaped.
	\item \textbf{Organisational Agility}: Indirect. By stressing that supreme command is a process of committees and staff structures, Cohen highlights how bureaucratic agility underpins strategy.
	\item \textbf{Hybrid Leverage}: Limited. While not addressing cyber or economic coercion, his emphasis on the political shaping of strategy implies that even hybrid tools require political legitimacy to matter.
	\item \textbf{Soft Power Synergy}: Moderate. Cohen implies that legitimacy in civil--military relations enhances credibility abroad, since friction and probing are necessary for coherent strategy that others respect.
	\item \textbf{Legitimacy Effect}: Strong. This is Cohen’s core contribution: legitimacy comes from frictional, probing dialogue between civilians and the military. Excluding the military undermines credibility, while over-harmony can mask abdication of responsibility \parencite{COHEN_2002}.
\end{itemize}

\section*{Comparative PEEL-C: Legitimacy Effect (Gray 2005 vs Cohen 2002)}

\textbf{Point} \\
Both Gray and Cohen argue that legitimacy is central to strategy, but they emphasise different mechanisms: Gray highlights contextual continuity, while Cohen stresses active civil--military dialogue.  

\textbf{Evidence} \\
Gray (2005) insists that the nature of war is unchanging and legitimacy derives from connecting political ends to military means, with context always decisive \parencite{GRAY_2005}.  
Cohen (2002) rejects the ``normal theory'' of civil--military separation, arguing that legitimacy comes from frictional engagement---an ``unequal dialogue''---between civilians and the military \parencite{COHEN_2002}.  

\textbf{Explain} \\
Together, these views suggest that small states such as Ireland cannot rely on symbolism alone (neutrality, token peacekeeping). Legitimacy requires both contextual alignment (Gray) and meaningful civil--military dialogue (Cohen). Without this, strategy risks being hollow.  

\textbf{Limit} \\
Gray’s analysis is highly abstract and rooted in great-power geopolitics. Cohen’s cases centre on major wartime leaders, which may not scale to small states with marginal militaries.  

\textbf{Consequent} \\
The synthesis strengthens the Legitimacy Effect: for Ireland, credibility requires grounding neutrality in context (Gray) while also engaging the Defence Forces in active, if unequal, dialogue with political leaders (Cohen). Otherwise, legitimacy risks being performative rather than substantive.  

\section*{Comparative PEEL-C: Gray (2005) vs Cohen (2002)}

\subsection*{Niche Specialisation Effect}

\textbf{Point} \\
Small states can exploit niche roles, but legitimacy and dialogue shape whether those roles endure.

\textbf{Evidence} \\
Gray (2005) stresses continuity: technological innovation alone cannot overcome political or contextual weakness \parencite{GRAY_2005}.  
Cohen (2002) highlights that legitimacy comes from an unequal dialogue where political leaders continually probe military subordinates \parencite{COHEN_2002}.  

\textbf{Explain} \\
Niche roles must be politically contextualised (Gray) and legitimised through civil--military dialogue (Cohen), or they risk being tokenistic.

\textbf{Limit} \\
Gray is abstract, focused on grand trends; Cohen assumes a large military establishment.

\textbf{Consequent} \\
Ireland’s niche contributions (peacekeeping, training) can be influential only if embedded in political legitimacy and civil--military dialogue.  

\subsection*{Organisational Agility Effect}

\textbf{Point} \\
Agility requires both adaptation to context and frictional legitimacy between leaders and military organisations.

\textbf{Evidence} \\
Gray (2005) warns against rigid adherence to single visions of future war, emphasising adaptability to uncertainty \parencite{GRAY_2005}.  
Cohen (2002) shows that probing civilian oversight forces militaries to adapt intellectually and organisationally \parencite{COHEN_2002}.  

\textbf{Explain} \\
Organisational agility is not only structural (Gray) but also relational (Cohen). For small states, this means military survivalist cultures must be challenged by political engagement.

\textbf{Limit} \\
Gray’s caveats are at the strategic level; Cohen’s evidence comes from great-power war cabinets.

\textbf{Consequent} \\
Ireland’s Defence Forces need not just structural agility but political dialogue to ensure relevance in crises.  

\subsection*{Hybrid Leverage Effect}

\textbf{Point} \\
Exploiting hybrid tools requires contextual awareness and legitimate civil--military integration.

\textbf{Evidence} \\
Gray (2005) highlights the enduring uncertainty of war and the limits of technological revolutions \parencite{GRAY_2005}.  
Cohen (2002) underscores that effective command emerges from the constant questioning of military advice by civilians \parencite{COHEN_2002}.  

\textbf{Explain} \\
Small states adopting hybrid tools (cyber, drones) must adapt to shifting contexts (Gray) and integrate them into legitimate political oversight (Cohen).

\textbf{Limit} \\
Gray downplays the disruptive potential of hybrid innovations; Cohen underplays how small states operate with limited means.

\textbf{Consequent} \\
For Ireland, hybrid tools amplify influence only if embedded in EU frameworks (context) and legitimised by political--military dialogue.  

\subsection*{Soft Power Synergy Effect}

\textbf{Point} \\
Soft power requires contextual continuity and legitimising political oversight.

\textbf{Evidence} \\
Gray (2005) notes that political and cultural contexts shape war more than technology \parencite{GRAY_2005}.  
Cohen (2002) shows legitimacy stems from leaders’ ability to question military advice and ensure alignment with political objectives \parencite{COHEN_2002}.  

\textbf{Explain} \\
Ireland’s soft power (neutrality, peacekeeping) depends on contextual credibility (Gray) and civil--military legitimacy (Cohen). Without both, soft power risks being symbolic.

\textbf{Limit} \\
Gray addresses global power shifts; Cohen focuses on high-intensity conflicts.

\textbf{Consequent} \\
Ireland’s soft power synergy is strongest when both external context (EU, UN) and internal legitimacy are aligned.  

\subsection*{Legitimacy Effect}

\textbf{Point} \\
Civil--military legitimacy requires contextual continuity and active dialogue.

\textbf{Evidence} \\
Gray (2005) insists that legitimacy comes from aligning military means with enduring political contexts \parencite{GRAY_2005}.  
Cohen (2002) demonstrates that legitimacy arises from tension and dialogue between political leaders and the military \parencite{COHEN_2002}.  

\textbf{Explain} \\
Together, they show that small states cannot rely on neutrality or peacekeeping as symbolic policies. Legitimacy requires political grounding (Gray) and engagement with the military (Cohen).

\textbf{Limit} \\
Gray’s framework is overly abstract; Cohen’s evidence is drawn from great powers.

\textbf{Consequent} \\
Ireland’s legitimacy effect depends on embedding neutrality in context and ensuring Defence Forces are engaged in political dialogue, avoiding symbolic marginalisation.

\section*{Concluding PEEL-C}

\textbf{Point} \\
Small states shape outcomes by balancing institutions, innovation, continuity, and legitimacy.  

\textbf{Evidence} \\
Keohane (1969, 1988) shows that institutions amplify influence through both rationalist and reflective mechanisms \parencite{KEOHANE_1969,KEOHANE_1988}.  
Thorhallsson (2006) highlights the importance of perceptions and shelter in enhancing credibility \parencite{THORHALLSSON_2006}.  
Krepinevich (1994) argues that organisational adaptation to revolutions offers opportunities for small states to ``steal a march'' on great powers \parencite{KREPINEVICH_1994}.  
Gray (2018) stresses legitimacy and the political essence of strategy, while Gray (2005) reminds us that continuity, context and uncertainty limit the transformative promise of technology \parencite{GRAY_2018,GRAY_2005}.  
Cohen (2002) reinforces this by showing that legitimacy stems from active, frictional civil--military dialogue rather than harmony or deference \parencite{COHEN_2002}.  

\textbf{Explain} \\
Taken together, these perspectives demonstrate that small states cannot rely on any single pathway to influence. Institutions provide shelter and voice (Keohane, Thorhallsson), innovation offers niche advantages (Krepinevich), but only when grounded in continuity and context (Gray 2005, 2018). Above all, legitimacy must be actively produced through both external institutional embedding and internal civil--military dialogue (Cohen).  

\textbf{Limit} \\
Each source is partial: Keohane and Thorhallsson privilege institutions, Krepinevich risks technocentrism, Gray can be overly deterministic, and Cohen focuses on great-power cases.  

\textbf{Consequent} \\
The synthesis supports a multi-dimensional framework: Ireland’s strategic influence depends on combining institutional shelter, selective innovation, and above all, legitimacy rooted in context and unequal dialogue. This balanced approach avoids symbolic neutrality and grounds Irish strategy in credible practice.  


\section*{DIMER Analysis — HLAP (2022)}

\textbf{Describe} \\
The High Level Action Plan (HLAP, 2022) is the Irish Government’s official response to the Commission on the Defence Forces. It accepts or evaluates 130 recommendations, structured around five strategic objectives: (1) HR and cultural transformation, (2) new command and control structures, (3) service reform, (4) revitalisation of the Reserve, and (5) joint capability development. It commits Ireland to move to Level of Ambition 2 (LOA2) by 2028, raising the defence budget to €1.5bn and adding 2,000 personnel \parencite{HLAP_2022}.  

\textbf{Interpret} \\
For small-state strategy, the HLAP highlights a shift from survivalist minimalism to cautious modernisation. It acknowledges chronic legitimacy gaps — cultural issues, weak HR structures, and command ambiguities — while embedding Ireland’s Defence Forces in EU/NATO cooperative frameworks. The plan embodies both \textbf{organisational agility} (new structures, HR reform) and \textbf{legitimacy} (addressing culture, diversity, oversight), but also underscores reliance on external shelter (EU capability development, PESCO).  

\textbf{Methodology} \\
The HLAP is a government policy document — programmatic, not analytical. It aggregates recommendations into an implementation plan with phased timelines and oversight structures. Its strength lies in being an authoritative primary source for Ireland’s official strategy. Its weakness is aspirational bias: progress is contingent on funding, political will, and cultural change.  

\textbf{Evaluate} \\
Contribution: The HLAP is crucial for grounding theory (Gray’s focus on ends–means mismatch; Bailes \& Thorhallsson on shelter) in Ireland’s contemporary defence policy. It shows institutional reform as the chosen route for small-state resilience, with modest military ambition. Weakness: It sidesteps the deeper neutrality debate, and risks repeating Ireland’s pattern of under-delivery versus stated policy.  

\textbf{(Autho)R} \\
Department of Defence and Defence Forces, issued under Ministerial authority. Institutional bias: pro-incremental reform, cautious about LOA3. Political context: 2022, amid Russian aggression in Ukraine, EU calls for “strategic autonomy,” and domestic criticism of Defence Forces’ crisis.  

\textbf{Limit → Implication} \\
\textit{Limit}: Aspirational, resource-dependent, and avoids revisiting neutrality doctrine.  
\textit{Implication}: For this essay, the HLAP exemplifies how small states use institutional reform and EU frameworks to bolster legitimacy and niche capability, but also how limited resources and domestic constraints inhibit strategic autonomy.  

\section*{Mapping HLAP into Effects}

\begin{itemize}
	\item \textbf{Niche Specialisation}: Incremental capacity in cyber, naval patrols, and peacekeeping.  
	\item \textbf{Organisational Agility}: Structural reform (CHOD, Joint HQ, HR transformation).  
	\item \textbf{Hybrid Leverage}: Cyber defence, EU capability initiatives, PESCO.  
	\item \textbf{Soft Power Synergy}: EU and UN embedding, diversity and cultural reform for legitimacy.  
	\item \textbf{Legitimacy}: Cultural change, oversight mechanisms, stakeholder engagement central.  
\end{itemize}

\section*{PEEL-C Paragraph (Application)}

\textbf{Point} \\
The HLAP (2022) illustrates Ireland’s attempt to balance modest capability growth with legitimacy-driven reform.  

\textbf{Evidence} \\
The plan commits to LOA2 by 2028, creating a Chief of Defence, Joint HQ, revitalised Reserve, and major HR/cultural reform, while embedding Ireland in EU frameworks such as PESCO and capability development \parencite{HLAP_2022}.  

\textbf{Explain} \\
For small-state strategy, this reflects a synthesis of institutional shelter, niche specialisation, and legitimacy-building. Ireland aims to modernise without abandoning neutrality, relying on EU frameworks to amplify influence and cover vulnerabilities.  

\textbf{Limit} \\
However, implementation depends on resources and political will, and neutrality tensions remain unresolved. The HLAP risks over-promising and under-delivering, repeating past disconnects between ambition and means.  

\textbf{Consequent} \\
Thus, HLAP (2022) underscores that small states derive influence less from raw capability than from legitimacy, institutional embedding, and carefully managed reforms — but credibility hinges on delivering tangible change.  

\section\*{DIMER Analysis — White Paper on Defence (2015) and Update (2019)}

\textbf{Describe} \\
The White Paper on Defence (2015) provided Ireland with a ten–year strategic framework up to 2025, reaffirming military neutrality while committing to international engagement through the UN, EU, NATO Partnership for Peace, and OSCE \parencite{WHITE_2015}. The 2019 Update reviewed implementation, reaffirmed the fundamentals, but highlighted serious challenges in recruitment, retention, and capability gaps (especially in Air Corps, Naval Service, and cyber). It also noted expanded overseas operations, deeper EU integration through PESCO, and Brexit–related uncertainties \parencite{WHITE_2015}.

\textbf{Interpret} \\
The White Paper positions neutrality not as isolation but as active multilateral engagement. For small states like Ireland, this reflects a strategy of \`\`institutional shelter’’ — embedding in EU and UN frameworks to amplify legitimacy while compensating for weak military means. However, the 2019 Update underscores the fragility of this model: chronic under–staffing and reliance on symbolism (peacekeeping, naval patrols) raise doubts about the Defence Forces’ ability to sustain commitments. Neutrality thus functions as both soft power and cost–saving mechanism, but risks hollowing out substantive capability.

\textbf{Methodology} \\
These are state policy documents — declaratory, consensus–driven, and programmatic. They reflect government priorities more than independent analysis. Their strength lies in signalling political intent and resource allocation; their weakness is aspirational overreach and political spin (e.g. neutrality framed as \`\`active’’ while military underinvestment persists).

\textbf{Evaluate} \\
Contribution: The White Paper and Update reveal how Ireland instrumentalises neutrality and EU frameworks as part of its security identity, aligning with Bailes and Thorhallsson’s shelter theory. They also show structural constraints: persistent personnel shortages, capability gaps, and dependence on multilateral legitimacy. Weakness: the documents avoid confronting strategic trade–offs — neutrality vs. EU integration, capability vs. symbolism, or autonomy vs. shelter.

\textbf{(Autho)R} \\
Official documents by the Department of Defence and Government of Ireland. Bias: self–justifying, focused on portraying progress and stability. Influence: high — they are the authoritative framework for Irish defence policy and widely cited in parliamentary debates.

\textbf{Limit → Implication} \\
\textit{Limit}: Aspirational framing obscures capability shortfalls and strategic incoherence.
\textit{Implication}: For this essay, the White Paper is crucial for illustrating the \textbf{Legitimacy Effect} (neutrality as soft power) and \textbf{Organisational Agility Effect} (institutional adaptation to EU frameworks), but it also demonstrates the limits of symbolic strategies when material gaps persist.

\section\*{PEEL–C Paragraph (Application to Legitimacy and Organisational Agility)}

\textbf{Point} \\
Ireland uses neutrality and EU membership to project legitimacy, but persistent capability shortfalls undermine credibility.

\textbf{Evidence} \\
The 2019 Update acknowledges Defence Forces strength at just 8,762, well below the 9,500 target, with critical gaps in air, naval, and cyber roles \parencite{WHITE_2015}. Yet it highlights Ireland’s deepened EU role through PESCO, naval deployments in the Mediterranean, and continued peacekeeping.

\textbf{Explain} \\
This duality shows neutrality instrumentalised as legitimacy: internationally, Ireland appears engaged and principled; domestically, neutrality justifies low investment. Organisational agility is evident in adapting to EU frameworks, but operational delivery is constrained by underfunding and retention crises.

\textbf{Limit} \\
As a government policy document, the White Paper avoids questioning whether neutrality is sustainable given capability gaps, or whether EU integration erodes traditional non–alignment.

\textbf{Consequent} \\
Thus, while the White Paper illustrates how small states leverage legitimacy and institutional shelter, it also shows that without addressing material shortfalls, neutrality risks becoming symbolic rather than strategic.

\section*{Mapping White Paper on Defence (2015/2019) into Effects}

\begin{itemize}
	\item \textbf{Niche Specialisation Effect} \\
	The White Paper highlights Ireland’s peacekeeping, naval patrols, and training missions as niche contributions. These reinforce Ireland’s international visibility despite limited resources, but the 2019 Update shows that overstretch and under–manning threaten sustainability \parencite{WHITE_2015}.
	
	\item \textbf{Organisational Agility Effect} \\
	Ireland adapts institutionally by embedding in EU initiatives such as PESCO, MPCC, and CARD, signalling agility in aligning with collective frameworks. However, recruitment and retention crises reveal limited agility at the operational level \parencite{WHITE_2015}.
	
	\item \textbf{Hybrid Leverage Effect} \\
	The documents acknowledge non–traditional threats (cyber, pandemics, climate) but offer limited capacity to address them, exposing vulnerability. Reliance on EU frameworks hints at potential leverage through regulatory and multilateral approaches, but agency remains weak \parencite{WHITE_2015}.
	
	\item \textbf{Soft Power Synergy Effect} \\
	Neutrality and peacekeeping are consistently framed as Ireland’s strategic identity, projecting credibility within UN and EU contexts. The White Paper illustrates how soft power is institutionalised into defence policy as a substitute for hard power \parencite{WHITE_2015}.
	
	\item \textbf{Legitimacy Effect} \\
	Neutrality serves as the cornerstone of domestic and international legitimacy. However, the 2019 Update’s admission of critical capability shortfalls risks hollowing out this legitimacy if rhetoric is not matched by credible capability \parencite{WHITE_2015}.
\end{itemize}

\section*{DIMER Analysis — Department of Defence (2025–2028)}

\textbf{Describe} \\
The 2025--2028 Strategy Statement sets the strategic direction for Ireland’s Department of Defence and Defence Forces. It operationalises the recommendations of the Commission on the Defence Forces (2022), confirming a move towards Level of Ambition 2 (LOA2) by 2028, with the possibility of advancing to LOA3. Priorities include maritime security, integrated monitoring and surveillance (air, land, sea), revitalisation of the Reserve Defence Force, cultural reform, cyber defence, and climate action. The document emphasises neutrality, multilateralism, and whole-of-government resilience \parencite{DOD_2025}.

\textbf{Interpret} \\
For the presentation, this shows Ireland institutionalising transformation after years of under-investment. The focus is dual: building credible national defence (radar, cyber, naval renewal) while sustaining legitimacy through neutrality, peacekeeping, and EU/UN engagement. The explicit link to resilience and climate action illustrates the broadening of defence beyond traditional hard power. However, it also demonstrates that Ireland’s strategy remains one of ``shelter’’ within EU/UN frameworks rather than autonomous power projection.

\textbf{Methodology} \\
This is a strategic policy document, not empirical research. It sets programmatic goals, funding commitments, and timelines. Strength: authoritative, binding on Defence policy. Weakness: aspirational tone, may outpace actual implementation (notably recruitment/retention and capability gaps).

\textbf{Evaluate} \\
Contribution: The Strategy Statement institutionalises transformation, providing concrete indicators (funding to €1.5bn, radar, cyber command, naval regeneration). It embeds defence within whole-of-government resilience and EU multilateralism, reinforcing small-state shelter theory. Weakness: neutral framing and resource constraints may limit delivery, risking a gap between rhetoric and capacity.

\textbf{(Autho)R} \\
Produced jointly by the Department of Defence and Defence Forces leadership (civil–military). Bias: strongly pro-reform and pro-institutionalist. Influence: high—this is the official guiding document for Ireland’s defence through 2028.

\textbf{Limit → Implication} \\
\textit{Limit}: Aspirational targets (11,500 personnel, radar, fleet renewal) may exceed Ireland’s fiscal and political capacity.  
\textit{Implication}: For the presentation, this illustrates the tension between ambition and credibility in small-state strategies: legitimacy is strong on paper, but organisational and resource gaps persist.

\section*{Mapping into Effects}

\begin{itemize}
	\item \textbf{Niche Specialisation}: Peacekeeping, training missions, maritime surveillance framed as Ireland’s contributions.
	\item \textbf{Organisational Agility}: Major reforms to command, culture, and HR show institutional adaptation.
	\item \textbf{Hybrid Leverage}: Cyber defence strategy, anti-drone, and subsea awareness projects respond to hybrid threats.
	\item \textbf{Soft Power Synergy}: Neutrality, peacekeeping, and EU Presidency (2026) enhance Ireland’s visibility and credibility.
	\item \textbf{Legitimacy}: Neutrality reaffirmed as central; legitimacy reinforced by reforms (tribunal of inquiry, oversight body, IRG implementation).
\end{itemize}

\section*{PEEL-C Paragraph (Application to Legitimacy)}

\textbf{Point} \\
Legitimacy in civil–military affairs requires not only neutrality but demonstrable reform and accountability.  

\textbf{Evidence} \\
The Department of Defence Strategy Statement 2025--2028 reaffirms neutrality while committing to cultural reform, new oversight mechanisms, and implementation of the Commission on the Defence Forces recommendations. It links legitimacy to transparency, diversity, and public trust as well as neutrality’s international resonance \parencite{DOD_2025}.  

\textbf{Explain} \\
For Ireland, legitimacy is twofold: externally, neutrality and peacekeeping continue to frame it as a principled small state; internally, reforms to culture, HR, and oversight aim to rebuild public confidence in the Defence Forces. This reflects the shift from symbolic neutrality towards substantive legitimacy rooted in credible reform.  

\textbf{Limit} \\
However, resource constraints and recruitment crises may undermine delivery, exposing a gap between aspirational legitimacy and operational credibility.  

\textbf{Consequent} \\
Thus, the 2025--2028 Strategy Statement illustrates that for small states, legitimacy is a strategic asset, but one that must be matched by organisational delivery to sustain influence.  

\begin{table}[h!]
	\centering
	\caption{Mapping the DoD Strategy Statement 2025--2028 into Framework Effects}
	\begin{tabular}{|p{3cm}|p{11cm}|}
		\hline
		\textbf{Effect} & \textbf{Application from DoD Strategy Statement 2025--2028 \parencite{DOD_2025}} \\
		\hline
		Niche Specialisation & Commitment to UN peacekeeping and crisis-management missions highlights Ireland’s continued use of neutrality and international service as its strategic niche. \\ 
		\hline
		Organisational Agility & Reform agenda (culture, HR, oversight) seeks to address recruitment/retention and modernise force structures; agility framed as institutional transformation rather than battlefield manoeuvre. \\ 
		\hline
		Hybrid Leverage & Recognition of cyber and emerging security threats; investment in digital resilience and cross-governmental partnerships shows how Ireland leverages networks rather than hard power. \\ 
		\hline
		Soft Power Synergy & Emphasis on EU and UN cooperation, climate security, and humanitarian roles enhances Ireland’s visibility and credibility as a principled small state. \\ 
		\hline
		Legitimacy & Neutrality reaffirmed externally, while domestic legitimacy targeted through transparency, diversity, and implementation of Commission on the Defence Forces reforms. Gaps remain if resource constraints undermine credibility. \\ 
		\hline
	\end{tabular}
\end{table}

\section*{DIMER Analysis — Nye (2008)}

\textbf{Describe} \\
Nye (2008) defines soft power as the ability to shape others’ preferences through attraction rather than coercion or payment. Its sources lie in culture, political values, and foreign policies that are seen as legitimate. Public diplomacy is a tool of soft power, encompassing communication, strategic messaging, and long-term relationship-building. Nye argues that credibility and consistency are vital: propaganda or contradictory policies can undermine attraction. He introduces the notion of ``smart power’’—the effective combination of hard and soft power resources \parencite{NYE_2008}.

\textbf{Interpret} \\
For small states, this framework underscores that legitimacy and credibility are as important as resources. While they lack hard power, small states can amplify their voice through values, cultural diplomacy, and principled foreign policies. Ireland’s neutrality, peacekeeping, and EU membership can thus be read as soft power strategies, providing legitimacy and attraction that outweigh material weakness. Nye’s emphasis on ``credibility as the scarcest resource’’ reinforces the essay’s Legitimacy and Soft Power Synergy Effects.

\textbf{Methodology} \\
This article is a conceptual essay, rooted in theory and historical examples (e.g., USIA, Cold War broadcasting, cultural diplomacy). Its strength is conceptual clarity and applicability across cases. Weakness: it is US-centric and assumes that values/policies can be credibly projected without deeper structural critique of power hierarchies.

\textbf{Evaluate} \\
Contribution: Establishes soft power as a central analytical tool for IR, shifting debates from coercion to attraction. Weakness: risks idealising soft power and downplaying how even small states’ attraction is mediated by larger institutional or structural contexts. Nonetheless, it provides a normative and empirical rationale for small-state influence strategies.

\textbf{(Autho)R} \\
Joseph Nye, Harvard professor and former US policymaker, is the originator of soft power theory. His perspective is liberal-institutionalist, favouring cooperative and normative explanations of influence. Bias: US-centric, but influential globally.

\textbf{Limit → Implication} \\
\textit{Limit}: US-centric; underplays structural constraints faced by small states. \\
\textit{Implication}: For this essay, Nye clarifies how Ireland’s neutrality and peacekeeping constitute resources of attraction, but these only translate into influence if credibility is sustained. This aligns with the Legitimacy and Soft Power Synergy Effects.

\section*{PEEL-C Paragraph (Application to Soft Power Synergy and Legitimacy)}

\textbf{Point} \\
Small states extend influence through credibility and attraction rather than coercion, amplifying their limited material means.

\textbf{Evidence} \\
Nye (2008) shows that public diplomacy and cultural legitimacy can ``co-opt rather than coerce’’ others, provided policies and values are credible. He stresses that credibility, not slick propaganda, is the key resource of soft power \parencite{NYE_2008}.

\textbf{Explain} \\
For Ireland, neutrality, peacekeeping, and EU engagement function as credible signals of principled behaviour, attracting support and legitimacy disproportionate to its size. This explains why Irish influence rests less on material force and more on perceived values and identity.

\textbf{Limit} \\
Nye’s framework overlooks how structural dependencies (e.g., reliance on US-led networks or EU shelters) constrain small states’ ability to project soft power autonomously.

\textbf{Consequent} \\
Thus, Nye underpins the essay’s Soft Power Synergy and Legitimacy Effects: Ireland’s influence derives from credibility and attraction, but risks erosion if domestic legitimacy falters or external dependencies expose contradictions.

\section*{Mapping Nye (2008) into Effects}

\begin{itemize}
	\item \textbf{Niche Specialisation}: Limited relevance. Nye does not focus on material military niches, but one can infer that small states can specialise in cultural or normative diplomacy as their ``niche’’ contribution. 
	\item \textbf{Organisational Agility}: Indirect. Public diplomacy requires institutions that can adapt quickly to reputational challenges and communicate consistently; small states must develop agile foreign ministries and information services to maintain credibility. 
	\item \textbf{Hybrid Leverage}: Minimal direct application. Nye emphasises attraction over coercion, but in practice small states can combine soft power narratives with economic or institutional leverage to offset vulnerabilities. 
	\item \textbf{Soft Power Synergy}: Central. Nye (2008) explicitly frames culture, values, and legitimate policies as sources of influence, stressing that credibility is the scarcest resource of soft power. This directly validates the Soft Power Synergy Effect. 
	\item \textbf{Legitimacy}: Strong. Nye’s insistence that propaganda undermines attraction highlights legitimacy as foundational. For Ireland, neutrality and peacekeeping derive power not from symbolism alone but from their credibility as consistent, legitimate policies \parencite{NYE_2008}.
\end{itemize}

\section*{Comparative PEEL-C (Nye vs. Gray)}

\textbf{Point} \\  
Both Nye and Gray highlight that legitimacy is central to the exercise of power, but they frame it differently: Nye (2008) grounds it in attraction and credibility, while Gray (2005) insists it derives from the political context that gives war meaning.  

\textbf{Evidence} \\  
Nye (2008) argues that ``credibility is the scarcest resource of soft power,'' and that propaganda erodes attraction, weakening a state’s influence. Gray (2005), by contrast, stresses that war and strategy cannot be divorced from their political and cultural contexts; strategic success depends on legitimacy in those domains, not just on military prowess \parencite{NYE_2008,GRAY_2005}.  

\textbf{Explain} \\  
For small states such as Ireland, this convergence underscores that neutrality and peacekeeping only generate influence when they are credible to both domestic and external audiences. Nye provides the normative rationale for credibility as a resource of attraction, while Gray reminds us that without political legitimacy, military or diplomatic gestures collapse into symbolism.  

\textbf{Limit} \\  
Nye underplays the constraints of power asymmetry, while Gray offers little on communicative or reputational dimensions.  

\textbf{Consequent} \\  
Together, they reinforce the \textbf{Legitimacy Effect}: Ireland’s ability to project influence rests not only on normative attraction but also on embedding those practices in meaningful political contexts that align ends, ways, and means.


\section*{Comparative PEEL-C (Nye vs. Farrell \& Newman)}

\textbf{Point} \\  
Nye (2008) views globalisation as expanding the scope of soft power through attraction and credibility, whereas Farrell \& Newman (2019) reveal its darker side: asymmetric networks weaponised by great powers to coerce others.  

\textbf{Evidence} \\  
Nye (2008) argues that legitimacy and credibility generate influence that cannot be coerced, emphasising that attraction often yields more sustainable outcomes than force. Farrell \& Newman (2019), however, show how the same interdependence can be exploited through \textit{panopticon} and \textit{chokepoint} effects, citing US pressure on SWIFT to disconnect Iranian banks \parencite{NYE_2008,FARRELL_2019}.  

\textbf{Explain} \\  
For Ireland, this contrast highlights a strategic dilemma: neutrality and peacekeeping rely on soft power credibility, but deep embedding in EU and US-centric financial and digital networks exposes it to coercion. Soft power can amplify Ireland’s voice, but weaponised interdependence limits its autonomy.  

\textbf{Limit} \\  
Nye is overly optimistic about the benign character of global networks, while Farrell \& Newman largely discount the agency of small states in shaping regulatory frameworks.  

\textbf{Consequent} \\  
Together, they justify the \textbf{Hybrid Leverage Effect}: Ireland’s influence depends on using institutional platforms (e.g., GDPR, UN peacekeeping) to convert network vulnerability into opportunities for normative leadership, balancing attraction with regulatory shelter.



\section*{Synthesis Mapping: Nye (2008) and Farrell \& Newman (2019)}

\begin{itemize}
	\item \textbf{Niche Specialisation}: Nye (2008) implies that small states can cultivate distinctive reputations in peacekeeping or mediation as sources of attraction, while Farrell \& Newman (2019) warn that such niches remain constrained by structural vulnerabilities in finance and digital networks \parencite{NYE_2008,FARRELL_2019}.
	
	\item \textbf{Organisational Agility}: Nye suggests that credibility derives from consistent behaviour, requiring institutions to adapt and maintain reputational capital. Farrell \& Newman highlight how small states must adjust organisationally within EU and UN platforms to mitigate exposure to coercive chokepoints.
	
	\item \textbf{Hybrid Leverage}: Nye stresses persuasion over coercion, but Farrell \& Newman show that interdependence can be weaponised. For small states, hybrid leverage means exploiting multilateral regulatory frameworks (e.g., GDPR) to constrain coercion while projecting normative authority.
	
	\item \textbf{Soft Power Synergy}: Nye places legitimacy and attraction at the heart of influence, aligning directly with the synergy effect. By contrast, Farrell \& Newman demonstrate that without structural safeguards, soft power risks being overridden by coercive interdependence. The synergy lies in coupling legitimacy with institutional shelter.
	
	\item \textbf{Legitimacy}: Nye anchors influence in credibility and attraction, while Farrell \& Newman expose how legitimacy can be undermined when great powers dominate networks. For Ireland, legitimacy depends on demonstrating principled neutrality and peacekeeping within EU and UN frameworks, while also shaping norms that discipline coercive practices.
\end{itemize}

\textbf{Synthesis}: Together, Nye (2008) and Farrell \& Newman (2019) show that soft power cannot be understood apart from structural vulnerabilities. For small states such as Ireland, the challenge is to translate reputational capital into institutionalised safeguards, ensuring that attraction and legitimacy endure in an environment where interdependence is a source of both influence and coercion.








\section*{Gray C. S, Theory of Strategy, 2018}

\parencite{GRAY_2018}Tactics are always and solely about the actual conduct of military action, on any scale and for any purpose, employing weapons of any character.  
Operations are always about the direction, indeed the orchestration, of any military action (tactics).  
Strategy is about (political) consequences of tactical and operational military behaviour. (because strategy is most profoundly about consequences, there can be no such phenomenon as a strategic action).  

Periods of peace are not novelties invented and practiced only in the modern world. It is depressing to recognize that the history of our species has been all but shattered from time to time by great wars.  

Gray seems to value the key general less than the system.  
Clausewitz got it wrong by only using contemporary situations to draw a general theory.  

Probably the single most useful navigation aid for the researcher and theorist is the elementary conceptual triad comprising Ends, Ways, and Means. When the ideas of Assumptions and Consequences are added to the core three we have the rather bare basis of a usable method for exploring.  

Political ends, served by strategic ways, employ military means with the whole activity largely governed by relevant assumptions.  

Grand strategy is understood intelligently as a very occasional aspiration conceived and effected for a particular, extraordinary case of state need.  

His general theory does not account for nuclear weapons.  

\begin{enumerate}
	\item Grand strategy is the direction and use made of many or all the assets of a security community including its military instrument for the purposes of policy decided at the political level.
	\item Military strategy is the direction and use made of force and the threat of force for the purposes of policy decided at a political level.
	\item Strategy is the only bridge that connects policy and its politics with military power. 
	\item Strategic effect is a concept as essential and elusive as it has become unduly commonplace often in misuse.
	\item Strategy is always human.
	\item Strategy is not about threat and action; rather, it is about consequence of such behaviour.
	\item Strategy needs an enemy.
	\item Chaos, adaptability and surprise are persistent features in strategic matters.
	\item Strategy’s general theory is supplemented by strategics specifically for application in different environments.
	\item High intensity of feeling or passion is commonly an accompaniment to, if not always itself a case of, war.
	\item Personality can figure positively or negatively in strategy, but the relevance of culture remains controversial. It is possible for culture to play a significant strategic role.
	\item All societies as well as many institutions within them claim to hold to an ethical code. 
	\item Political, including diplomatic engagement, is complement to strategy.
	\item Intelligence and deception are permanent features in the character of strategy.
	\item The relative challenges differ among policy, strategy, operations and tactics.
	\item All strategy is based on some assumptions.
	\item Strategy does not change, but implementing operations and tactics certainly does.
	\item Strategy always has geographical context with political meaning.
	\item Technology for weapons does not win wars but it can usually help significantly.  
	\begin{enumerate}
		\item Weapons are no more than tactical tools that must have some strategic consequence in use, great or small.
		\item The strategist in all ages has recognized the potential value of a technical lead over an enemy.
		\item While there is virtue in a quest after the best technical answer to essentially technical problems, strategically focused communities need to recognize that the technology best suited to operation in the field of probable action is that known and trusted by troops to be good enough.
		\item Technologists always strive to advance system performance because that is what they have been educated to do.
		\item What the strategist seeks are affordable weapons for both offence and defence that soldiers only average in competence and motivation can use when tired.
		\item Had military technology alone been sufficient to achieve victory in the land war, then Germany would have been victorious. The truly appalling German political cause was beyond rescue even by the imperiously unmatched quality of the Panther and Tiger tanks in particular. Despite the fact that the Wehrmacht conducted itself with superior operational art, tactics and generalship overall, the war was so far lost politically – and therefore strategically – that no measure of technological superiority could rescue the Reich.
	\end{enumerate}
	\item All strategy has both temporary and broader historical contexts.
	\item Supply and movement are fundamental enablers of strategy.
	\item Doctrine expresses what is believed to be best contemporary practice in tactics and operations but not in strategy.
\end{enumerate}

\section*{DIMER Analysis of Bessner (2015)}

\subsection*{Describe}
\textcite{BESSNER_2015} interrogates the trajectory of American grand strategy and the intellectual traditions underpinning it. He contends that realism, liberalism, and their institutional incarnations obscure the persistence of U.S. hegemonic ambitions. The article explores how academic theorists have historically provided justificatory narratives that legitimate the projection of power. Its central claim is that American strategy is characterised less by coherent doctrine than by intellectual legitimation serving policy needs.

\subsection*{Interpret}
The analysis applies primarily to great powers such as the United States, but its implications matter for small states indirectly. For Ireland and other small states, the piece highlights how dominant powers set the strategic discourse, narrowing the space for alternative narratives. The lacuna is its neglect of small-state agency: Bessner focuses on the intellectual architecture of U.S. hegemony, not on how minor powers can navigate or contest it. This matters because in Irish Defence and Strategic Studies, influence depends not only on material contributions but also on how small states position themselves within hegemonic discourses.

\subsection*{Methodology}
This is a theoretical-historical article, drawing on intellectual history and secondary literature. It is best situated as commentary and synthesis rather than empirical research. On the hierarchy of evidence, it ranks as expert opinion informed by historical case references. The method is strong for tracing ideas and narratives but weak in empirical testing, limiting external applicability to small states or policy contexts outside the U.S.

\subsection*{Evaluate}
The contribution lies in exposing the link between strategy and intellectual legitimation. Unlike structural realists (e.g. Waltz), Bessner foregrounds the ideational layer. However, compared to constructivists such as Wendt, his analysis is narrower: he sees ideas primarily as instruments of great-power dominance rather than as intersubjective structures shaping all actors. For Defence Forces analysis, this raises a contrast: while Bessner sees legitimacy as a hegemon’s narrative tool, small states rely on legitimacy as their centre of gravity. Thus, the value of the piece lies in offering a foil against which to test small-state frameworks.

\subsection*{(Autho)R}
Bessner is a historian of U.S. foreign policy at the University of Washington, with a critical approach to American exceptionalism. His institutional background and disciplinary lens bias him toward scepticism of hegemonic strategies. This makes him a useful counterbalance to optimists like Nye or Thorhallsson, but less helpful for understanding small-state agency. His critique of legitimising narratives underlines the importance of recognising how intellectual discourse can constrain Irish strategic autonomy.

\subsection*{Limit → Implication}
Limit: The article focuses exclusively on U.S. hegemonic traditions, ignoring small-state dynamics.  
Implication: For Ireland, the relevance lies not in mirroring the U.S. but in understanding how hegemonic narratives shape the space for small-state legitimacy.

\subsection*{PEEL-C Application to Framework}
\textbf{Point:} Bessner demonstrates that legitimacy is often a narrative tool wielded by great powers to rationalise hegemony.  
\textbf{Evidence:} His critique shows how American academics embedded liberal-internationalist rhetoric into strategic discourse \parencite{BESSNER_2015}.  
\textbf{Explain:} For small states, this indicates that legitimacy cannot be assumed neutral: it is contested terrain defined by stronger actors.  
\textbf{Limit:} The analysis underplays how small states like Ireland use neutrality and peacekeeping to build bottom-up legitimacy.  
\textbf{Consequent:} Within the five-effects framework, Bessner underscores why legitimacy is fragile for small states—it can be co-opted by hegemonic narratives, making strategic clarity and institutional embedding essential.

\section*{Mapping Bessner (2015) into the Five Effects Framework}

\subsection*{Niche Specialisation}
\textcite{BESSNER_2015} suggests that great powers monopolise the intellectual space by legitimising their own strategic preferences. For small states, this constrains the scope of niche specialisation: peacekeeping or cyber expertise risks being interpreted as support for hegemonic agendas rather than independent contributions. Thus, niches are vulnerable to appropriation by dominant narratives.  
\textbf{Limit → Implication:} Limit: Bessner underplays small-state agency. Implication: Ireland must ensure niches are framed as autonomous contributions tied to neutrality, not subordinated to external legitimising projects.

\subsection*{Organisational Agility}
Bessner indirectly highlights how intellectual traditions shape reform debates. Small states seeking agility may face structural limits if hegemonic narratives define the parameters of what is considered legitimate reform. For example, agility through EU PESCO participation might be reframed as alignment with U.S. or NATO strategic logics.  
\textbf{Limit → Implication:} Limit: No engagement with small-state reform. Implication: Ireland’s agility must include narrative agility, resisting co-option into great-power frames.

\subsection*{Hybrid Leverage}
Although hybrid strategies are not Bessner’s direct concern, his critique of legitimising narratives has implications. Great powers often delegitimise hybrid tactics (cyber, disinformation) when used by others, while legitimising their own. For small states, this double standard constrains the effectiveness of hybrid leverage.  
\textbf{Limit → Implication:} Limit: Absence of hybrid focus. Implication: Ireland’s hybrid posture must prioritise regulatory and normative tools (e.g., GDPR) that maintain legitimacy rather than relying on contested coercive instruments.

\subsection*{Soft Power Synergy}
Bessner’s argument reveals how liberal rhetoric serves U.S. dominance. For small states, this challenges Nye’s assumption that soft power is neutral attraction. Instead, soft power synergy must be carefully managed to avoid co-optation. Irish neutrality and peacekeeping can project attraction, but only if framed independently of hegemonic liberal narratives.  
\textbf{Limit → Implication:} Limit: Focuses only on hegemonic soft power. Implication: Ireland must deploy soft power as identity-driven rather than derivative of great-power discourse.

\subsection*{Legitimacy}
The central lesson from Bessner is that legitimacy is contested and often instrumentalised by dominant actors. For small states, legitimacy is the centre of gravity, but it is fragile if not rooted in coherent policy. Ireland’s neutrality, peacekeeping, and reforms gain influence precisely because they resist hegemonic appropriation.  
\textbf{Limit → Implication:} Limit: Legitimacy treated as top-down tool of hegemons. Implication: For Ireland, legitimacy must be bottom-up and embedded in institutions (UN, EU) to insulate against narrative capture.

\subsection*{Synthesis}
Bessner’s analysis provides a cautionary counterweight to optimists like Nye and Thorhallsson. While he underestimates small-state agency, his emphasis on hegemonic narrative control underscores why legitimacy must anchor the other four effects. For Ireland, success depends on ensuring that niche specialisation, agility, hybrid leverage, and soft power are all framed in ways that reinforce independent legitimacy rather than reproducing great-power discourse.

\section*{DIMER Analysis of Mearsheimer (1994)}

\subsection*{Describe}
\textcite{MEARSHEIMER_1994} argues that international institutions have minimal independent effect on state behaviour. They reflect the distribution of power rather than shaping it. Examining liberal institutionalism, collective security, and critical theory, he concludes that each rests on flawed causal logics and lacks empirical support. Institutions may appear influential, but for realists they are arenas where great powers act out underlying power politics.

\subsection*{Interpret}
The article’s central message is that power, not institutions, determines outcomes. For small states, this is sobering: their strategy of leveraging institutions (UN, EU) to magnify influence appears futile from Mearsheimer’s perspective. Yet, indirectly, it underscores why legitimacy and institutional embedding matter so much—without these, small states are even more invisible. In an Irish context, peacekeeping or EU membership may not shift great-power behaviour but still provide platforms to project credibility.

\subsection*{Methodology}
Mearsheimer employs theoretical critique and selective historical cases (NATO, the League of Nations, OECD cooperation). His approach is conceptual and polemical rather than empirical. The strength lies in stripping institutionalist theories down to their assumptions. The weakness is a lack of systematic evidence and dismissal of small-state roles.

\subsection*{Evaluate}
The article is foundational in the realism–institutionalism debate. Compared with Keohane, who sees institutions as enabling cooperation, Mearsheimer insists they merely mirror power distributions. This pessimism usefully cautions against overstating Irish influence via institutions. However, constructivists such as Wendt challenge his determinism, arguing that identities and norms shape interests. Thus, Mearsheimer is best read as a counterweight: his realist scepticism should temper Irish institutional optimism.

\subsection*{(Autho)R}
Mearsheimer, a University of Chicago political scientist, is the leading proponent of offensive realism. His positionality biases him toward viewing institutions as epiphenomena of great-power rivalry. For small states, this means he underplays legitimacy and agency. Nonetheless, his realist scepticism is valuable for exposing vulnerabilities in Irish reliance on multilateral institutions.

\subsection*{Limit → Implication}
Limit: Ignores small-state agency and treats institutions as powerless facades.  
Implication: For Ireland, the challenge is to prove him wrong by using institutions to amplify legitimacy and shape norms, even if they cannot alter material power distributions.

\subsection*{PEEL-C Application to Framework}
\textbf{Point:} Mearsheimer denies institutions shape outcomes, reducing their role to reflections of power.  
\textbf{Evidence:} He argues NATO’s effectiveness during the Cold War derived not from institutional rules but from the bipolar balance of power \parencite{MEARSHEIMER_1994}.  
\textbf{Explain:} This undermines small states’ hopes of leveraging institutions to multiply influence.  
\textbf{Limit:} His framework neglects legitimacy-building functions—precisely where small states can matter.  
\textbf{Consequent:} For Ireland, legitimacy must be anchored in political coherence, with institutions used as platforms for visibility rather than illusions of control.

\subsection*{Mapping into the Five Effects}

\begin{itemize}
	\item \textbf{Niche Specialisation:} Realism suggests niches only matter when aligned with great-power interests. Irish peacekeeping niches risk marginalisation unless seen as contributing to broader balances of power.
	\item \textbf{Organisational Agility:} Agility is constrained; in a realist world, reforms cannot offset structural weakness. For Ireland, agility matters for credibility, but not for changing outcomes.
	\item \textbf{Hybrid Leverage:} Institutions offer little shield from coercion; small states remain exposed. Ireland’s cyber and regulatory tools may have symbolic but not strategic weight.
	\item \textbf{Soft Power Synergy:} Mearsheimer rejects Nye’s optimism, seeing attraction as irrelevant beside power. For Ireland, this implies soft power only matters when tied to great-power calculations.
	\item \textbf{Legitimacy:} Legitimacy is downgraded in realism, yet for small states it remains the decisive anchor. The implication is that Irish legitimacy is valuable, but mainly as moral cover for great-power strategies rather than independent influence.
\end{itemize}

\subsection*{Synthesis}
Mearsheimer is sceptical of the very foundations on which small-state influence rests. His realism forces a hard question: can Ireland’s reliance on neutrality, peacekeeping, and institutions genuinely shape outcomes, or are they marginal rituals? The answer is that while Mearsheimer’s structural logic holds for great-power rivalry, small states retain agency in constructing legitimacy, niches, and regulatory influence. His work thus functions not as a model to emulate, but as a critical foil against which small-state frameworks must justify their relevance.


\section*{DIMER Analysis of Mearsheimer (2019)}

\subsection*{Describe}
\textcite{MEARSHEIMER_2019} argues that the liberal international order (LIO) was “bound to fail” because it contained the seeds of its own destruction. The LIO depended on unipolarity and U.S. hegemony, but nationalism, balance-of-power politics, and hyperglobalisation produced fatal contradictions. As multipolarity returned with the rise of China and Russia, the system shifted toward bounded realist orders. The article concludes that realist logic, not liberal ambition, will shape future world politics.

\subsection*{Interpret}
For small states, this analysis is a stark warning: liberal institutional platforms such as the EU and UN are unstable when not underwritten by U.S. power. Nationalism and multipolar rivalry constrain small-state strategies of legitimacy and institutional shelter. Yet Mearsheimer implicitly acknowledges that weaker states benefit from rule-based orders—even if created by great powers—as these can provide breathing space. For Ireland, the challenge is whether neutrality and peacekeeping maintain credibility when institutions fragment under great-power rivalry.

\subsection*{Methodology}
The article blends structural realist theory with historical analysis of the Cold War, post-Cold War order, and recent crises. It is conceptual and deductive, offering sweeping claims rather than empirical testing. Strength: it clarifies causal mechanisms of order decline. Weakness: it downplays small-state agency and the constructive power of norms.

\subsection*{Evaluate}
Compared with \textcite{KEOHANE_1988}, who sees institutions as enabling cooperation, Mearsheimer sees them as tools of great powers. Where \textcite{NYE_2008} stresses the persistence of soft power, Mearsheimer dismisses it as irrelevant when faced with nationalism and rivalry. Against \textcite{GRAY_2018}, who sees strategy as about political consequences, Mearsheimer focuses narrowly on structural drivers. The piece is useful as a sceptical foil to institutionalist and constructivist optimism in the small-state debate.

\subsection*{(Autho)R}
Mearsheimer, a University of Chicago scholar, is the leading proponent of offensive realism. His disciplinary and institutional context biases him toward systemic explanations and scepticism of liberal projects. This makes him a rigorous critic, but one who offers little for understanding the micro-level agency of small states.

\subsection*{Limit → Implication}
Limit: Treats small states and legitimacy as epiphenomenal, erasing their influence.  
Implication: For Ireland, the task is to show that legitimacy and institutional embedding matter even in multipolar realist orders, not for changing power distributions but for sustaining visibility and credibility.

\subsection*{Mapping into the Five Effects Framework}

\begin{itemize}
	\item \textbf{Niche Specialisation:} Niches are viable only when tolerated by great powers; otherwise, they are irrelevant. Irish peacekeeping works only insofar as it aligns with major-power interests.
	\item \textbf{Organisational Agility:} Agility cannot overcome structural constraints; reforms enhance credibility but cannot alter outcomes in a multipolar order.
	\item \textbf{Hybrid Leverage:} Hybrid tools are dismissed as secondary—small-state cyber or regulatory initiatives cannot offset systemic rivalry.
	\item \textbf{Soft Power Synergy:} Soft power is marginalised; nationalism trumps liberal attraction. For Ireland, this underlines the fragility of neutrality as soft power.
	\item \textbf{Legitimacy:} Mearsheimer downplays legitimacy as an independent force. Yet this highlights why, for small states, legitimacy is essential: it is the only durable currency when structural power is absent.
\end{itemize}

\subsection*{Synthesis}
Mearsheimer’s pessimism challenges the very basis of small-state influence. His structural realism implies that Ireland’s reliance on legitimacy, neutrality, and institutions cannot shape systemic outcomes. Yet, as a counterpoint to Keohane, Nye, and constructivists, his work is valuable: it forces a more cautious appraisal of Irish influence. The synthesis suggests legitimacy does not replace power, but in a fragmented multipolar order, it remains the only lever through which small states can achieve recognition and limited voice.

\section*{DIMER Analysis of Tonra (1999)}

\subsection*{Describe}
\textcite{TONRA_1999} examines the Europeanisation of Irish foreign policy, focusing on how EU membership and the Common Foreign and Security Policy (CFSP) have transformed the Department of Foreign Affairs (DFA). He distinguishes between two models: complex interdependence (institutions as bargaining arenas) and polity-forming (identity and norms shaping interests). Tonra argues that in Ireland’s case, the latter is more evident: EU participation has broadened the agenda, socialised diplomats into collective norms, and redefined neutrality debates. The key finding is that Irish foreign policy has been structurally and normatively reshaped by the EU.

\subsection*{Interpret}
The analysis applies directly to small states within the EU, showing how institutions and norms can magnify their voice but also constrain autonomy. For Ireland, EU integration enhances access, information, and influence but erodes traditional independence, particularly around neutrality. The lacuna is Tonra’s relative neglect of how great-power dynamics (e.g., U.S. hegemony, Russian resurgence) interact with Europeanisation. For Defence Forces strategy, the “so what” is that small states cannot be understood in isolation: their agency is mediated through institutional embedding and identity change.

\subsection*{Methodology}
Tonra uses qualitative case analysis, drawing on institutional theory, interviews with policy actors, and documentary evidence. This is a medium-level evidence source: stronger than opinion pieces but weaker than large-N or systematic empirical studies. The strength is depth of context and insider perspectives; the weakness is possible bias toward elite narratives and limited generalisability beyond Ireland.

\subsection*{Evaluate}
Compared with \textcite{MEARSHEIMER_1994}, who dismisses institutions, Tonra demonstrates their transformative power on small states. Against \textcite{GRAY_2005}, who stresses continuity, Tonra shows significant change in norms and structures. Relative to \textcite{KEOHANE_1988}, Tonra offers empirical backing for institutionalist optimism, but with caution: Europeanisation redefines identity but raises tensions with neutrality. His contribution lies in grounding abstract institutionalism in the lived experience of a small EU member.

\subsection*{(Autho)R}
Ben Tonra, a scholar of Irish and EU foreign policy at University College Dublin, writes from a constructivist-institutionalist perspective. His positionality biases him toward highlighting European norms and identity-shaping. Critics might argue this underestimates the persistence of realist constraints, but his institutional grounding ensures credibility. The article is widely cited as a foundational case study in Europeanisation research.

\subsection*{Limit → Implication}
Limit: Focuses narrowly on EU dynamics, neglecting systemic great-power constraints.  
Implication: For Ireland, while Europeanisation empowers diplomacy, legitimacy still depends on reconciling EU commitments with neutrality and global credibility.

\section*{Mapping into the Five Effects Framework}

\begin{itemize}
	\item \textbf{Niche Specialisation:} EU presidencies and CFSP participation provide Ireland with niches of influence (agenda-setting, mediation). Yet, niches are contingent on EU collective priorities, not independent design.
	\item \textbf{Organisational Agility:} DFA expanded, restructured, and professionalised due to EU demands. Agility is enabled through adaptation but constrained by interdepartmental overlaps and resource limitations.
	\item \textbf{Hybrid Leverage:} Europeanisation enhances Ireland’s access to regulatory and normative power (e.g., co-location of missions, EU-wide consular protection). Hybrid leverage is thus collective, not unilateral.
	\item \textbf{Soft Power Synergy:} EU membership amplifies Ireland’s neutrality and peacekeeping credentials, reframing them as European values. Yet, soft power risks dilution if perceived as mere EU conformity.
	\item \textbf{Legitimacy:} The EU provides legitimacy platforms, but neutrality debates reveal fragility. Legitimacy depends on balancing Europeanisation with domestic consensus on sovereignty and independence.
\end{itemize}

\subsection*{Synthesis}
Tonra shows that small states like Ireland are not passive: Europeanisation empowers them by reshaping institutions, identities, and diplomatic practices. However, empowerment comes at the cost of autonomy, with neutrality increasingly challenged by CFSP norms. For the five-effects framework, Tonra provides strong evidence for organisational agility, soft power synergy, and legitimacy as interdependent outcomes of institutional embedding. His work thus supports institutionalist optimism while highlighting the political costs of integration for small-state distinctiveness.

\section*{DIMER Analysis of Tonra and Christiansen (2011)}

\subsection*{Describe}
\textcite{TONRA_2011} explore how EU foreign policy sits uncomfortably between International Relations (IR) and European Studies. They frame the Common Foreign and Security Policy (CFSP) as a “problem child”: formally intergovernmental yet increasingly institutionalised, norm-driven, and identity-shaping. The article surveys rationalist accounts (neo-realist and neo-liberal), critiques their limits, and promotes cognitive/constructivist approaches that stress identity, beliefs, and norms as constitutive forces. Key finding: CFSP is more than lowest-common-denominator bargaining—it has transformative effects on member state diplomacy.

\subsection*{Interpret}
For small states like Ireland, this analysis matters because it highlights how institutional and normative embedding in the EU magnifies diplomatic capacity. Unlike Mearsheimer’s realism \parencite{MEARSHEIMER_1994}, Tonra argues that identities and norms evolve through participation. However, the paradox of CFSP persists: formal unanimity preserves national sovereignty, yet socialisation shifts perceptions of interest. The lacuna is limited attention to power asymmetries—smaller states may be socialised but still face dominance from France, Germany, and the UK. The “so what” for Irish Defence Forces is that influence flows less from coercion and more from norm entrepreneurship and credibility in EU fora.

\subsection*{Methodology}
The study is theoretical and meta-analytical. It critiques rationalist IR and European integration theories, offering constructivist lenses (cognitive approaches, identity construction). Evidence is conceptual rather than empirical, ranking as mid-level in the evidence hierarchy. Strength: clarifies conceptual debates and bridges IR and European Studies. Weakness: limited empirical case testing, over-reliance on theoretical synthesis.

\subsection*{Evaluate}
Compared with Keohane’s institutionalism \parencite{KEOHANE_1988}, which stresses rational bargaining, Tonra extends the analysis to norms and identities. Against Mearsheimer’s realist scepticism \parencite{MEARSHEIMER_2019}, he shows how institutions do more than reflect power—they shape actor identities. Relative to Nye’s soft power \parencite{NYE_2008}, Tonra offers institutional grounding: attraction is embedded in EU socialisation processes. His contribution lies in offering a constructivist framework to explain why small states persistently invest in EU foreign policy despite sovereignty costs.

\subsection*{(Autho)R}
Ben Tonra (University College Dublin) and Thomas Christiansen (LUISS, Rome) are leading European foreign policy scholars, with constructivist-institutionalist leanings. Their disciplinary context biases them toward highlighting EU norms and identity as constitutive, potentially underplaying structural power politics. Still, both are widely cited, making this a credible and influential source in EU foreign policy literature.

\subsection*{Limit → Implication}
Limit: Theoretical and constructivist emphasis underplays material constraints and power asymmetry.  
Implication: For Ireland, while CFSP offers opportunities for legitimacy and influence, structural limits remain; niche diplomacy must be carefully framed within EU norms.

\section*{Mapping into the Five Effects Framework}

\begin{itemize}
	\item \textbf{Niche Specialisation:} CFSP presidencies and working groups provide opportunities for Ireland to specialise in mediation and consensus-building. Niches are institutionalised rather than unilateral.
	\item \textbf{Organisational Agility:} Irish diplomacy adapts through EU institutional engagement, demonstrating agility by embedding national priorities into collective agendas. Agility is mediated by EU structures.
	\item \textbf{Hybrid Leverage:} Normative and regulatory instruments (humanitarian aid, sanctions, Petersberg Tasks) extend Irish leverage via EU coalitions. Hybrid capacity is collective, not national.
	\item \textbf{Soft Power Synergy:} Participation in CFSP amplifies Ireland’s soft power, reframing neutrality and peacekeeping as part of a wider European normative identity. This strengthens credibility abroad.
	\item \textbf{Legitimacy:} EU norms confer legitimacy on small states, but legitimacy becomes conditional on alignment with CFSP consensus. Ireland’s neutrality is challenged yet legitimised through EU embedding.
\end{itemize}

\subsection*{Synthesis}
Tonra and Christiansen position CFSP as a laboratory for constructivist theory: small states gain influence not by shifting balances of power but by co-creating norms and identities. For Ireland, this validates investment in EU diplomacy as a multiplier of legitimacy and credibility. However, structural limits remind us that legitimacy remains fragile: niches, agility, hybrid leverage, and soft power only matter if coherently tied to EU norms. The work thus supports institutionalist optimism but highlights sovereignty costs for small states seeking relevance.


GPT Make sure that you Use this qUOTE FROM \parencite{THORHALLSSON_2006} ``ambitious political leaders may form a notion of the size of their state and its capacity for international action based on distinctive national
history or myths''. Neorealists say that if you ensure that your objectives do not exceed your capabilities a tiny state can do extraordinarily well on the international stage. I.e., for any state, it's crucial that an accurate view of your position is taken. Thorhallsson's model to analyse a state's power based on six categories is worth noting. Hence, should Ireland or other small states be able to identify an edge where they could `steal the march', (such as through Estonia's niche specialisation in cyber warfare), they can increase their influence international security outcomes - becoming the `system-affecting' states which Keohane spoke of. Indeed, the idea of a small stat's influence being linked to its systemic role perception is supported by Keohane's 1969 paper \parencite{KEOHANE_1969}. This adds nuances to realists such as Waltz who dismiss small states' influence, deeming them to be passive actors \parencite{WALTZ_1979}. 

\section*{DIMER Analysis of Keohane (1969)}

\subsection*{Describe}
\textcite{KEOHANE_1969} reviews a series of works on alliances and small states, framing what he terms the “Lilliputians’ dilemmas.” He identifies the conspicuousness of small states in an era dominated by great-power disparities, examining how nonalignment, alliance choices, and institutional participation condition their survival and influence. Keohane introduces a typology of systemic roles: system-determining, system-influencing, system-affecting, and system-ineffectual states. His key claim is that small states, unable to act decisively alone, seek influence through institutions, alliances, and norms.

\subsection*{Interpret}
The article applies directly to small states navigating Cold War dynamics but remains relevant in today’s multipolar system. For Ireland, it highlights the tension between neutrality and alliance dependence, underscoring why institutional participation is a rational strategy. However, Keohane does not address hybrid threats or regulatory leverage, limiting the direct applicability to contemporary security domains. The “so what” for Defence Forces analysis is that systemic role perception, not just material capacity, shapes small-state behaviour. This widens the scope beyond realists like Waltz \parencite{WALTZ_1979}, who reduce small states to passive actors.

\subsection*{Methodology}
The article is conceptual and synthetic, reviewing key texts and deriving typologies. It does not employ empirical case studies, placing it low in the hierarchy of evidence. Its strength is in providing definitional clarity and a heuristic framework for comparative analysis. Its weakness is the lack of systematic testing and reliance on Cold War assumptions.

\subsection*{Evaluate}
Keohane’s contribution is foundational: by moving beyond vague definitions of “small states,” he provides a typology that remains widely used. Compared to Rothstein’s psychological-material definition, Keohane’s systemic role approach is more flexible. Against Mearsheimer’s dismissal of institutions \parencite{MEARSHEIMER_1994}, Keohane highlights how small states rationally invest in international organisations to shape norms. However, he underplays the agency of small states in regulatory and hybrid domains, which later scholars like Farrell and Newman \parencite{FARRELL_2019} highlight.

\subsection*{(Autho)R}
Robert Keohane, then at the Brookings Institution, was an emerging institutionalist voice. His positionality—critical of realist determinism, yet cautious of overstatement—foreshadows his later leadership in neoliberal institutionalism. His disciplinary stance biases him toward highlighting the utility of institutions and alliances as rational strategies for small states, though critics argue this overlooks structural constraints.

\subsection*{Limit → Implication}
Limit: Framework developed during the Cold War, with limited attention to hybrid or regulatory power.  
Implication: For Ireland, the typology still guides analysis but must be adapted to modern contexts such as cyber and EU regulation.

\section*{Effects Mapping}
\begin{itemize}
	\item \textbf{Niche Specialisation:} Suggests small states can develop niches within alliances and organisations, but only when systemic roles allow.
	\item \textbf{Organisational Agility:} Implies agility is less about military reform and more about adjusting systemic role perceptions and alliance strategies.
	\item \textbf{Hybrid Leverage:} Absent in Keohane’s 1969 framework; does not anticipate economic or digital interdependence as tools.
	\item \textbf{Soft Power Synergy:} Strongly aligned; stresses institutions and norms as avenues for small-state influence.
	\item \textbf{Legitimacy:} Central; systemic roles and institutional participation confer legitimacy on small states unable to influence outcomes unilaterally.
\end{itemize}

\section*{PEEL-C Paragraph}
Keohane demonstrates that small states cannot alter the system independently but may act as “system-affecting” powers through institutions and alliances \parencite{KEOHANE_1969}. For example, Ireland’s participation in the UN and EU allows it to convert systemic weakness into credibility, signalling legitimacy through collective norms. This underscores that small states gain soft power and diplomatic visibility when embedded in rules-based systems. However, Keohane’s Cold War framing does not account for contemporary hybrid threats, limiting applicability to 21st-century challenges. Consequently, while the systemic role typology clarifies Ireland’s position, it must be adapted to include regulatory and cyber dimensions for modern relevance.

\section*{DIMER and Effects Analysis of \textit{The Dynamics of Military Revolution, 1300--2050}}

\subsection*{Describe}
Knox and Murray distinguish between broad \textbf{military revolutions (MRs)} and narrower \textbf{revolutions in military affairs (RMAs)}. MRs are society-driven transformations (e.g., the 17th-century state/army revolution, the French Revolution, the Industrial Revolution, World War I, and nuclear weapons), while RMAs are narrower innovations combining technology, doctrine, and organisation (e.g., Blitzkrieg, precision strike). The editors warn against technological determinism, stressing that political context and culture shape outcomes \parencite{MURRAY_2001}.

\subsection*{Interpret}
The analysis applies to states with robust institutions and industrial bases; it is less transferable to small or resource-poor states who cannot independently generate RMAs. For Defence Forces, the ``so what'' is that strategic change comes from organisational and political adaptation, not just procurement. The lacuna is the under-emphasis on small-state adaptation pathways, such as niche roles or institutional embedding.

\subsection*{Methodology}
The work is a historical-conceptual edited volume. It synthesises cases from 1300 to projections of 2050, using qualitative historical analysis. On the hierarchy of evidence, it is expert opinion and case-study synthesis, not systematic data. Reliability is high in terms of historical description, but predictive claims about information revolutions are speculative. Validity is strong for illustrating patterns but weak for forecasting.

\subsection*{Evaluate}
Compared to Krepinevich’s optimism about technological RMAs \parencite{KREPINEVICH_1994}, Knox and Murray are more cautious, stressing that without doctrine and organisational adaptation, technology disappoints. Gray concurs, emphasising continuity and cultural limits \parencite{GRAY_2005}. Against Nye’s optimism about information power \parencite{NYE_2008}, the editors underline that informational revolutions are unpredictable and prone to hubris. The volume’s contribution is to clarify the distinction between MR and RMA, and to warn against conflating technology with strategy.

\subsection*{(Autho)R}
MacGregor Knox and Williamson Murray are leading historians and strategists, associated with sceptical readings of military innovation. Their disciplinary stance biases them toward continuity, institutional inertia, and the primacy of politics over technology. Their U.S. defence connections frame their analysis against over-optimism in Pentagon RMA discourse of the 1990s. The source is authentic, authoritative, but not neutral—offering a counterweight to technological determinists.

\subsection*{Limit → Implication}
Limit: The analysis underplays small-state adaptation and treats revolutions as great-power phenomena.  
Implication: For Ireland, lessons must be adapted—small states cannot generate MRs, but can exploit niches within RMAs if legitimacy and institutional agility are aligned.

\section*{Effects Mapping}
\begin{itemize}
	\item \textbf{Niche Specialisation:} Late adopters cannot leapfrog without state capacity; small states can exploit niches only if doctrines are coherent.
	\item \textbf{Organisational Agility:} RMAs demand cultural and organisational reform; agility determines whether technology yields advantage.
	\item \textbf{Hybrid Leverage:} Information/network revolutions may offer tools, but are double-edged; small states risk overestimating their leverage.
	\item \textbf{Soft Power Synergy:} Mass politics and ideology show that informational and normative domains amplify military effects; small states can harness this through peacekeeping or neutrality narratives.
	\item \textbf{Legitimacy:} Each MR demonstrates that political legitimacy anchors military change: nationalism, industrial mobilisation, deterrence. For small states, legitimacy is the decisive multiplier.
\end{itemize}

\section*{PEEL-C Paragraph}
\textbf{Point:} Knox and Murray argue that revolutions in military affairs fail without doctrinal and cultural adaptation \parencite{MURRAY_2001}.  
\textbf{Evidence:} The German Blitzkrieg of 1940 appeared revolutionary but was in fact the culmination of interwar doctrinal evolution.  
\textbf{Explain:} This demonstrates that organisational agility and cultural receptivity, not technology alone, generate strategic advantage.  
\textbf{Limit:} The framework underplays how small states might adapt niches within these revolutions.  
\textbf{Consequent:} For Ireland, the implication is clear—legitimacy and agility, not procurement alone, are the levers through which influence is gained in a revolutionary environment.


\subsection*{Summary}
\textbf{Key Papers} provide the theoretical and conceptual backbone across the five effects. \textbf{Supplementary Papers} anchor the argument in Irish policy and EU context, allowing you to connect theory to practice. Priority should be given to the first ten key sources (Thorhallsson, Keohane, Krepinevich, Gray, Farrell/Newman, Nye, Mearsheimer) before moving to contextual policy documents.

definition of small power ``A Small Power is a state which recognizes that it can not obtain security
primarily by use of its own capabilities, and that it must rely fundamentally
on the aid of other states, institutions, processes, or developments to do so;
the Small Power's belief in its inability to rely on its own means must also be
recognized by the other states involved in international politics'' from Rothstein. Vital says ``f the rough limits of the isolated small power's strength can be delineated and its characteristic disabilities outlined, something that is
typical of all small states will have been shown. For the unaligned state can best be regarded as a limiting case for the class of small states, one from which all other small states shade off, in varying and progressively lessening degrees
of political and military isolation. What can be said of the limiting case is likely to be applicable mutatis mutandis to the others. It is of the unaligned power as the paradigm for all small powers that the present study is conceived''.

Rothstein devotes over ioo pages to the question of what types of alliances
are most beneficial to small powers. Contending that (p. I70) "in theory,
Small Power alliances are condemned; in practice, they remain popular," his
most novel conclusion is his defense of these arrangements. He concludes
(p. I77):
Small Powers ought to prefer mixed, multilateral alliances. They provide
the most benefits in terms of security and political influence. If unavailable,
they probably should choose a Small Power alliance in preference to an unequal, bilateral alliance, particularly if the Small Powers do not fear an immediate threat to their security, and if their goals in allying are primarily political.
An alliance with a single Great Power ought to be chosen only if all the other
alternatives are proscribed, and if the Small Powers fear an imminent attack and even then only in hopes of improving their deterrent stance.

% ========================
% Gray and Waltz Analysis
% ========================

Gray's 2018 book on strategy \parencite{GRAY_2018} notes:

\begin{quote}
	``the primary task of subordinate generals must be to organize and command the timely and appropriate necessities for combat for the purposes established by the most senior level of military command in their negotiation with civilian political authority''.
\end{quote}

This aligns with \parencite{COHEN_2002}'s discussion on the special friction between the military and civilian actors. Gray further argues:

\begin{quote}
	``strategy is about the purposes of action while tactics are about actually performing the action in question''. 
\end{quote}

\begin{quote}
	``No matter what the weapon technologies will be in the future decades, we know for certain that nothing fundamentally important and positive in moral terms is going to change''.
\end{quote}

K.~N. Waltz, \textit{Theory of International Politics} (1979), writes:

\begin{quote}
	``Laws establish relations between variables, variables being concepts that can take different values. If a, then b, where a stands for one or more independent variables and b stands for the dependent variable: In form, this is the statement of a law. If the relation between a and b is invariant, the law would read like this: If a, then b with probability x. A law is based not simply on a relation that has been found, but on one that has been found repeatedly. By definition, theories are collections or sets of laws pertaining to a particular behaviour or phenomenon. [...] Theories are, then, more complex than laws, but only quantitatively so. Between laws and theories no differences of kind appears. A theory is born in conjecture and is viable if the conjecture is confirmed.''
\end{quote}

% ==========================
% DIMER Analysis of Waltz
% ==========================

\section*{DIMER Analysis of Waltz (1969)}

\subsection*{Describe}
\textcite{WALTZ_1969} outlines the systemic constraints of an anarchic international order, advancing the idea that states exist in a self-help system where survival is the overriding imperative. He compares state behaviour to oligopolistic firms: cooperation is possible, but always fragile and conditioned by fears of relative gains. Waltz stresses the inevitability of the security dilemma: defensive measures by one state appear threatening to others, driving cycles of competition and mistrust. His central claim is that the distribution of capabilities determines outcomes, not intentions or norms.

\subsection*{Interpret}
The argument applies most strongly to great-power competition, where balance-of-power politics dominates. For small states, Waltz implies they are system-takers: their autonomy is constrained by structural forces, and their survival depends on alignment, prudence, or shelter. This undercuts institutionalist optimism (e.g., Keohane \parencite{KEOHANE_1969}) and constructivist accounts (e.g., Tonra \parencite{TONRA_1999}) that highlight agency through norms. The “so what” for Ireland is sobering: neutrality, peacekeeping, or EU membership cannot alter systemic dynamics; at best, they provide temporary stability. However, Waltz does not account for legitimacy as an independent resource or for small states’ ability to exploit institutional niches.

\subsection*{Methodology}
Waltz employs a theoretical and analogical method, drawing on market oligopoly to model international politics. His analysis is deductive, not empirical, and sits at the level of grand theory. Its strength is parsimony and clarity; its weakness is lack of data and neglect of domestic, normative, or institutional variables. Reliability is strong within realist paradigms, but external applicability to small states is limited.

\subsection*{Evaluate}
Waltz’s neorealism is foundational, providing a rigorous systemic explanation of state behaviour. Compared with Mearsheimer’s offensive realism \parencite{MEARSHEIMER_1994}, Waltz is more structural, seeing states as constrained balancers rather than aggressive power-maximisers. In contrast to Keohane’s institutionalism \parencite{KEOHANE_1988}, Waltz dismisses the ability of institutions to alter systemic incentives. While invaluable as a baseline, his theory underdetermines small-state agency and legitimacy. It remains a critical foil for testing institutionalist and constructivist claims about Ireland and other small states.

\subsection*{(Autho)R}
Kenneth Waltz, one of the most influential realist theorists, wrote from a U.S. academic context during the Cold War. His disciplinary stance biases him toward structural determinism, downplaying domestic politics and normative change. His authority makes his theory widely cited, but his scepticism of small-state influence reflects his context rather than universal applicability.

\subsection*{Limit $\rightarrow$ Implication}
\textbf{Limit:} Neorealism overstates systemic constraints and neglects the role of institutions, norms, and legitimacy. \\
\textbf{Implication:} For Ireland, Waltz’s framework warns of constraints but must be adapted, recognising that small states can gain conditional influence through legitimacy and institutional niches.

% =====================
% Effects Mapping
% =====================

\section*{Effects Mapping}
\begin{itemize}
	\item \textbf{Niche Specialisation:} Waltz is sceptical; niches are irrelevant unless tolerated by great powers.
	\item \textbf{Organisational Agility:} Agility is structurally constrained; reforms cannot offset systemic pressures.
	\item \textbf{Hybrid Leverage:} Absent from Waltz’s account; interdependence is framed only as vulnerability, not as leverage.
	\item \textbf{Soft Power Synergy:} Dismissed; norms and attraction cannot substitute for power.
	\item \textbf{Legitimacy:} Neglected as an independent force; survival rests on power and balance, not legitimacy.
\end{itemize}

% =====================
% PEEL-C Paragraph
% =====================

\section*{PEEL-C Paragraph}
Waltz argues that international politics is governed by anarchy, forcing states to prioritise survival in a self-help system \parencite{WALTZ_1969}. For small states, this means limited scope for independent influence: niches, agility, or soft power are fragile under systemic constraints. This explains why Irish neutrality cannot guarantee security, as it depends on the tolerance of stronger states. However, Waltz neglects how institutions and legitimacy can create conditional space for influence, as later shown by Keohane and Thorhallsson. Consequently, Waltz serves as a critical realist baseline: his scepticism sharpens the need to demonstrate how small states defy structural limits through legitimacy and institutional embedding.

\parencite{WALTZ_1979} also speaks of the interdependence and integration of states. He allows for specialisation when vulnerability is reduced through interdependence, but stresses that in an anarchic international system such protection does not exist. Realpolitik therefore dominates. For small states, influence is easily diminished when caught between great powers. De Valera’s Ireland during the Second World War illustrates this logic: sovereignty was acutely vulnerable, given British fears that the Free State could be used as a base for attack. His realpolitik required acquiescing on the Treaty Ports and adopting neutrality. 

\textcite{FANNING_2015} shows that neutrality was a neorealist strategy for survival, not an immutable cultural identity. De Valera’s refusal to join the Allies, despite Churchill’s pressure, affirmed independence as the overriding priority: \textit{“no small nation adjoining a great power could ever hope to ... go its way in peace”}. Secrecy and ambiguity, as Fanning notes, enabled both blocs to interpret Ireland’s stance favourably. This aligns with Gray’s principle 14, that \textit{“intelligence and deception are permanent features of strategy”} \parencite{GRAY_2018}.

Gray further stresses:

\begin{quote}
	``Two factors in particular serve to limit a state's freedom of strategic choice. The first is geography — location and terrain shape options. The second is the existence, or presumption, of an opponent: strategy presumes enemies.'' \parencite{GRAY_2018}
\end{quote}

De Valera embodied this, concealing intentions, managing ambiguity, and understanding geography and great-power proximity. In this sense, Irish neutrality was structural realism concealed beneath performative liberalism.

% =====================
% Thorhallsson, Keohane, Rothstein
% =====================

\parencite{THORHALLSSON_2006} stresses the importance of both internal and external perceptions of competence and vulnerability — legitimacy as a resource. Similarly, \textcite{KEOHANE_1969} and Rothstein (1966) highlight that small powers rely on alliances, but warn against unequal partnerships with great powers, which risk subordination. Fanning (2013) shows that Britain even offered reunification in return for Irish participation in WWII, but De Valera prioritised sovereignty. Realism dictated caution: unequal alliances undercut small-state leverage.

% =====================
% Carroll 2023
% =====================

\section*{\parencite{CARROLL_2023}}

\textcite{CARROLL_2023} argues that like many small nations facing conventional and hybrid threats, Ireland confronts a disconnect between ends, ways, and means. Cleary highlights that the Defence Forces have steadily diminished while commitments expanded. Dan Ayiotis notes that “the stronger the military, the more neutral a state can afford to be,” tracing neutrality’s phases as necessity (1923–39), expediency (1939–55), and convenience (1955–present). This echoes De Valera’s pragmatism: neutrality is always realpolitik/neorealism, adapted to circumstance. 

% =====================
% Notes on Realism
% =====================

\section*{Notes on Realism and Liberalism}
\begin{itemize}
	\item Realism centres on power and geography; security is paramount, unemotional, and the end justifies the means.
	\item Neorealism stresses an anarchic state system: survival, self-help, balance-of-power. 
	\item Classical realism emphasises human nature (Morgenthau), neoclassical realism blends systemic and domestic factors.
	\item Neutrality, for Ireland, was always a realist instrument — a means of survival under anarchy.
	\item Liberalism offers a contrasting paradigm: Kant, Mill, Rousseau, and Smith stress cooperation, trade, and institutions. The EU epitomises liberal institutionalism, while realists remain sceptical.
\end{itemize}

  
  
\section*{DIMERS Analyses of Key Sources}

\subsection*{Andrew Cottey (2022) – A Celtic Zeitenwende? Continuity and Change in Irish National Security Policy}

\textbf{D – Describe}  
Cottey (2022) examines whether the Ukraine war has triggered fundamental change in Irish security policy. He argues that Ireland’s policy is defined by four enduring traits: very low threat perception, free-riding with minimal defence spending, political resistance to militarisation, and cautious EU cooperation. He concludes that the Ukraine war has not shifted these patterns — continuity prevails \parencite{COTTEY_2022}.

\textbf{I – Interpret}  
This applies directly to small European states with traditions of neutrality and under-investment in defence, but not to heavily militarised small states like Israel. For Ireland, it means legitimacy through neutrality and peacekeeping is unlikely to be replaced by robust hard power. Limit: focuses narrowly on continuity, less on potential shocks. Implication: supports the realist critique that small-state agility is overstated.

\textbf{M – Methodology}  
The paper is conceptual and policy-analytical rather than empirical. It uses case-based reasoning drawn from contemporary events. Limit: absence of data weakens causal claims. Implication: adopt as a framing source, but supplement with empirical studies.

\textbf{E – Evaluate}  
Cottey’s emphasis on continuity challenges more optimistic claims about rapid adaptation in small states. Other scholars (Thorhallsson, Bailes) highlight niches and agility, which Cottey implicitly downplays. Limit: underplays the role of legitimacy. Implication: useful for the “against” section of your essay.

\textbf{R – (Autho)R}  
Cottey is a senior academic at UCC with strong credibility in European security. His background suggests a bias towards cautious, empirically grounded arguments rather than optimistic institutionalist claims. Implication: strengthens the critical balance in your essay.

\textbf{S – Synthesis}  
The article directly challenges the “organisational agility” effect in your framework, reinforcing realist scepticism. It complements Flynn (2019) by providing a counterpoint. Limit: generalises from one case. Implication: Ireland’s legitimacy may be more vulnerable than Flynn’s optimism suggests.

\textbf{PEEL Paragraph}  
\textit{Point:} Cottey (2022) contends that despite Ukraine, Ireland’s security policy remains rooted in continuity.  
\textit{Evidence:} He identifies persistent features — underfunding, political resistance, and reliance on EU “good citizenship” \parencite{COTTEY_2022}.  
\textit{Explain:} This undercuts claims that small states can rapidly adapt, showing that political culture and structural constraints dominate.  
\textit{Link:} The essay can therefore use Cottey to reinforce the realist limit on small-state agility.  

---

\subsection*{John F. Quinn (2018) – Dreaming of things that never were: Irish Soft Power and Peacekeeping in the 21st Century}

\textbf{D – Describe}  
Quinn (2018) explores how peacekeeping has become a central pillar of Irish soft power. He frames peacekeeping as both identity and foreign policy tool, enabling Ireland to sustain legitimacy disproportionate to its material power \parencite{QUINN_2018}.

\textbf{I – Interpret}  
This applies to small states that use peacekeeping as a reputational strategy (Ireland, Ghana, Nordic states). It does not apply where peacekeeping is minimal or contested. Limit: assumes peacekeeping’s prestige remains intact. Implication: Ireland’s reliance on peacekeeping may expose vulnerability if missions lose legitimacy.

\textbf{M – Methodology}  
A conceptual-interpretive essay linking soft power theory with Irish practice. Not empirical, but strongly aligned with constructivist approaches. Limit: lacks quantitative evidence. Implication: best used as normative framing.

\textbf{E – Evaluate}  
Quinn’s claims align with Nye’s theory of soft power, and with Thorhallsson’s identity-based model of small states. However, critics (e.g., Minihan 2018) question whether peacekeeping retains relevance. Limit: possibly romanticised view. Implication: valuable but requires counterpoint.

\textbf{R – (Autho)R}  
Quinn is a Defence Forces officer-scholar, with institutional alignment towards highlighting Ireland’s constructive global role. This may shape a normative preference for peacekeeping as inherently good. Implication: use with awareness of bias.

\textbf{S – Synthesis}  
The article supports your “soft power synergy” effect directly, providing evidence for how legitimacy is cultivated through peacekeeping. Limit: assumes future continuity. Implication: useful as evidence for the “for” side of your framework.

\textbf{PEEL Paragraph}  
\textit{Point:} Quinn (2018) argues that peacekeeping underpins Ireland’s soft power identity.  
\textit{Evidence:} He frames it as the primary means through which Ireland gains legitimacy in international security \parencite{QUINN_2018}.  
\textit{Explain:} This supports the constructivist claim that myths and narratives matter as much as material capacity.  
\textit{Link:} In the essay, Quinn can illustrate how legitimacy functions as Ireland’s centre of gravity.  

---

\subsection*{Jonathan Carroll \& Neil Richardson (2022) – After the War Ends: Ireland, the Reserve Defence Forces, Peacekeeping, and the Russo-Ukraine War}

\textbf{D – Describe}  
Carroll and Richardson (2022) assess the role of the Reserve Defence Forces in sustaining Ireland’s peacekeeping commitments in the aftermath of the Russo-Ukraine war. They argue that Ireland faces structural challenges in maintaining its traditional contributions \parencite{CARROLL_RICHARDSON_2022}.

\textbf{I – Interpret}  
Applies to small states that link legitimacy to continuity in peacekeeping commitments. Does not apply to those disengaged from UN operations. Limit: focuses on Irish reserves, not broader global context. Implication: demonstrates fragility of Ireland’s ability to sustain legitimacy.

\textbf{M – Methodology}  
A policy-focused commentary, informed by institutional practice. Not data-heavy, but grounded in Defence Forces context. Limit: narrow scope. Implication: valuable for Irish-specific analysis.

\textbf{E – Evaluate}  
This complements Quinn (2018) by showing that peacekeeping legitimacy faces practical sustainability issues. It diverges from Flynn (2019), who emphasises capability enhancement. Limit: focuses on supply-side issues. Implication: adds nuance to the essay.

\textbf{R – (Autho)R}  
Both authors are connected with Irish military/academic circles; this lends insider insight but may bring institutional bias. Implication: use as grounded but context-specific evidence.

\textbf{S – Synthesis}  
The piece shows how legitimacy grounded in peacekeeping can be undermined by organisational strain. Limit: context-specific. Implication: supports the essay’s theme of fragility in small-state influence.

\textbf{PEEL Paragraph}  
\textit{Point:} Carroll and Richardson (2022) argue that Ireland’s peacekeeping commitments may be undermined by reserve force fragility.  
\textit{Evidence:} They stress that sustaining contributions after Ukraine requires structural reform \parencite{CARROLL_RICHARDSON_2022}.  
\textit{Explain:} This highlights that legitimacy via peacekeeping is not automatic but contingent on organisational capacity.  
\textit{Link:} Their work provides evidence for the essay’s claim that small-state influence is fragile when domestic resources falter.  

---
\section*{DIMERS Analysis of COTTEY (2022)}

\subsection*{Describe}
COTTEY (2022) \nocite{COTTEY_2022} examines whether the Ukraine war marks a turning point in Irish security policy. He identifies four enduring features of Ireland’s strategic posture: (1) a very low-threat environment, (2) systematic free-riding through small forces, minimal combat capacity, and low spending, (3) a domestic political culture hostile to arguments for robust defence, and (4) cautious engagement in EU security and defence, framed as good citizenship but never transformative. His central claim is that, despite the shock of Ukraine, these features have not changed. Ireland continues to prioritise political legitimacy, low exposure, and selective cooperation rather than hard balancing:contentReference[oaicite:2]{index=2}:contentReference[oaicite:3]{index=3}.  
\textbf{Limit:} This description reflects Ireland’s historical continuity but does not explain deeper structural drivers.  
\textbf{Implication:} The persistence of these features shows a neo-realist logic at play: Ireland adapts minimally but avoids altering its structural dependence.

\subsection*{Interpret}
The article applies most directly to small states like Ireland, but also to neutral states such as Austria or pre-2022 Finland. It does not apply to revisionist or militarised small powers (e.g. Israel, North Korea) that actively shape security outcomes. COTTEY’s reading implies that Ireland’s posture is not one of idealist neutrality but of pragmatic hedging: remaining formally outside NATO while embedding in EU structures. This aligns with the historical evidence that, since de Valera’s realist manoeuvres in WWII, Ireland has been tacitly dependent on British security guarantees. As \parencite{FANNING_2015,AYIOTIS_2023} elucidate, Ireland’s neutrality masked a covert alignment with the UK as the proximate great power.  
\textbf{Limit:} COTTEY does not interrogate the covert alliance dynamic.  
\textbf{Implication:} Ireland’s hedging is understated in his framework but can be substantiated through historical and archival analysis.

\subsection*{Methodology}
The paper is conceptual and policy-analytical rather than empirical. It draws on defence-spending data, historical practice, and EU policy frameworks. It is best understood as expert commentary (lower on the evidence hierarchy), but situated within a broader Defence Forces Review tradition. While academically peer-reviewed, its limitations include reliance on secondary evidence and absence of counterfactuals (e.g., what a non-hedging Ireland might look like).  
\textbf{Limit:} Methodologically it cannot demonstrate causality between external shocks and Irish strategic choices.  
\textbf{Implication:} Findings should be read as indicative of structural trends, not as predictive models.

\subsection*{Evaluate}
COTTEY’s contribution is valuable in highlighting continuity despite systemic shocks. Compared to realist perspectives (Waltz, Mearsheimer), his work complements the claim that small states adapt within constraints but cannot reshape structures. Where scholars such as TONRA emphasise soft power and EU legitimacy, COTTEY undercuts this optimism by showing Ireland’s strategic inertia. However, he does not fully situate this in neorealist theory, missing the chance to frame Ireland as a hedging state that seeks cover under EU and UK security umbrellas.  
\textbf{Limit:} The evaluation undervalues theoretical clarity.  
\textbf{Implication:} To be most useful, his analysis should be read through a neo-realist lens of hedging.

\subsection*{(Autho)R}
COTTEY is a long-standing authority on European security and neutrality. His background at UCC and within NATO-linked research networks suggests institutional bias towards situating Ireland within broader European frameworks. He hedges in his own argument: acknowledging structural change (Ukraine) but ultimately concluding continuity. This reflects an academic tendency to emphasise caution in Irish policy rather than overtly realist categorisation.  
\textbf{Limit:} Authorial caution may understate the strategic dependence on the UK.  
\textbf{Implication:} His perspective should be balanced with FANNING and AYIOTIS, who explicitly link Ireland’s neutrality to tacit alliance with Britain.

\subsection*{Synthesis}
COTTEY’s account resonates with a broader body of realist scholarship: Ireland continues to hedge by appearing neutral while aligning informally with great powers. The article demonstrates that material weakness and political culture entrench continuity. When read alongside FANNING (2015) and AYIOTIS (2023), the synthesis shows that Ireland’s security strategy is less about idealist neutrality and more about neo-realist hedging: leveraging legitimacy for cover while free-riding on British and European security frameworks.  
\textbf{Limit:} Synthesis requires adding external sources not fully integrated by COTTEY.  
\textbf{Implication:} Ireland’s case underscores how small states survive through hedging—remaining formally neutral but tacitly dependent.

\section*{PEEL Paragraph}
Point: Ireland is best understood as a neo-realist state engaged in hedging.  
Evidence: COTTEY (2022) shows continuity in low defence spending, limited forces, and cautious EU engagement, while \parencite{FANNING_2015,AYIOTIS_2023} trace a tacit alignment with Britain since WWII.  
Explain: This demonstrates that Irish neutrality masks a pragmatic strategy of dependence, leveraging legitimacy abroad while avoiding costly commitments.  
Limit: COTTEY does not fully theorise hedging, leaving the concept implicit.  
Consequence: Framed through neorealism, Ireland’s “neutrality” is revealed as a form of security hedging, balancing autonomy with tacit reliance on the UK and EU.

\nocite{COTTEY_2022}Ireland’s security posture is best interpreted through a neo-realist lens of hedging. COTTEY (2022) underscores that despite the systemic shock of the Ukraine war, Irish national security continues to rest on very low defence spending, limited combat capability, and only cautious EU engagement. This continuity reflects not idealist neutrality but a pragmatic calculation: small states cannot afford unilateralism and therefore hedge by balancing autonomy with tacit alignment to stronger powers. \parencite{FANNING_2015,AYIOTIS_2023} demonstrate that this logic is longstanding—since de Valera’s realist manoeuvres in WWII, Ireland has relied upon covert security dependence on Britain, cloaked in the language of neutrality. While COTTEY does not explicitly frame this as hedging, the evidence of structural free-riding and reliance on great power guarantees illustrates precisely that dynamic. The limit is that neutrality remains politically potent at home, constraining acknowledgement of dependence; the implication is that Irish policy is best understood as a form of security hedging, combining symbolic neutrality with material reliance on others.

\section*{DIMERS Analysis of FLEMING (2015)}

\subsection*{Describe}
FLEMING (2015) examines Irish foreign policy during the Free State period (1922–1932), focusing on how neutrality was articulated before de Valera’s leadership. He argues that neutrality served as a convenient “all purpose” policy: symbolically asserting sovereignty, dampening domestic divisions, and signalling independence internationally, while in reality masking Ireland’s material dependence on Britain for security and trade. Fleming shows that Cumann na nGaedheal governments relied on British military guarantees while projecting an image of autonomous neutrality.  
\textbf{Limit:} Descriptive focus is historical rather than theoretical.  
\textbf{Implication:} Neutrality was a façade for hedging, not an idealist principle.

\subsection*{Interpret}
The article applies to small states constrained by a dominant neighbour. Fleming demonstrates that Ireland’s proclaimed neutrality was less a rejection of alliances and more a strategic hedge: maximising political autonomy while avoiding costly defence spending and confrontation with Britain. This resonates with broader small-state behaviour in asymmetric relationships. His account therefore reinforces the neo-realist argument that small states cannot escape structural dependence.  
\textbf{Limit:} He does not explicitly use the language of hedging or neorealism.  
\textbf{Implication:} Readers must supply the theoretical framing to see Ireland’s strategy as neo-realist.

\subsection*{Methodology}
Fleming draws primarily on historical sources—government documents, parliamentary debates, and contemporaneous commentary—set against the political context of the Free State. His methodology is qualitative and narrative, grounded in archival and secondary sources. The work is interpretive history, not statistical analysis.  
\textbf{Limit:} Absence of theoretical modelling.  
\textbf{Implication:} It is most useful for contextual grounding rather than abstract theorisation.

\subsection*{Evaluate}
Fleming’s article is valuable because it reveals the pragmatic and tactical use of neutrality in the Free State era. Compared to later analyses (e.g. COTTEY 2022), Fleming shows the origins of Ireland’s hedging posture—neutrality as performance, dependence as reality. This complements realist interpretations (Waltz, Mearsheimer) by offering a historical case study of how small states survive through posturing. However, Fleming stops short of explicitly theorising this as hedging or embedding it in IR theory.  
\textbf{Limit:} Lack of explicit theoretical engagement.  
\textbf{Implication:} Its value lies in providing historical depth to current realist interpretations.

\subsection*{(Autho)R}
Fleming was a Cadet (later officer) writing in the Defence Forces Review, meaning his perspective combines practitioner insight with historical analysis. His institutional location suggests a concern with the practical implications of neutrality and defence policy. This adds authenticity but also potential bias towards interpreting neutrality as pragmatic rather than ideological.  
\textbf{Limit:} Military-practitioner lens may underplay political-cultural factors.  
\textbf{Implication:} Must be read as a realist-leaning interpretation shaped by Defence Forces perspectives.

\subsection*{Synthesis}
Fleming shows that even before de Valera, Ireland’s neutrality was less about idealism and more about hedging against Britain. When synthesised with \parencite{FANNING_2015,AYIOTIS_2023}, the pattern becomes clear: Irish foreign policy has consistently involved realist calculation masked by symbolic neutrality. Together with COTTEY (2022), Fleming provides historical evidence that Ireland’s posture is best described as small-state hedging—projecting autonomy while relying on the great power next door.


\nocite{FLEMING_2015} FLEMING (2015) demonstrates that Irish neutrality in the 1920s was not an expression of principled isolationism but rather a pragmatic hedge against dependence on Britain. He shows that Cumann na nGaedheal invoked neutrality as an “all purpose policy” to mask the Free State’s reliance on British security guarantees, maintain low military expenditure, and project sovereignty abroad while quietly organising its army along British lines. This strategy illustrates neo-realist logic: faced with structural weakness and treaty restrictions, the Free State adopted hedging—symbolically neutral yet materially reliant. \parencite{FANNING_2015,AYIOTIS_2023} underscore that since de Valera’s realist interventions during the War of Independence, Ireland tacitly accepted a security dependence on Britain, which neutrality conveniently concealed. The limit of Fleming’s analysis is that he frames this policy as posturing rather than explicitly theorising it as hedging. The implication, however, is clear: even in its formative years, Ireland behaved as a neo-realist small state, balancing sovereignty claims with covert alignment to the proximate great power.

\parencite[pp. 125--128]{TONRA_2006} describes the 1958 Fianna Fail government's  major economic policy shift towards economic liberalisation. This is of note in the context of its continuation as the bedrock of Irish economic success. Indeed, the existence of large multinational corporations headquartered and manufacturing in Ireland is case and point. This shows an alternative aspect of Ireland's political strategy. De Valera was a realist and focused on independence, eschewing official alliances with hte UK while entering into a tacit alliance. Following from Ireland's 1954 entry to the UN, a mechanism arose where Ireland could leverage its military externally to gain international influence. It is of note that much of the evidence (such as \parencite{AYIOTIS_2023}) indicate that the military was not trusted for responsibility, authority and action within the state. Hence perhaps the UN deployment of the military was a political win-win.

\section*{DIMERS Analysis – Tonra (2006)}

\textbf{D – Describe}  
\parencite{TONRA_2006} identifies four competing narratives of Irish security and defence policy: the Irish Nation, Global Citizen, European Republic, and Anglo-American State. Each frames neutrality differently: as a violated principle, endangered treasure, modest contribution, or moral obligation. He argues that while Ireland’s UN-based contributions have been broadly honoured, divisions emerge over which institutional framework (UN, EU, NATO) legitimises Irish action.  

\textbf{I – Interpret}  
This is directly relevant to small-state legitimacy debates. The Irish Nation and Global Citizen narratives valorise the UN, but only in an idealised form; the Anglo-American perspective dismisses the UN as irrelevant. For Ireland, the choice of framework (UN vs EU vs NATO) determines the legitimacy of its contributions. Limit: written in 2006, it predates the more recent decline of UN peacekeeping. Implication: provides a baseline for understanding the erosion visible today.  

\textbf{M – Methodology}  
Tonra employs discourse analysis, mapping how competing identity narratives shape foreign policy. It is interpretive and qualitative, not empirical. Limit: abstract and narrative-driven. Implication: valuable for theoretical framing, less so for hard data.  

\textbf{E – Evaluate}  
Tonra’s framing aligns with constructivist IR theory (identity, discourse) but also anticipates realist concerns: the “actually existing UN” is compromised by P5 vetoes, echoing later findings of \parencite{DUMAN_RAKIPOGLU_2025}. Limit: does not anticipate Ireland’s shrinking peacekeeping footprint post-2010s. Implication: useful for linking historical continuity to current decline.  

\textbf{R – (Autho)R}  
Ben Tonra is a leading Irish IR scholar (UCD), with credibility in EU and foreign policy analysis. His perspective is academic rather than institutional, giving balance against Defence Forces insiders like \parencite{QUINN_2018} or \parencite{KING_2021}.  

\textbf{S – Synthesis}  
Tonra provides the theoretical foundation: Ireland’s legitimacy depends on how neutrality is discursively framed within UN/EU/NATO contexts. Combined with Flynn’s operational benchmarks, Quinn’s soft-power romanticism, Cole’s pragmatism, King’s resourcing critique, and Duman’s evidence of UN paralysis, Tonra (2006) anchors your essay in the identity dimension of legitimacy.  

\textbf{PEEL Paragraph}  
\textit{Point:} Tonra (2006) identifies four identity narratives that shape Ireland’s approach to neutrality and security.  
\textit{Evidence:} He shows that the Irish Nation and Global Citizen frame legitimacy in terms of UN primacy, while the Anglo-American dismisses the UN as irrelevant \parencite{TONRA_2006}.  
\textit{Explain:} This highlights how legitimacy is discursively contested, and why Ireland’s choice of institutional framework (UN vs EU vs NATO) is politically fraught.  
\textit{Link:} In the essay, Tonra’s work situates Ireland’s foreign policy as structurally realist in practice, but cloaked in performative liberalism.  

\section*{DIMERS Analysis of MINIHAN (2018)}

\subsection*{Describe}
\nocite{MINIHAN_2018} MINIHAN (2018) asks whether United Nations peacekeeping, in its current form, has a viable future. He highlights structural challenges: the UN Security Council’s paralysis, declining Western troop contributions, reliance on Global South contingents, and mission overstretch. He concludes that UN peacekeeping faces existential questions, as it is increasingly undermined by great-power rivalry, resource constraints, and mandate inflation. For Ireland, traditionally a strong contributor, these developments diminish the strategic value of peacekeeping as a foreign policy tool.  
\textbf{Limit:} The piece focuses on structural fragility but does not map alternatives in detail.  
\textbf{Implication:} Small states like Ireland cannot rely on UN peacekeeping as their primary niche.

\subsection*{Interpret}
The analysis applies directly to small states such as Ireland, whose international identity has been built around peacekeeping. MINIHAN shows that if UN peacekeeping declines, Ireland’s strategy of legitimacy through neutrality and UN service is weakened. This is consistent with a neo-realist reading: small states lose tools of influence when institutions fail. It does not apply to great powers, who can act unilaterally or through ad hoc coalitions.  
\textbf{Limit:} The argument does not fully explore regional or EU alternatives.  
\textbf{Implication:} Ireland must hedge by seeking legitimacy elsewhere (e.g. EU, UK partnership).

\subsection*{Methodology}
The article is conceptual and policy-oriented. It synthesises UN reports, historical case studies, and contemporary mission analysis. Methodologically it sits at the “expert commentary” level rather than empirical research. Its strength lies in practitioner insight; its weakness is lack of systematic data analysis or comparative case method.  
\textbf{Limit:} No quantitative evidence to measure effectiveness or decline.  
\textbf{Implication:} Conclusions are persuasive but not empirically conclusive.

\subsection*{Evaluate}
MINIHAN’s pessimism aligns with other realist critiques of peacekeeping effectiveness. Compared with QUINN (2018), who frames Irish peacekeeping as identity-driven, MINIHAN is more structural, arguing that institutional decline undermines that identity. His argument adds to the literature by situating Ireland’s predicament within UN systemic fragility, not just domestic politics.  
\textbf{Limit:} He does not address soft-power or constructivist accounts that stress reputational gains despite institutional weakness.  
\textbf{Implication:} Peacekeeping’s fragility reinforces realist claims that small states must hedge.

\subsection*{(Autho)R}
MINIHAN is a former Irish Army officer, writing within the Defence Forces Review context. His institutional lens privileges operational realities and the perspective of contributors. This background biases him towards highlighting operational and political dysfunction rather than constructivist or legitimacy-based optimism.  
\textbf{Limit:} Perspective may underplay non-material dimensions such as reputation, diplomacy, or symbolic legitimacy.  
\textbf{Implication:} His neo-realist framing is credible but must be balanced with other schools.

\subsection*{Synthesis}
MINIHAN’s article reinforces the view that Ireland’s peacekeeping niche is structurally endangered. Synthesised with Cottey (2022) \nocite{COTTEY_2022} and FLEMING (2015) \nocite{FLEMING_2015}, it shows continuity in Ireland’s reliance on hedging: neutrality and peacekeeping project autonomy but mask dependence on external powers and institutions. As peacekeeping falters, Ireland’s hedging must adapt, shifting towards EU and bilateral alignments.  
\textbf{Limit:} Synthesis highlights peacekeeping decline but cannot forecast the shape of future niches.  
\textbf{Implication:} Ireland’s legitimacy strategy is fragile, demanding diversification beyond the UN.

\nocite{MINIHAN_2018}MINIHAN (2018) argues that United Nations peacekeeping, long the cornerstone of Ireland’s foreign policy identity, faces existential decline. He highlights that great-power paralysis within the UN Security Council, coupled with overstretched missions and waning Western troop contributions, has left peacekeeping increasingly ineffective. For small states like Ireland, this erosion undermines neutrality’s utility as a source of legitimacy and international influence. While QUINN (2018) frames peacekeeping as central to Irish identity, MINIHAN shows that its structural viability is collapsing. The limit of his analysis is that it downplays reputational or symbolic gains Ireland might still accrue from limited contributions. The implication, however, is stark: if peacekeeping no longer functions as a credible niche, Ireland must hedge by shifting its strategy towards EU and bilateral frameworks, rather than relying solely on the UN.

\subsection*{Damien Cole (2018) – Pragmatic Evolution? Reflections on the foreign policy motivations, implications and impact of Ireland’s experience of peacekeeping in the Middle East}

\textbf{D – Describe}  
Cole (2018) analyses Ireland’s peacekeeping experience in the Middle East, reflecting on motivations and implications. He frames it as a pragmatic evolution — not idealism, but a calculated strategy to balance neutrality, legitimacy, and international expectations \parencite{COLE_2018}.

\textbf{I – Interpret}  
Applies to states that use peacekeeping as both identity and pragmatic strategy. Shows how legitimacy masks realist calculation. Limit: specific to Middle East cases. Implication: supports nuanced reading of Irish neutrality.

\textbf{M – Methodology}  
Historical-interpretive, drawing on case analysis of Middle Eastern missions. Not quantitative. Limit: bounded to regional cases. Implication: valuable for showing tension between rhetoric and practice.

\textbf{E – Evaluate}  
Cole bridges realist and constructivist perspectives: legitimacy is cultivated, but underpinned by survival strategies. This aligns with your argument that legitimacy is both genuine and instrumental. Limit: case-specific. Implication: strengthens essay’s theoretical balance.

\textbf{R – (Autho)R}  
Cole is an academic with expertise in Irish foreign policy. His framing suggests caution against idealist narratives. Implication: credible support for realist-informed analysis.

\textbf{S – Synthesis}  
This source is especially useful for showing how neutrality and legitimacy can mask pragmatic hedging. Limit: scope limited to Middle East. Implication: use as a bridge between Quinn’s constructivist optimism and Cottey’s realist scepticism.

\textbf{PEEL Paragraph}  
\textit{Point:} Cole (2018) contends that Ireland’s peacekeeping in the Middle East reflects pragmatic evolution.  
\textit{Evidence:} He demonstrates how neutrality and legitimacy were balanced against realist constraints \parencite{COLE_2018}.  
\textit{Explain:} This illustrates that peacekeeping was not purely idealist but also a survival strategy.  
\textit{Link:} In the essay, Cole helps position legitimacy as both a normative and pragmatic asset for small states.  

---

\subsection*{Comdt Conor King (2021) – Resourcing the State’s National Insurance Policy: The Case for Defence}

\textbf{D – Describe}  
King (2021) frames the Defence Forces as Ireland’s “national insurance policy,” arguing that underfunding and poor retention undermine resilience. He contends that legitimacy requires credible resourcing \parencite{KING_2021}.

\textbf{I – Interpret}  
Applies to small states where legitimacy is tied to reliability and credibility. Ireland’s underfunding risks reputational damage. Limit: domestic focus. Implication: demonstrates internal-external linkage of legitimacy.

\textbf{M – Methodology}  
Policy-analytical, based on institutional data and comparisons with other neutral small states. Limit: not theoretical. Implication: strong applied evidence.

\textbf{E – Evaluate}  
King’s argument aligns with realist concerns about capability gaps. It complements Cottey (2022), showing continuity in underfunding, and challenges Quinn (2018), which assumes legitimacy without resourcing. Limit: institutionally embedded. Implication: highlights fragility of Ireland’s “insurance policy.”

\textbf{R – (Autho)R}  
As a Defence Forces officer, King speaks with insider authority. Potential bias: institutional advocacy for resourcing. Implication: useful but must be balanced with external academic critique.

\textbf{S – Synthesis}  
King ties legitimacy directly to resources, bridging realist and constructivist positions. Limit: case-specific. Implication: vital for your conclusion that legitimacy is fragile without material backing.

\textbf{PEEL Paragraph}  
\textit{Point:} King (2021) argues that the Defence Forces act as Ireland’s national insurance policy, but are undermined by underfunding.  
\textit{Evidence:} He highlights retention crises and capability shortfalls as threats to legitimacy \parencite{KING_2021}.  
\textit{Explain:} This supports the view that legitimacy requires credible resourcing, not just narratives.  
\textit{Link:} In the essay, King provides evidence for the conclusion that legitimacy is fragile without material support.  
  
\section*{DIMERS Analysis: Flynn (2019)}

\textbf{D – Describe}  
Brendan Flynn’s (2019) article, \textit{Small States’ Capability Enhancement for Peacekeeping: What can Ireland learn from other countries?}, examines how small states leverage peacekeeping as a means of influence and legitimacy. The focus is comparative: Ireland’s experience is placed alongside other small states (notably in Scandinavia and Africa) that have sought to maximise visibility through peace operations. The central argument is that while small states cannot project power independently, they can enhance influence by developing niche expertise and sustaining credible peacekeeping commitments \parencite{FLYNN_2019}.

\textbf{I – Interpret}  
The text applies most clearly to states with a tradition of peacekeeping and a reliance on international legitimacy, such as Ireland, Sweden, or Ghana. It does not apply equally to small states that pursue deterrence through hard military capability (e.g., Israel or North Korea). The article highlights peacekeeping as a legitimacy multiplier, but this may not hold in situations where peacekeeping missions are contested or fail to deliver political results. In the Irish Defence Forces context, the “so what” is clear: legitimacy derived from peacekeeping grants Dublin disproportionate diplomatic visibility, though only insofar as international partners still value UN-led operations. Limit: Flynn assumes UN peacekeeping will remain relevant. Implication: For Ireland, diminishing UN credibility could undermine this niche role.

\textbf{M – Methodology}  
The paper is a policy-oriented comparative analysis, drawing on secondary literature and examples rather than primary data. It falls into the category of expert commentary rather than empirical study. This makes it useful for conceptual framing, but less robust in evidentiary terms. Limit: absence of systematic data weakens claims. Implication: adopt with caution; the arguments are illustrative, not conclusive.

\textbf{E – Evaluate}  
Flynn’s claims align with broader small-state literature (Thorhallsson, Bailes) that stresses niches and legitimacy. Other scholars support the idea that small states benefit from visible international engagement. However, critics might argue that peacekeeping no longer yields the same reputational dividends it once did, particularly after contested missions in Lebanon, Mali, and the DRC. Limit: assumes continuity of peacekeeping’s prestige. Implication: while relevant for illustrating Ireland’s past strategy, the argument is less predictive of future influence.

\textbf{R – (Autho)R}  
Flynn is a recognised scholar of maritime and security studies at NUIG, with a record of publishing on small states and Irish security policy. His institutional background suggests credibility but also a policy-aligned interest in validating peacekeeping as a worthwhile investment. There is no obvious evidence of overstatement, though the framing leans towards legitimising the Irish state’s peacekeeping tradition. Limit: potential normative bias towards peacekeeping as inherently good. Implication: requires balancing against realist critiques of peacekeeping’s effectiveness.

\textbf{S – Synthesis}  
Flynn’s article fits directly into the “niche specialisation” effect in the five-effects framework of small-state influence. It reinforces the argument that Ireland leveraged peacekeeping as a legitimacy anchor, but also illustrates fragility: niches are only as strong as the institutions that sustain them. Compared with constructivist takes on soft power, Flynn emphasises practice and institutional learning rather than identity. Limit: offers a positive reading but underplays peacekeeping failures. Implication: use as evidence for the case \textit{for} small-state influence, while setting up critique later in the essay.

\bigskip
\textbf{PEEL Paragraph}  
\textit{Point:} Flynn (2019) argues that small states can “punch above their weight” by enhancing peacekeeping capabilities and learning from other countries.  
\textit{Evidence:} He notes that states such as Ireland have built credibility by consistently deploying forces to UN operations, mirroring how Nordic states leveraged similar commitments \parencite{FLYNN_2019}.  
\textit{Explain:} This supports the claim that legitimacy is a small state’s centre of gravity, as peacekeeping visibility amplifies influence disproportionate to material power.  
\textit{Link:} However, as peacekeeping’s prestige wanes, Flynn’s optimism may be time-bound. This strengthens the essay’s overall argument that legitimacy remains fragile and contingent for small states.  
  
\parencite{FLYNN_2019} states that ``. If small states want to be relevant and influential in peacekeeping, they need to figure out how to offer force packages that cross a threshold well beyond the tokenistic or niche nor have them burdened by excessive national political caveats that limit their operational flexibility. This implies land units of at  least reinforced company size, and credible aerial and maritime assets as well''. This is of note in the context of Ireland's shrinking peacekeeping footprint - indeed the possibility that peacekeeping is a dead. Given the impending dissolution of UNSCR 1701, the withdrawal from UNDOF and the withdrawal from Africa (save for a single officer in Uganda), Ireland's peacekeeping footprint has withered significantly from the first decade of the century. This is of note in the context of Ireland identifying its peacekeeping as a legitimate and welcome projection of influence and execution of foreign policy. Does this sound the death knell for a substantial portion of Ireland's international legitimacy? Does it require a revision of Ireland's ways and means of international influence? Since evolving from a poor small state to a wealthy one (following the Celtic Tiger), Ireland appears to have developed notions of her own self importance on the world stage. In that context, Ireland has long traded on the legitimacy derived through perceived neutrality and virtue through peacekeeping. Ireland's political policies have evolved from the pragmatic realpolitik of Eamon de Valera to the structural-realism of today. Indeed, Ireland's structural realism is concealed behind performative liberalism. Staying rich safe in noble clothes. The public actually believes. Underneath is a state which knows it can't defend itself so it it does the virtuous tokenism of ration-packs and old armoured vehicles. Virtue signaling pragmatism. Ireland's use of peacekeeping  (constrained to those with a UN mandate) as a foreign policy tool projecting legitimacy has always been vulnerable to the whims of the UN Security Council and the continued use of peacekeeping in post-conflict areas. During the Cold War, while the bipolar world resulted in many successful UN missions and the prestige/legitimacy which went with them. However, in recent years, the Security Council is increasingly dysfunctional, hamstringing Ireland's `means' of foreign policy. The imminent dissolution of UNSCR 1701 is case and point. Furthermore, the legitimacy of peacekeeping as an endeavour has been targeted in the hybrid grey zone. For example, there were suggestions of collusion between UNIFIL and Hezbollah, or that UNIFIL is a Hezbollah proxy. The number of new UNSCRs is [GPT find the answer, it's fuck all .give me the source], which evidences the waning of peacekeeping as a tool. The UNSCR can no longer reach a consensus. Hence, it is assessed that peacekeeping is likely in its twilight. Given that assessment, what does it mean for Ireland's foreign policy means? Concurrently, the world appears to be returning to a realist or structural realist approach to policy - with rearmament and cynicism coming to the fore. \parencite{QUINN_2018} is of note where he frames peacekeeping as both and an Irish identify and foreign policy means, enabling Ireland ot sustain legitimacy disproportionate to its material power. However, his commentary has limited generality - applying to small states which use peacekeeping as a reputational means. Peacekeeping is ineffective where it's contested or where its prestige has waned. It appears that Quinn is offering a romanticised view of peacekeeping, through a liberal rather than a structural-realist lens. By highlighting comments from Obama and Ki Moon, Quinn draws on what Thorhalsson would describe as her `perceptual size'. Quinn states that ``Tonra ultimately concludes that the purpose of Ireland’s engagement in international  peacekeeping, consistent campaigning and lobbying on human rights issue and genuine  leadership position with relation to non-proliferation demonstrates that Ireland seeks to 
increase its security through multilateral security cooperation initiatives rather than by force of  arms'' \parencite{TONRA_2007}.  I concur in full with Tonra's analysis and believe that is supports my contention that Ireland's foreign policy is best viewed through the structural-realist lens which is concealed behind performative liberalism. Quinn further states ``In the initial stages of this research the  author encountered the idea that Irish peacekeeping is more concerned with the values-based 
ethical imperatives of peacekeeping: the simple act of contributing and doing good in the  interests of international peace and security''. This reminds the author of a familial debate this week following calls by certain celebrities that the Government should escort a shipment of aid to Gaza and provide protection (from the IDF). When it was clear that no military vessel could be supplied, the discussion moved to the Government supplying a civilian ship to observe. When challenged by the author that the IDF are very much a realist state who only understands power, that this symbolic gesture would literally be meaningless to the IDF, the response was that there is a moral imperative be seen to do something, even if it won't be successful. GPT can you use this anecdote or something from it>?

\section*{DIMERS Analysis – Duman \& Rakipoğlu (2025)}

``This paper argued that the primary source of the Council’s paralysis is the veto power, which, rather than serving its intended purpose of preventing great-power conflict, has been  systematically instrumentalised to advance national interests at the expense  of humanitarian imperatives.'' 

\textbf{D – Describe}  
\parencite{DUMAN_2025} analyse the UN Security Council’s paralysis during the Gaza war (Oct 2023–Jan 2025). They find that the Council voted 13 times; only 4 resolutions passed while 9 failed, largely due to 6 U.S. vetoes blocking ceasefire language. The article argues that veto use has shifted from a safeguard of consensus to an instrument of obstruction, undermining the UNSC’s credibility as a guarantor of collective security.  

\textbf{I – Interpret}  
This applies to states like Ireland whose foreign policy legitimacy is bound to UN peacekeeping mandates. If the UNSC cannot generate or sustain credible resolutions, small states lose their primary platform. For Ireland, the potential dissolution of UNSCR 1701 exemplifies how legitimacy rooted in peacekeeping may vanish. Limit: focus is Gaza-specific, but implications extend globally. Implication: peacekeeping as a foreign policy “means” is structurally eroding.  

\textbf{M – Methodology}  
Qualitative content analysis of UNSC debates, voting records, and draft resolutions. Robust in terms of primary evidence. Limit: case-specific to Gaza timeframe. Implication: strong empirical backing for claims of institutional paralysis.  

\textbf{E – Evaluate}  
The findings align with critiques that UN peacekeeping is in twilight (cf. \parencite{FLYNN_2019} optimism looks dated; \parencite{QUINN_2018}’s soft-power narrative is fragile). Duman adds empirical weight by demonstrating UNSC’s dysfunction, reinforcing the realist critique that small states relying on UN legitimacy are increasingly exposed.  

\textbf{R – (Autho)R}  
Both are credible academics in Middle East/IR fields. Their perspective is shaped by regional focus on Gaza/Israel but not by Irish institutional bias. Implication: useful external corroboration of Ireland-centric critiques.  

\textbf{S – Synthesis}  
Duman \& Rakipo\u{g}lu (2025) provide the hard evidence that the UNSC is structurally paralysed, reinforcing Cole’s (2018) point about realism cloaked in liberal rhetoric and King’s (2021) concern that legitimacy without resources is brittle. Together, they show Ireland’s peacekeeping legitimacy is tethered to a failing institution.  

\textbf{PEEL Paragraph}  
\textit{Point:} Duman \& Rakipo\u{g}lu (2025) reveal the UNSC’s paralysis in Gaza, where 9 of 13 resolutions failed due to vetoes.  
\textit{Evidence:} They show the U.S. blocked six ceasefire drafts, prioritising alliance politics over humanitarian imperatives \parencite{DUMAN_2025}.  
\textit{Explain:} For Ireland, which anchors its legitimacy in UN mandates, this collapse erodes the very stage on which it projects influence.  
\textit{Link:} Combined with Flynn’s (2019) insistence on non-tokenism, Quinn’s (2018) romanticised soft power, and King’s (2021) resourcing critique, Duman underlines that peacekeeping is not just fragile — it may be obsolete.  

\textcite{DUMAN_2025} demonstrate that the UNSC’s structural dysfunction is starkly evident in its handling of the Gaza crisis (2023–2025).  Across thirteen draft resolutions, only four were adopted, while nine failed, primarily due to six U.S. vetoes blocking ceasefire initiatives.  Even adopted texts were hollowed out, with Washington demanding the removal of terms such as ``permanent'' ceasefire and recasting Resolution 2728 as ``non-binding''.  The authors argue that the veto has shifted from a mechanism for consensus to an instrument of strategic obstruction, paralysing the Council’s ability to address humanitarian crises.  This paralysis has profound implications: it enables Israel to continue projecting military power despite growing international condemnation, because its actions are structurally protected by U.S. vetoes within the Security Council.  In realist terms, Israel’s capacity to ignore normative pressure is not a function of its own legitimacy but of American shelter.  For small states like Ireland, the lesson is sobering: reliance on the UN for legitimacy is increasingly untenable when great-power sponsorship, rather than institutional consensus, dictates outcomes. \textcite{DUMAN_2025} The authors suggests peacekeeping’s obsolescence, pushing Ireland toward structural-realist adaptation.

                                                                                                                          
\parencite{HIRST_2010} descriptions of Israeli conduct towards Lebanon clearly paint a reaslist perspective, where power was wielded primarily by the military. He states that ``a nation born of the sword was forever going ot live by it''. Indeed, the suggestion by Ben Gurion that the foreign ministry's job is to explain the actions of the defence ministry to the west of hte world underpins this. \parencite[p. 53]{HIRST_2010}. It is of note that such an aggressive perspective is at odds with considerations that Israel is a small state (read small power).

\parencite{ROTHSTEIN_1966} notes that ``Intervention was also bound to cause trouble between the rest of the nonaligned states and the two superpowers; and  as their votes in the UN and their political support on a wide range of issues became increasingly more important than their military contribution, that too became a consideration''. This highlights another aspect of  Thorhallsson's ``perceptual size'' \nocite{THORHALLSSON_2006}, where a small state may have elevated influence due to their votes within an international institution such as the UN. It is of note therefore that the increasing logjammed nature of hte UNSCR diminishes the influence of small states who leverage that multilateral organisation. Data from \parencite{HELLMUELLER_2024} demonstrate that the \textit{establishment of new UN peace operations has collapsed since the 1990s}. During the immediate post-Cold War decade, the Security Council authorised on average five to six new missions per year (1991--1995). This rate declined to roughly two to three new missions per year through the 2000s.Since 2012, however, the creation of new peacekeeping operations (PKOs) has virtually ceased, with the few new mandates established being overwhelmingly \textit{special political missions} (SPMs) or special envoys/advisors rather than robust peacekeeping deployments. The empirical trend is stark: whereas the 1990s represented the high-water mark of UN activism, the 2010s and 2020s have seen the Security Council \textit{almost entirely abandon the authorisation of new PKOs}.This pattern provides strong evidence for the contention that peacekeeping is in structural decline.For small states such as Ireland, which have historically derived disproportionate legitimacy and influence from participation in UN peacekeeping, the erosion of new mandates signals the waning of a core foreign-policy instrument. Where once Ireland could reliably translate its neutrality and peacekeeping contributions into international credibility, the paralysis of the Security Council now undermines this niche. In strategic terms, the \textit{ends} (influence and legitimacy) no longer align with the available \textit{ways and means} (credible peacekeeping deployments). This decline therefore marks not only the twilight of UN peacekeeping but also the need for small states to reconsider alternative avenues for sustaining international relevance. 

Flynn (2019)\nocite{FLYNN_2019} cautions that small states must avoid tokenism if they wish to remain relevant in peacekeeping: only reinforced company-level contributions, with credible air and maritime assets, cross the threshold of meaningful influence. Ireland’s shrinking footprint—complete withdrawal from UNDOF in 2024, near-total exit from African missions (save for minimal support in Uganda), and the planned phaseout of its UNIFIL deployment under UNSCR 1701 by 2027—suggests it has now fallen below that threshold, with only around 300 troops remaining in Lebanon as of August 2025 amid the mission's final mandate extension to December 2026. The realist implication is stark: peacekeeping, once Ireland’s primary niche and source of legitimacy, is entering twilight, exacerbated by UN Security Council dysfunction, with only about 25 resolutions adopted by September 26, 2025, and no new peacekeeping missions established since 2014. \nocite{QUINN_2018} Quinn (2018) frames peacekeeping as an expression of Irish identity and a liberal foreign-policy tool, but this is increasingly untenable in a multipolar world where consensus on mandates wanes. As de Valera’s wartime neutrality revealed \parencite{FANNING_2015,AYIOTIS_2023}, Ireland’s posture has always been realist: a veil of moralism masking strategic hedging. Today, performative liberalism sustains the illusion of influence, but beneath it lies structural realism and material weakness, with defense spending stagnant at around 0.3 percent of GDP. My own experience of recent debates about Gaza illustrates this tension: while public figures argued for symbolic gestures such as escorting aid ships or deploying civilian observers, the reality is that Israel, as a hard realist state, is unmoved by such symbolism. Gestures may satisfy domestic expectations, but they do not alter outcomes. This underscores the risk that Ireland’s legitimacy, long derived from peacekeeping, is collapsing into symbolic performance without strategic effect. Quinn \nocite{QUINN_2018} reports on perceptions regarding Ireland's participation in peacekeeping as values-based rather than transactional. I suggest that it was naive of him not to consider the realist or neo-realist aspect of Ireland's participation. I assess that while Ireland has a while the performative libaral aspect of peacekeeping is real (and certainly believed by the public), its origins have been realist in nature. Given \parencite{WALTZ_1979}'s concepts of structual realism, I assess that Ireland's IR engagement is structiural-realist in nature. 

Flynn (2019)\nocite{FLYNN_2019} cautions that small states must avoid tokenism to remain relevant in peacekeeping, requiring reinforced company-level contributions with credible air and maritime assets to cross the threshold of meaningful influence. Ireland’s shrinking footprint—complete withdrawal from UNDOF in 2024, near-total exit from African missions (save for minimal support in Uganda), and the planned phaseout of its UNIFIL deployment under UNSCR 1701 by 2027—suggests it has fallen below this threshold, with only around 300 troops remaining in Lebanon as of August 2025 amid a final mandate extension to December 2026. The realist implication is stark: peacekeeping, once Ireland’s primary niche and source of legitimacy, is entering twilight, exacerbated by UN Security Council dysfunction—only 25 resolutions adopted by September 26, 2025, with no new peacekeeping missions since 2014 \parencite{HELLMUELLER_2024}. Quinn (2018) \nocite{QUINN_2018} frames peacekeeping as an expression of Irish identity and a liberal foreign-policy tool, but this is increasingly untenable in a multipolar world where consensus wanes. As de Valera’s wartime neutrality revealed \parencite{FANNING_2015,AYIOTIS_2023}, Ireland’s posture has always been realist, a veil of moralism masking strategic hedging. Today, performative liberalism sustains the illusion of influence, yet beneath it lies structural realism and material weakness, with defense spending stagnant at 0.3 percent of GDP. My recent experience in Gaza aid debates underscores this tension: public figures advocated symbolic gestures like escorting aid ships, yet Israel, a hard realist state, remains unmoved, highlighting that such actions satisfy domestic expectations but lack strategic effect. This risks collapsing Ireland’s legitimacy into symbolic performance. Quinn’s (2018) view of values-based peacekeeping overlooks this realist foundation; I assess, per Waltz (1979), \nocite{WALTZ_1979}that Ireland’s engagement is structurally realist, necessitating a pivot—e.g., to EU cyber niches—to sustain influence amid failing UN frameworks


\section*{Reading Priorities for DSS Essay (Mk2 Structure)}

Based on the Mk2 essay framework \parencite{DSS_STRUCTURE_MK2}, the bibliography divides into two categories: 
\textbf{Key Papers} (must-read, directly tied to your five effects and central debates) and 
\textbf{Supplementary Papers} (supportive, comparative, or contextual). Within each, sources are ordered by importance.

\subsection*{Key Papers}
\begin{enumerate}
	\item \textbf{Thorhallsson (2006)} — foundational on small-state size and shelter theory; frames Niche and Legitimacy \parencite{THORHALLSSON_2006}. Read 24/9/25.
	\item \textbf{Keohane (1969)} — classic on dilemmas of small states; baseline for institutionalist optimism \parencite{KEOHANE_1969}. Read 25/9/25.
	\item \textbf{Keohane (1988)} — contrasts rationalist vs. reflective institutionalism; crucial for Organisational Agility and Legitimacy \parencite{KEOHANE_1988}. read 29/9/25
	\item \textbf{Krepinevich (1994)} — on military revolutions; core to Niche Specialisation \parencite{KREPINEVICH_1994}. read 15/9/25
	\item \textbf{Gray (2005)} — sceptical of change; anchor for continuity vs. transformation \parencite{GRAY_2005}. read 29/9/25
	\item \textbf{Gray (2018)} — strategy as political consequence; grounds Legitimacy effect \parencite{GRAY_2018}. read 20/9/25
	\item \textbf{Farrell \& Newman (2019)} — weaponised interdependence; defines Hybrid Leverage \parencite{FARRELL_2019}.
	\item \textbf{Nye (2008)} — soft power as credibility; central to Soft Power Synergy \parencite{NYE_2008}.
	\item \textbf{Mearsheimer (1994)} — ``False Promise''; realist scepticism of institutions \parencite{MEARSHEIMER_1994}.
	\item \textbf{Mearsheimer (2019)} — ``Bound to Fail''; critique of liberal order; frames limits on small-state agency \parencite{MEARSHEIMER_2019}.
	\item \textbf{Bessner \& Guilhot (2015)} — realism and intellectual legitimation; counterpoint for Legitimacy \parencite{BESSNER_2015}.
	\item \textbf{Tonra (1999)} — Europeanisation of Irish foreign policy; empirical case study \parencite{TONRA_1999}.
	\item \textbf{Tonra \& Christiansen (2011)} — theoretical bridge between IR and EU studies; frames constructivist angle \parencite{TONRA_2011}.
	\item \textbf{Cohen (2002)} — civil–military dialogue; anchors Legitimacy and Agility \parencite{COHEN_2002}.
	\item \textbf{Bailes \& Thorhallsson (2013)} — small states and EU shelter; legitimacy and synergy \parencite{BAILES_2013}.
\end{enumerate}

\subsection*{Supplementary Papers (Comparative/Contextual)}
\begin{enumerate}
	\item \textbf{Waltz (1979)} — theoretical anchor for neorealism; supports sceptical side \parencite{WALTZ_1979}.
	\item \textbf{Bailes \& Thorhallsson (2012)} — EU as small-state instrument; reinforces institutionalist case \parencite{BAILES_2012}.
	\item \textbf{EU Global Strategy documents (2016; 2017; 2018; 2019)} — contextual policy framing for EU role \parencite{EU_2016,EU_2017,EU_2018,EU_2019}.
	\item \textbf{HLAP (2022)} — Irish organisational reform commitments; empirical grounding \parencite{HLAP_2022}.
	\item \textbf{Commission on the Defence Forces (2022)} — context for organisational agility debates \parencite{CODF_2022}.
	\item \textbf{DoD Strategy Statement (2025)} — legitimacy and reform trajectory \parencite{DOD_2025}.
	\item \textbf{White Paper Update (2019)} — policy continuity and limits \parencite{WHITE_2015}.
\end{enumerate}


GPT: ensure that these are used somwhere. they're my own words "Introduction (Chapter 1)
For Colin Gray, strategy is about political ends, not reach or effects. Reach belongs to tactics. Operations are the sequencing of tactics. Strategy exists only when means are aligned to political purpose. From this view, coercive, controlling, or collaborative approaches describe relationships of power, not technical reach. Small states, lacking hard power, can only pursue the collaborative.

Body (For/Against/Discussion)
Collaboration can still matter politically: small states gain visibility and legitimacy by aligning their limited means to multilateral ends. However, the absence of coercive or controlling options reveals a structural ceiling. If strategy is only about political ends, small states face a hard limit. They may contribute tactically or organise operationally, but they cannot set the strategic terms of international security.

For the proposal (collaboration as influence)
Point. Small states can exercise influence through collaborative strategies.
Evidence. Ireland’s long participation in UN peacekeeping shows how modest forces can gain legitimacy and visibility.
Explain. This matters because collaboration aligns limited means to political ends, which Gray argues is the essence of strategy.
Limit. Such influence depends on multilateral recognition and is easily marginalised.
Consequent. Collaboration offers visibility but not decisive power.

Against the proposal (hard ceiling on influence)
Point. Small states cannot pursue coercive or controlling strategies.
Evidence. Without military mass or economic weight, they cannot compel outcomes or dictate terms.
Explain. For Gray, strategy is political ends; tactics and operations deliver reach and effects. Small states cannot translate these into control.
Limit. Even strong niche roles remain dependent on larger actors.
Consequent. Small states face a structural ceiling on their strategic influence.

Synthesis signpost
Taken together, these points show that small states can influence only through collaboration. This influence matters, but it remains conditional and secondary to the power of larger actors."