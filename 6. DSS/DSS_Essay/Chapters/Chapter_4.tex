\chapter{Discussion}


'NORM SETTING' bhy Dr. Gëzim Visoka (DCU)
SO FAR THE EVIDENCE SHOWS THAT IT MATTERS
IRELAND, SPAIN, NORWAY (DENMARK??), NEXT PHASE LIKELY WOULND’T HAPPEN WITHOUT IRELAND AND SMALL STATES. Ireland’s leadersjip on the Palestine matter indicates that small states can have disproportionate influence
He said that normally, the recognition of statehood is a matter for the big 5


In his review of 2021, \parencite{MCDONAGH_2021} highlighted the divisions crated by the challenge to source PPE in response to the Covid-19 pandemic. Ireland, whose policy of institutional engagement and multilateralism, found herself like smaller animal when elephants were fighting. ``competition for essential supplies could lead to fracturing of instutitions'' . he also noted that Ireland, while particpating in Europe, quietly hedges economically with our corporate taxation rate. While it regularly comes under-fire, Ireland's corporate taxation policies are realist in nature. For example, Apple's case by the European Commission (14 buillion). `` The Apple case is a good example of the fine line Ireland walks in relation to 
its relationship with the EU: while Ireland tends to be a constructive member 
state, taxation policy has been a key plank of its FDI strategy since the early 
1960s and is seen as a core national interest. The Apple case found Ireland in the 
odd position of arguing against the receipt of several billion in additional tax 
(although not all of the moneys concerned would end up in the Irish exchequer), 
but it was felt that the principle of having an independent taxation policy was 
worth the reputational risk even at a time when Ireland depended on the support..'' Furthermore, Ireland has long used the ``shelter'' of the United States to leverage its influence. The first Trump administration took note of Ireland's corporate tax rules, which were seen as an unfair way for US companies to avoid paying the fair share of tax at home. Ireland as an ``instinctively multilateral country'', was never likely to have much in common with the Trump administraoitn.

2020 of an Asia-Pacific strategy...to increase Ireland’s visibility in the 
region—politically, economically and culturally—with a view to continuing the 
trajectory of growth in economic partnerships while recognising the need to 
engage with Asia-Pacific partners on issues such as climate change, security and 
development.This geopolitical shift demands a recalibration of Ireland’s 
traditional foreign policy focus, which the new Asia-Pacific strategy signals