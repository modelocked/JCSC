\chapter{Module 1: \gls{clem}}

The Command, Leadership and Management module combined presentations, academic frameworks, psychometric testing and applied exercises. I invested considerable effort in the assessed presentation with Finola. We worked well together and it was rewarding to return to public speaking, although I remain uncertain how much genuine learning it produced. My judgement may shift once I receive results and feedback, but at present I see it as worthwhile practice rather than new knowledge.  

Cranfield University provided academic input led by Dr Dennis Vincent. His sessions were of high quality, introducing models such as the seven S’s, the EAST framework and the force field model. These were well delivered, but much of the content was already familiar at this stage of my career. They were useful as reminders and provided a shared vocabulary for the group, but they offered limited novelty. By contrast, the ethics discussion on the final day was genuinely valuable, though curtailed by time. It was the most engaging part of the week, yet did not receive the depth it merited. Several peers also noted that it was the only session that generated authentic debate, which highlights a collective appetite for deeper ethical engagement.  

Other elements were less effective. The outsourced strategic communications session was weak and added little professional value. The facilitators did not critique our work and the exercise became superficial. From a participant’s perspective this was frustrating, but it also underlined the importance of quality assurance when relying on external providers. The workload associated with the thesis proposal further distracted from engagement in class, a reminder that competing academic tasks can dilute the impact of professional education.  

The Emergenetics exercise was useful mainly as an icebreaker and a shared language for the class. It did not reveal anything new about myself. I was surprised that some peers were shocked by their profiles, which seemed self-evident. The value for me was in building group cohesion rather than personal insight. Informal conversations also shaped my reflection. Brian and I frequently discussed aspects of the course, including our presentations. His topic on diversity led us into debates about its place in the Defence Forces and more broadly. I noticed that his views were sympathetic and empathetic, whereas mine were more sceptical. This scepticism is not casual but informed by my long-standing reading of critical theory and its critiques. Thinkers such as Foucault \parencite{FOUCAULT_1977}, Derrida \parencite{DERRIDA_1978} and more recent commentators like Pluckrose and Lindsay \parencite{PLUCKROSE_2020} or Soh \parencite{SOH_2021} have shaped my suspicion that activist scholarship can over-extend concepts of power and identity. The contrast with Brian’s empathetic stance highlighted how personal reading and intellectual formation shape interpretation of leadership themes. It also showed me that while I approach such issues analytically and critically, others place greater weight on values of inclusion and empathy. Recognising this divergence is important if I am to avoid allowing my scepticism to close off dialogue on topics that matter to colleagues.


Commandant Gavin Egerton’s session on mission command was more relevant. The key lesson was that leaders must sometimes intervene at the lowest level. This can be labelled micromanagement, however if done with judgement it is consistent with mission command as described in Defence Forces doctrine \parencite{DF_LDR_2023}. It also aligns with Hersey and Blanchard’s situational leadership model, which emphasises adapting directive and supportive behaviour to the ability and willingness of subordinates \parencite{HERSEY_1969}.  

The methodology of the module was broad, combining academic theory, peer-led discussion and psychometric testing. This provides a mid-level of evidence: stronger than anecdote but weaker than empirical study. Its strength was in dialogue and exposure to multiple perspectives. Its weakness was inconsistency, with rigorous academic input sitting alongside content that was superficial.  

My evaluation is that the module reaffirmed principles I already held: that credibility, ethics and trust are the foundation of leadership. It did not radically alter my views, but it reinforced them. The ethics discussion, though brief, validated my experience of command where fairness and consistency mattered more than authority. The strategic communications component fell short, yet the lesson is that not all inputs will have equal value and that selective adoption is part of professional learning.  

My stance is shaped by assertiveness and scepticism. I am quick to dismiss what I perceive as superficial, which risks closing me off from potential insights. Recognising this bias is important. Even flawed inputs can prompt reflection if approached critically. Going forward I intend to apply at least one of the models, such as the force field or the seven S’s, to a Defence Forces scenario to test its utility before discarding it. In this way I can balance scepticism with openness and continue learning, in spite of myself.  

\textbf{Thesis:} The CLEM module offered limited novelty but reinforced core principles of ethics and credibility, reminding me that selective adoption and critical openness are essential to professional growth.  

\begin{itemize}
	\item Ethics discussion was too brief to be conclusive; implication: Defence Forces education should allocate more time to structured ethical debate to deepen reflection.  
	\item Outsourced inputs lacked rigour; implication: external facilitators must be quality assured to protect professional credibility.  
	\item Emergenetics revealed little new insight; implication: psychometric tools should be framed primarily as cohesion-building instruments rather than as diagnostic of individuals.  
\end{itemize}

\textbf{Next step:} Apply one academic model from the module to a live Defence Forces planning problem to assess its practical value, ensuring I balance scepticism with a willingness to test unfamiliar frameworks.  
