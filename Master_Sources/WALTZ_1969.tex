\documentclass[12pt,a4paper]{article}

% Packages
\usepackage[T1]{fontenc}
\usepackage[utf8]{inputenc}
\usepackage{lmodern}        % cleaner fonts
\usepackage{geometry}       % control page layout
\geometry{margin=1in}
\usepackage{setspace}       % for line spacing
\usepackage{fancyhdr}       % for headers/footers
\usepackage{titlesec}       % tidy section spacing
\usepackage{verbatim}       % to allow verbatim blocks

% Header/Footer
\pagestyle{fancy}
\fancyhf{}
\lhead{Transcribed Text}
\rhead{\thepage}

% Section formatting
\titleformat{\section}{\large\bfseries}{\thesection}{1em}{}

% Document
\begin{document}
	
K. N. WALTZ "Theory of international politics" (1979).
"Laws establish relations between variables, variables being concepts that can take different values. If a, then b, where a stands for one or more independent varaibales and b stands for the dependent variable:  In form, this is the statement of a law. IF the relation betwen and b is invariant, the law would read liek this: If a, then b with probability x. A law is based on not simply on a relation that has been found, btut on one that has been found repeatdly. " "by defnition, theories are collections or sets of laws pertaning to a particular behaviour or phenonomenon. [...] Theories are, then, more complex than laws, but only quantitatively so. Between laws and theories no differenes of kind appears". "A theory is born in conjecture and is viable if the conjecture is confirmeD".

Waltz's chapter 6 of particular interest to those considering the impact which small powers have on the international stage. "The state among states, it is often said, conducts its affairs in the brooding shadow of violence. Because some states may at any time use fore, al lstates must be prepared to do so - or live at the mercy of their militar ymore vigorous neighbors. Amonst states, teh state of nature is a state of war. This is meant not in the sense that war constantly occurs but in the sense that, with each stat deciding for itself whether or not ot use forces ,war may at any time break out. Whether in the family, the community, or the world at large, ocntact without at least the occasional conflict is conconveivable; and the hope that in the absence of an agent to manage or manipulate conflicting partiels the use of fore wil lalways be avoided cannot be realistically entertained. Among men as among states, anarcy, or the absence of government, is associated with the occurrence of violence." Waltz speaks of the ``division of labour" among states. "integration draws the parts of a nation closely together. Interdependence among nations leaves them loosely connected. Although hte integration of nations is often talked about, it seldom takes place. Nations could mutually enrich themselves by further dividing not just hte labour but that goes into the production of goods nut also some ofhte other tasks they perofrm, such as political management and military defense [...] In a self-help system each of the units spends a portio nof its effort, not in forwarding its own good, but in providing the means of protecting itself against others. Specialisation in a system of divided labour works to everyone's advantage, though not equally so. Inequality in the expected distribution of increased product works strongly against extension of the division of labour internationally. When face with the possiblity of cooperating for mutual gain, states that feel insecure must ask how the gain will be divided.  [..] In any self-help system, units worry about their survival, and the worry conditoins their behaviour. Oligopolistic markets limit the cooperation of firms in much te way that international-political structures limit the copperation of states. Witin the rules laid down by governments, whether firms survive and propser depends on their own efforts. Firms need not protect themselves physically agianst assaults from other firms. They are free t oconcentrate on their economic interests. As economic entities, however, they live in a self-help world. All want to increase profits. If they run undue risks in the effort t odo so, they must expect to suffer the consequences.""

\section*{DIMER Analysis of Waltz (1969)}

\subsection*{Describe}
\textcite{WALTZ_1969} outlines the systemic constraints of an anarchic international order, advancing the idea that states exist in a self-help system where survival is the overriding imperative. He compares state behaviour to oligopolistic firms: cooperation is possible, but always fragile and conditioned by fears of relative gains. Waltz stresses the inevitability of the security dilemma: defensive measures by one state appear threatening to others, driving cycles of competition and mistrust. His central claim is that the distribution of capabilities determines outcomes, not intentions or norms.

\subsection*{Interpret}
The argument applies most strongly to great-power competition, where balance-of-power politics dominates. For small states, Waltz implies they are system-takers: their autonomy is constrained by structural forces, and their survival depends on alignment, prudence, or shelter. This undercuts institutionalist optimism (e.g., Keohane \parencite{KEOHANE_1969}) and constructivist accounts (e.g., Tonra \parencite{TONRA_1999}) that highlight agency through norms. The “so what” for Ireland is sobering: neutrality, peacekeeping, or EU membership cannot alter systemic dynamics; at best, they provide temporary stability. However, Waltz does not account for legitimacy as an independent resource or for small states’ ability to exploit institutional niches.

\subsection*{Methodology}
Waltz employs a theoretical and analogical method, drawing on market oligopoly to model international politics. His analysis is deductive, not empirical, and sits at the level of grand theory. Its strength is parsimony and clarity; its weakness is lack of data and neglect of domestic, normative, or institutional variables. Reliability is strong within realist paradigms, but external applicability to small states is limited.

\subsection*{Evaluate}
Waltz’s neorealism is foundational, providing a rigorous systemic explanation of state behaviour. Compared with Mearsheimer’s offensive realism \parencite{MEARSHEIMER_1994}, Waltz is more structural, seeing states as constrained balancers rather than aggressive power-maximisers. In contrast to Keohane’s institutionalism \parencite{KEOHANE_1988}, Waltz dismisses the ability of institutions to alter systemic incentives. While invaluable as a baseline, his theory underdetermines small-state agency and legitimacy. It remains a critical foil for testing institutionalist and constructivist claims about Ireland and other small states.

\subsection*{(Autho)R}
Kenneth Waltz, one of the most influential realist theorists, wrote from a U.S. academic context during the Cold War. His disciplinary stance biases him toward structural determinism, downplaying domestic politics and normative change. His authority makes his theory widely cited, but his scepticism of small-state influence reflects his context rather than universal applicability.

\subsection*{Limit → Implication}
Limit: Neorealism overstates systemic constraints and neglects the role of institutions, norms, and legitimacy.  
Implication: For Ireland, Waltz’s framework warns of constraints but must be adapted, recognising that small states can gain conditional influence through legitimacy and institutional niches.

\section*{Effects Mapping}
\begin{itemize}
	\item \textbf{Niche Specialisation:} Waltz is sceptical; niches are irrelevant unless tolerated by great powers.
	\item \textbf{Organisational Agility:} Agility is structurally constrained; reforms cannot offset systemic pressures.
	\item \textbf{Hybrid Leverage:} Absent from Waltz’s account; interdependence is framed only as vulnerability, not as leverage.
	\item \textbf{Soft Power Synergy:} Dismissed; norms and attraction cannot substitute for power.
	\item \textbf{Legitimacy:} Neglected as an independent force; survival rests on power and balance, not legitimacy.
\end{itemize}

\section*{PEEL-C Paragraph}
Waltz argues that international politics is governed by anarchy, forcing states to prioritise survival in a self-help system \parencite{WALTZ_1969}. For small states, this means limited scope for independent influence: niches, agility, or soft power are fragile under systemic constraints. This explains why Irish neutrality cannot guarantee security, as it depends on the tolerance of stronger states. However, Waltz neglects how institutions and legitimacy can create conditional space for influence, as later shown by Keohane and Thorhallsson. Consequently, Waltz serves as a critical realist baseline: his scepticism sharpens the need to demonstrate how small states defy structural limits through legitimacy and institutional embedding.
