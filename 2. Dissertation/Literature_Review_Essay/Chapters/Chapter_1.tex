\chapter{Literature}



\section{Introduction}Much of warfare conducted during the first two decades of the 21\textsuperscript{st}~century was heavily influenced by the IED. The threat and the response evolved continually. Crucially, techniques were rarely rendered obsolete. Instead, their effectiveness fluctuated over time. This evolutionary pattern accords with Krepinevich’s insistence that RMAs require organisational adaptation alongside technology \parencite{KREPINEVICH_1992} and Metz’s observation of incremental rather than discontinuous change \parencite{METZ_2000,KREPINEVICH_1994}. This is evidenced by the continued relevance of the Philosophy of IEDD, which endures as international doctrine since the 1970s  \parencite{COCHRANE_2012,DORD_2022}. An RMA would perhaps fundamentally transform the battlefield by rendering existing techniques and equipment permanently obsolete. This example may reflect what critics of the RMA thesis argue, that adaptation is cyclical and rarely results in true discontinuity. %[Consider integrating a brief reference to a key RMA theorist or critic who adopts a similar evolutionary view to ground the framing academically (e.g., Colin Gray or Steven Metz).]

 
%Krep's 1992 This article appears to foretell a lot from the future. the use of hybrid warfare by Russia and Iran to counter their clear inferiority in conventional forces. The requirement of agile and flat experimentation. The requirement of structures and people to implement the technological advancements. that older legacy systems can be used but in a different manner as newer systems take over the battle. the use of drones and long range precision fires instead of MBTs. that information is powera and that change in a bureacratic organisation can be very difficult. for example, the brits invented the tank and had thinkers who wanted to use it. but their system failed andinstead it wa sthe germans who could be agile enough to implement. that the use of civilian technolgoy is important and that the military can no longer practice a form of socialism where it owns all the tech. \parencite{KREPINEVICH_1992}

\section{Foundational RMA Theories and Techno-Optimist Views}

Cohen suggested to consider the longevity of, ``basic counting pieces of
military power"  \parencite{COHEN_1995}, echoing RCP. Discontinuous jumps in \gls{rcp} aligns well with \textcite{KREPINEVICH_1992}'s description, whereby to disregard an RMA likely condemns your soldiers to slaughter. 

Krepinevich 1992's paper placed him as a seminal RMA thinker \nocite{KREPINEVICH_1992}. He framed the RMA as more than just new platforms. It required agile, flat experimentation and organisation. He stressed that information itself would become power and military adoption of superior civilian technology. He also suggested that older systems could still play a role, but increasingly in support of new technologies such as long-range precision strike and drones which would displace armour as the centre of battle \nocite{KREPINEVICH_1992}. %His later work reaffirmed that true revolutions require the co-evolution of technology, operational concepts, and institutions, rather than technology alone \parencite{KREPINEVICH_1994}.%
 This sits alongside Cohen’s idea that RMAs function like hypotheses to be tested in war \parencite{COHEN_1996} and Owens’ vision of linked “systems of systems” reshaping force structure \parencite{OWENS_2002}. RMA optimists have felt that evolution in technology and thinking which were displayed during Desert Storm\index{Gulf War, 1990-1991} would clear the ``fog of war''  \parencite[p.~49]{ALACH_2008}. More recently, Brose argued that AI, autonomy and ubiquitous sensors will upend legacy concepts in the same way that smokeless powder once did, demanding a radically different force design built around swarms of cheap, expendable system \parencite{BROSE_2019}.

Their works suggest that Cohen and Krepinevich view the RMA concept as not just an abstract model within which to attempt to understand warfare. Instead an RMA is akin to a scientific hypothesis which can be tested. If one's inaction resulted in the `needless' slaughter of your soldiers, then one erroneously dismissed an RMA. It is clear from the literature, that proponents of the RMA framework do not simply see it as technological. Cultural, bureaucratic, financial, political and other factors influence to transform a technological invention to a military breakthrough. Metz (2000) similarly stresses that revolutions in military affairs are as much social, political and organisational as they are technological\nocite{METZ_2000}. The cultural context shapes whether new capabilities amount to true revolution.

\section{Critical and Skeptical Perspectives on RMA}

Krepinevich also recognised that technology on its own was insufficient. For him, invention without reorganisation was a dead end \parencite{KREPINEVICH_1992}. This reinforces Betts’ warning that militaries often misuse technology if it is not embedded in doctrine and culture \parencite{BETTS_1996}. Krepinevich’s review of the Gulf War air campaign likewise underlined the limits of technology’s impact when political control and cultural factors remained intact \parencite{KREPINEVICH_1996}. It also reflects Gray’s argument that strategic culture, not hardware, usually decides whether an apparent breakthrough becomes a genuine revolution \parencite{GRAY_2005}. Betts (1996) (and to a lesser extent Owens (2002)\nocite{OWENS_2002}) cautioned against concluding that the results of Desert Storm confirm the existence of an RMA\index{Gulf War, 1990-1991}. He assesses it as improbable that that the U.S., is able to repeat the results of the 1990-1991 Gulf War in the future. Indeed he appears to have correctly predicted the US' inability to succeed against insurgencies it faced during GWoT. William Owens  similarly highlights that while technological advances offer extraordinary military effectiveness, organisational and cultural transformation lag behind and dilute its impact \parencite{OWENS_2002}. 

The skeptical tradition applies a kind of Occam’s razor to claims of revolution. Rather than positing epochal breaks, it stresses that continuity and incremental adaptation usually suffice to explain apparent change. As Alach argues, much of the supposed RMA discourse is rhetorical excess layered onto what are, in practice, evolutionary adjustments \parencite[.p~50-52]{ALACH_2008}. His conclusion is that there is an evolution in military affairs rather than a revolution. Rassler’s 2015 study of non-state drone innovation reinforces this skepticism, showing that much of the apparent novelty stems from incremental civilian adaptation rather than decisive military transformation”  \parencite{RASSLER_2015}.

\section{Doctrinal and Organisational Learning Perspectives}

The tank case is also an example of organisational learning and failure. Krepinevich showed that Britain had the concept but not the structures to exploit it \parencite{KREPINEVICH_1992}. He later extended this logic, showing that historical patterns of military change confirm the decisive role of organisational adaptation over invention alone \parencite{KREPINEVICH_1994}. This appears to align with Betts' description of military commanders approaching new technology with ``conservative progressivism  '' \parencite{BETTS_1996}.  Conversely inter-war Germany's circumstances allowed for a more flexible approach, resulting in transformation. Metz's (2000) and Owens (2002) articles accurately correlates recent defeat or a perception of weakness with openness to creativity\nocite{METZ_2000,OWENS_2002}. This was clearly the case for inter-war Germany.



Cohen's (1996) and Krepinevich's (1992) articles suggest that revising hierarchical structures is among the hardest tasks in realising an RMA. The recurring prescription is not mere acquisition/invention. It is often organisational change, faster decision cycles, decentralised command which realise the gain. Yet both tend to understate the cultural, political, budgetary and career incentives which impede revolutionary change\nocite{COHEN_1995,COHEN_1996,KREPINEVICH_1992}. Owen's 2002 article clearly places him as an RMA optimist \nocite{OWENS_2002}. However, when reviewing Desert Storm\index{Operation Desert Storm|see{Gulf War, 1990-1991}}, he observed that apparent technological difficulties were, ``rooted in deeper differences of service culture, procedures, and operational concepts''. As Keller (2002) observed in his interviews with reform advocates, even within the Pentagon the frustration was less about technology itself than about a sclerotic culture that rewarded continuity over disruptive change\nocite{KELLER_2002}. The Stimson Center similarly noted that while UAS had become indispensable to U.S. commanders, their adoption was shaped less by radical reorganisation than by incremental integration into existing bureaucratic routines \parencite{STIMSON_2015}.

 Schaus and Johnson caution that UAS use also alters escalation dynamics, lowering thresholds without necessarily changing organisational cultures, which may create dangerous gaps between intent and perception” \parencite{SCHAUS_2018}. This is echoed by \textcite{KREPINEVICH_1992}.

%adam The lesson fits with Nagl’s claim that military culture decides whether adaptation happens (Nagl, 2002 NO -LINK) and with Fitzgerald’s description of doctrinal “amnesia” when lessons are ignored (Fitzgerald, 2013 - NO LINK). His work therefore links early RMA theory with later accounts of how culture and doctrine shape learning.

\section{AI, Autonomy, Mission Command and Contemporary Debates} 
In 2000, Metz  posited that historic commanders such as  or Guderian would likely have found the \gls{usalb} compatible with their operational style\index{AirLand Battle}\index{Doctrine!AirLand Battle}\index{Mission Command!AirLand Battle}\nocite{METZ_2000}. Central to this is the concept of mission command\index{Doctrine!Mission Command}. The tension between mission command's rapid operational tempo and the temptation of micro-management due to digital visibility is noteworthy.

Cohen warns of that technologies facilitating modern to commanders to perch ``cybernetically'' beside their troops in combat could have an insidious effect - undermining subordinate commanders \parencite{COHEN_1996}. He contrasts this with General Eisenhower and Field Marshall von Moltke lying on a sofa reading a book on the eve of battle. His skeptical characterisation of commanders prone to intervene indicates that Cohen is likely a proponent of mission command. Yet as Betts observes, new technology does not sharpen judgment by itself. Militaries often misinterpret or misuse innovations \parencite{BETTS_1996}. Together these cautions suggest that without cultural and organisational restraint, information systems may become instruments of centralised control rather than enablers of mission command. Alach (2008) concurs, stating that, ``mental evolution was as critical", as technological progress\nocite{ALACH_2008}.

As a proponent of mission command, the actions recounted by General Guderian are somewhat incongruent\index{Auftragstaktik| see{Mission Command}}\index{Mission Command!Auftragstaktik}\index{Guderian, Heinz}\index{Mission Command!Guderian, Heinz}. He described driving around the battlefield in his staff car to observe and intervene, at times confronting subordinates he believed were disobeying orders. One such episode involved SS Oberf\"uhrer  Dietrich\index{Dietrich, Sepp}\index{Mission Command!Guderian, Heinz}, whose apparent insubordination proved correct once Guderian examined the situation: ``I approved the decision taken by the commander on the spot'' \parencite[p.~117]{GUDERIAN_1952}. His preference for active oversight was further reflected in his praise for SS Gruppenf\"uhrer Paul Hausser\index{Hausser, Paul}\index{Mission Command!Hausser, Paul} \parencite[p.~73]{YEIDE_2011}. General George S. Patton was similarly known for a hands-on style often criticised as ``micromanaging'' \parencite[p.~34]{ZALOGA_2010}\index{Patton, George S.}\index{Mission Command!Patton, George S.}.  

The tension between personal intervention and the philosophy of mission command is illustrated by Major Dick Winters\index{Winters, Dick}\index{Mission Command!Winters, Dick} during the Battle of Bastogne. Though instinct urged him to relieve Lieutenant Dike and lead Easy Company himself, Winters chose instead to uphold his broader responsibilities as battalion commander. His solution of appointing Lieutenant Ronald Speirs proved decisive \parencite[p.~186]{WINTERS_2006}.  

Steven Metz argues that the USALB blended modern technology with Auftragstaktik and rapid tempo in ways Guderian or Patton would have recognised \parencite{METZ_2000}. More recently, Husain suggested that AI could compress the OODA loop to machine speed, accelerating conflict while still leaving space for the enduring principles of mission command \parencite{HUSAIN_2021}.


So it appears that Cohen underestimates the potential for technology to reinforce Auftragstaktik. His framing risks a reductive binary between passive commanders ``on the sofa” and intrusive commanders ``in the hatch”. Even if a commander’s judgment is superior, they cannot personally manage every decision across a dispersed and complex battlefield. What matters is leveraging subordinates’ initiative at scale, while reserving intervention for moments of crisis or opportunity. In this sense, the commander amplifies rather than replaces subordinate action. Hence, technology such as uncrewed systems and AI may enhance mission command. Krepinevich (1992) modified structures can enable technology to reinforce Auftragstaktik rather than undermine it\nocite{KREPINEVICH_1992}. This highlights that the impact of technology is contingent less on the tools themselves than on organisational culture and doctrinal restraint. In favourable conditions, AI and uncrewed systems may extend, rather than erode, the practice of mission command. %Ochs' findings within multinational units are of note, wherein culture and interpretation can blunt mission command - despite the use of digital tools which would otherwise enhance it \parencite{OCHS_2020}.
 Authors such Alach, Gray, Betts (along with conditional optimists such as Cohen, Krepinevich and Metz) identify the incremental change/evolution combined and interactions between many systems to be a key consideration.

%Ochs finds that in multinational units, differences in culture and interpretation of Auftragstaktik often blunt the effectiveness of mission command, despite digital tools that could otherwise enhance decentralisation” (Oc
% get Gray and Kaldor's articles


\section{Hidden Tensions and Authorial Silences}


This section explores hidden tensions and authorial silences in RMA debates. \textcite{HUNDLEY_1999}'s report elucidates authorial biases or silence. Stating that, ``the Gulf War showed, advances in technology can bring about dramatic changes in military operations''.  \textcite{BIDDLE_1996} contends that this hypothesis was wrong, citing human error which was suited to technological exploitation. The war was fought at a chosen time, on terrain ideally suited to AirLand Battle\index{AirLand Battle}. It is reminiscent of the trope that physicists seek to reduce the physical world to an idealised `spherical chicken in a vacuum'. Techno-optimists\index{Techno-optimist} often demonstrate cognitive dissonance by presenting so-called `discontinuous' advances as solely due to technology \parencite{HUNDLEY_1999,KREPINEVICH_1992,KREPINEVICH_1994,KREPINEVICH_1996,OWENS_1996,OWENS_2000,OWENS_2002,COHEN_1995,COHEN_1996}. Sceptics acknowledge the value of improved technology and present a more nuanced view \parencite{BETTS_1996,BROSE_2019,GRAY_2005,ALACH_2008A}. The central silence arguably refers to the nature versus character of war. For example \textcite{HUNDLEY_1999,KREPINEVICH_1992} speak of technology producing a change to the nature of war. However, given that war remains a human endeavour, how could technology change its nature? \index{Techno-sceptic}The sceptics assessment that organisational and intellectual change (human components) are required to realise technological advancement is borne out through history. For example, while the Wehrmacht had arguably inferior tanks to the Allies out the outbreak of war, it was their organisational and intellectual adaptations which proved decisive - at that time. Betts (1996) warns of the ``interactive nature'' of strategy, whereby should a true RMA present, it is unlikely to deliver lasting gains\nocite{BETTS_1996}. Indeed, the \index{Clausewitz!Fog of War}Clausewitzian fog of war may in-fact preclude the possibility outright. In 2018, Gray provided a similar warning, \nocite{GRAY_2005,GRAY_2018} where warfare's chaotic nature makes prediction impossible. A dilemma presents where, during conflict with a greater power, weaker nations may elect simply to avoid large-scale confrontation in favour asymmetric alternatives \parencite{JORD_2003}. U.S., actions in Afghanistan and Iraq\index{Afghanistan}\index{Iraq} remind us of Gray's (2005) ironic lament, ``Just when we found the answer, they changed the question''\nocite{GRAY_2005}. Hence, it is assessed that techno-optimists present a narrow view of the importance of technological advancement. This thesis accepts \textcite{ALACH_2008A}'s citing of Occam's razor. An RMA must manifest as revolutionary success on the battlefield and not simply in the minds or on paper. For the Irish Defence Forces, scepticism towards technological determinism suggest that doctrine and organisation matter more than technology.

\section{Mission Command under Surveillance: What Authors Do Not Admit}
Cohen's 1996 paper frames mission command against generals ``perch[ed] cybernetically'' \index{Perched Cybernetically}to tactical commanders. His metaphor illustrates a temptation of intervention. What is absent however is an acknowledgment of the organisational imperatives quietly urging senior leaders to intervene. While \textcite{KRULAK_1999} ``strategic corporal'' \index{Strategic Corporal}was coined in 1999, GWoT's\index{GWoT} legal imperatives pushed Legads and senior commanders into the tactical space. While commanders such as Guderian's practice of mission command put them on the front lines, their interventions weren't about micromanagement. Conversely, technology and frankly legal accountability in an age where everything is documented in video have resulted in commanders cynically focused on self preservation rather than mission accomplishment. Betts' (1996) paper underscores the tendancy of humans to misuse technology. Owens' (2002) article speaks of challenges to command via structures and culture\index{Culture}\index{Structures}. Stovepiping, duplication of efforts and resisting jointness hinder the RMA be believes to be occurring. The literature therefore hints at but rarely foregrounds an uncomfortable truth. Mission command in the digital age is less a doctrine than a fragile climate, constantly threatened by cultural habits and political pressure.

\section{Organisational Adaptation: Evolutionary Rhetoric and Revolutionary Silences}
``The Amaerican military loves organizational tradition at the same time that it hungers for technological progress'' \parencite{BETTS_1996}. Krepinevich (1992; 1994) bridges optimist and sceptic camps\index{Techno-sceptic}. While clearly a techno-optimist, he nonetheless acknowledges that organisational adaptation is decisive. He credits must of the Gulf War's success to organisational innovation  \parencite[p.~8]{KREPINEVICH_1992}. His relative silence on the Gulf War is noteworthy. Writing after a war which many techno-optimists hailed as revolutionary, he hedges and merely suggests that the revolution was beginning. This contrasts with Owen's straightforward statements accepting the RMA's existence \parencite{OWENS_2002}.
 
The silence likely lies in institutional loyalty. As a Pentagon insider, Krepinevich could hardly dismiss the campaign’s significance outright. Similarly, Metz (2000) stresses incremental change, but does so in a way that avoids alienating U.S. reform constituencies \parencite{METZ_2000}. Both are \textit{really saying} that militaries must appear open to revolution, while privately recognising the inertia that makes evolution the safer description.


The Stimson Center's 2015 report highlights how bureaucracies inculcated drone technology. Instead of embracing transformative change, it suggests that organisations integrated UAS\index{UAS} through routine processes. Is underscores this bureaucratic absorption. Its seven recommendations on \parencite[p.~9]{STIMSON_2015} hint at an unspoken truth. Bureaucracies frequently resist innovation to protect existing structures. The unsaid conclusion is that innovation is sometimes adopted merely to preserve established hierarchies. Betts' (1996) assessment is less cynical, recognising that conservative evolution is sensible. While this observation is most applicable to liberal democracies, generalisation may be limited. For the Irish Army, it is clear that organisational adaptation is central to any technological advancement.


\section{Character of Warfare: Revolution Talk, Evolution in Practice}
The RMA debate turns on a central tension: revolution in the character of war or continued evolution. The Clausewitzian conception of the nature of war has stood the test of time \parencite{CLAUSEWITZ_1984}\index{Clausewitz!Nature of War}. Gray, a prolific contemporary strategist, concurs that warfare's nature is human in its conduct and political in its aims \parencite{GRAY_2018}. Hence, it is consistent. This longevity counsels caution toward claims of change in war's nature. The techno-optimist RMA literature demonstrates rhetorical excess. This is perhaps best evidenced by claims regarding changes to the nature of war \textcite{HUNDLEY_1999,KREPINEVICH_1992,JORD_2003}. Alach (2008) is explicit in his assessment of evolution rather than revolution in the \textit{character} of warfare.

Operation Desert Storm is emblematic. For optimists, it was seen as empirical evidence that an RMA was underway. The political and human nature of the conflict persisted. Suggestions regarding a revolution in the nature of war were overstated. By contrast, the character of war is evolutionary. Techno-optimists rarely concede Desert Storm was oversold. This silence reflects an incentive to build momentum for technological advancement. It is somewhat ironic that proponents of technological prominence rely on rhetoric than evidence. 

\textcite{HUSAIN_2021}, pushes the debate forward by arguing that drone swarms may replicate the RCP\index{Relative Combat Power} of large forces. Once again, the silences are instructive. He places emphasis on technological potential while downplaying associated vulnerabilities. Such vulnerabilities are best evidenced by the Russo-Ukraine War. Electronic warfare, attrition and bandwidth considerations hamper the use of technology. The concept of AI swarms may appear revolutionary (and indeed compelling), but only against an unprepared opponent. Adaptation by the adversary attenuates revolution. Much of this debate is U.S. and NATO centric, with the Chinese lens absent at present.


\section{Tensions within Ethics and Governance}
This section considers ethical literature, which also contains implicit tensions. \textcite{SINGER_2010,SPARROW_2016} argue that autonomy is a serious moral hazard.\index{Ethics}\index{Legal Considerations} Their tone can be interpreted as advocacy against autonomy, which is presented as scholarship. This is an implicit tension, whereby authors present themselves as analysts while acting as advocates. \textcite{SAUER_2020,ALTMANN_2017} emphasise the importance of ethical governance - through Article 36 \index{Article 36}reviews. However, they are silent on the weaknesses of legal enforcement mechanisms. This is of particular relevance to moral dilemmas arising from immoral actors' use of AI in kill-chains\index{Kill Chains}. There is a tacit acceptance that  power politics will drive its use. Said differently, while the law can frame the debate, it appears that deterrence and coercion decide outcomes. This is a structural realist viewpoint. Voices in support of ethical machines are less prevalent. For example, \textcite{ARKIN_2008} is optimistic that machines could surpass humans in ethical conduct. Arkin's work implied human programming, perhaps accounting for his optimism. However it pre-dates large language models, whose learning is genuinely autonomous. Hence, this thesis does not share his optimism. For the Defence Forces, ethical debates must be updated to consider genuine machine learning, rather than being assumed from legacy arguments.

\section{The Rhetoric-Reality Gap}
Taken together, these readings reveal commonalities. Optimists (Cohen, Owens, Metz, Krepinevich, Husain, Hundley) speak of revolution, though all inherently hedge by requiring organisational change also. Sceptics (Betts, Gray, Alach) dismiss revolution in favour of incremental evolution. Perhaps the sceptics do not go far enough. Given the human nature of war, perhaps true discontinuities are simply impossible. Institutional reports such as \textcite{STIMSON_2015} split the difference. Evolution is recognised, though masked in managerial vocabulary. 

For the Irish Army, the tensions are instructive. One cannot take authors at face value. Instead, one has to ask what they omitting from consideration, or what they can not say due to external factors - such as organisational politics. Advocates of mission command may tacitly concede the inevitability of micromanagement. Techno-optimists may know that significant advances in capability will disperse to others and become a `new normal' amongst their peers. Ethicists may understand that governance almost always lags behind competition. The silences are equally informative and worthy of consideration. 



%Techno-optimists (Krepinevich, Cohen, Owens and Metz) Sceptics (Betts, Gray, Kaldor and van Creveld) the author does not share Arkin's optimism. 


\paragraph{Summary and Conclusion}\mbox{}\\
Krepinevich stands out as a bridge across the literature. He shared the techno-optimist belief that information, precision strike and new organisational forms could drive a revolution. Yet he also stressed that bureaucratic resistance and doctrinal inertia often prevent militaries from realising this potential. His recognition that invention without reorganisation fails, links early RMA theory to later institutional learning perspectives. In this sense, he links the optimism of Metz, Cohen and Owens with the caution of Betts, Alach and Gray. His insights anticipate contemporary debates on AI, autonomy and dual-use technology. This dual perspective is evident across his writings—from his broad historical analysis of past military revolutions \parencite{KREPINEVICH_1994} to his tempered assessment of Desert Storm’s limited revolutionary character \parencite{KREPINEVICH_1996}.







