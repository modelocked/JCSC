\chapter{Importance of the Research}
% ~200 words: why this matters (theoretical, practical, personal/DF relevance).

The implications of uncrewed systems and AI to the Irish Army are interesting to consider. Uncrewed systems comprise a clear threat to our soldiers. The narrowing of the technology gap has created vulnerabilities  from hybrid and irregular forces \parencite{HUSAIN_2021}\index{Technlogy Gap!bridging}. Conversely, they could provide considerations for an Ireland which is seeking to invest more in defence. 

%Boldly adopting and leveraging uncrewed and AI-enabled systems could enable the Army to bypass traditional, slow, and costly development and procurement processes - just like submarine during the early 20\textsuperscript{th}~century\parencite{COHEN_1996}. %On 15 February 2023, the Deputy Chief of Staff (Operations) published the ``Army Roadmap for Uncrewed and Optionally Crewed Vehicles (UOS\gls{uos})" \parencite{DCOS_UOS_2024}.  


Whether \gls{ugs} and \gls{uas}\index{UAS|see{drones}} \footnote{Also known as "drones"\index{Drones|see{UAS \& UGS}}.} constitute a revolution or evolution of warfare, their proliferation warrants attention and careful consideration. UAS in particular have been seen to lower the `barrier to entry' to technology which, to-date, was the exclusive purview of technologically advanced conventional militaries. Similarly, they are closing previous technology gaps between unconventional, hybrid and conventional forces.

Drones, once novel, are now routine instruments of warfare. The Second Nagorno-Karabakh War (2020) and the Russo-Ukraine War (2022 to present) are particularly noteworthy in that regard. 



Historical precedent illustrates the influence of technology on command, such as through the invention of the telegraph \parencite{COHEN_1996}. A famous case of this influence was of President Barack Obama supervising May 2011's Operation Neptune Spear\index{Barack Obama}\index{Operation Neptune Spear}. The use of AI shall create tension with the exercise of the doctrine of mission command\index{Mission Command!tension with AI}.

Research on the adoption of uncrewed systems and AI is therefore significant. It is disruptive not only for the conduct of warfare but for military organisation, doctrine and command.  
