\chapter{Implications for the Irish Defence Forces}
% ~100 words: leadership/management/doctrine/capability/IHL value to DF.

Can the history of warfare be disentangled from a history of human technology and innovation? Are the methods of war entirely separate from the tools by which it is waged? Said differently, what advantage if any is secured solely through technological advancement?

Advances in technology are transformational in a military context - interestingly discussed by Cohen's 1995 paper \nocite{COHEN_1995}. In military planning, there is a tool known as \gls{rcp} which is used as to broadly assess parity of forces prior to combat. It is interesting to consider what impact uncrewed systems have on RCP. It is suggested that drones may constitute both an opportunity to modernise and to close the RCP gap with other conventional forces. Boldly adopting and leveraging uncrewed and AI-enabled systems could enable the Irish Army to fast-track capability development. Opportunities exist to bypass traditional, slow and costly development \& procurement processes. This could be analogous to weaker naval nations leveraging the submarine during the early 20\textsuperscript{th}~century\parencite{COHEN_1996}.  In 2021 Husain\footnote{Cohen's 1996 paper provided similar analysis \nocite{COHEN_1996}.} \nocite{HUSAIN_2021}posited that the RCP of a drone swarm which is optimised via \gls{ai} could be comparable to a substantially larger conventional force. Specifically, that AI could coordinate a decisive concentration of force, while drones by their nature are particularly manoeuverable. Indeed, the corresponding structural and command-related impacts could be equally valuable considerations. As Crino and Dreby show, the disruptive effect of small drones is already evident in repeated strikes on critical infrastructure, underscoring how non-state and state actors alike can weaponise commercially available systems \parencite{CRINO_2020}. As reported by the EDA, a significant proportion of tactical reconnaissance and strikes are being conducted remotely \parencite{NICHOLESCU_2023}. These considerations are important influences on command, leadership, structure and culture.


%It is further suggested that drones constitute a threat which makes our soldiers increasingly vulnerable to hybrid and irregular forces. I.e., while there is an opportunity to significantly enhance the Irish Army's conventional RCP, the RCP of non state actors (who are the most direct threat to deployed Irish troops) is enhanced. This observation is supported by the Army's \gls{uos} Roadmap \parencite{DCOS_UOS_2024}. Hence, in parallel to the conventional considerations, there exists a requirement to mitigate risks posed to our soldiers during crisis management operations.

%On the conventional battlefield, the advantages gained following innovations in technology and techniques are being countered and negated in a matter of weeks \parencite{MARTINHO_2025,EDA_GCCEG_2025,MUELLER_2023,EDA_HERLIN_2025}.
%While UAS have received much of the attention in this regard, uncrewed ground systems cannot be ignored. One of the first documented deployments of an UGS was the Goliath Tank of the German Wehrmacht during the second world war. UGS have been less prevelant than their aerial cousins until quite recently. UGS are being commissioned and experimented with for myriad tasks, from kinetic action to casualty evacuation. 


%Can a modern section fight a ww2 platoon? etc.

%technological revolutoin requires a cultural change to implement

%operational reach, force protection adn cost effectiveness. 20hour loitering. shorten the kill chain. go to locaitons which are too dangerous to humans . ISR and strike simultaneous.

%strategic risks. escalation. may be percieved of less escalatory but when they access places that other forces that can't access can erode sovreignty (``denied areas") percieved as an act of war

%no silver bullets to CUAS. all sources agree

%hyper war and hte collapse of hte OODA loo (by AI)

%2020 ALPHA DOGFIGHT (DARPA AND AI)
