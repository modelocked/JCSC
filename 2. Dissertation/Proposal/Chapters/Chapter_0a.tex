Tier 1 – Core Articles / Reports (must include)

These are directly tied to your RQ and are article/report format:

Andrew F. Krepinevich (1992) – The Military-Technical Revolution: A Preliminary Assessment (CSBA report).

Andrew F. Krepinevich (1994) – Cavalry to Computer: The Pattern of Military Revolutions (The National Interest).

Andrew F. Krepinevich (1996) – Revolution in Warfare? Airpower in the Persian Gulf (Foreign Affairs).

Richard K. Betts (1996) – The Downside of the Cutting Edge (The National Interest).

Eliot A. Cohen (1996) – A Revolution in Warfare (Foreign Affairs).

Steven Metz (2000) – The Next Twist of the RMA (Parameters).

William A. Owens (2002) – The Once and Future Revolution in Military Affairs (Joint Force Quarterly).

Zachary J. Alach (2008) – The Revolution in Military Affairs: Slowing Military Change (Defence Studies / monograph).

Azhar Hussain (2021) – AI is Shaping the Future of War (PRISM).

Benjamin Jensen (2025) – Russia’s Massed Strikes: Strategy of Coercion by Salvo (CSIS Report).

Stacie L. Pettyjohn et al. (2024) – The Drones of War: Ukraine’s sUAS (RAND).

Jack Watling & Justin Bronk (2024) – Protecting the Force from Uncrewed Aerial Systems (RUSI).

Michael Raska & Richard Bitzinger (2023) – The AI Wave in Defence Innovation (Routledge edited volume but with article-style chapters; RSIS profiles available).

David Rassler (2015) – Remotely Piloted Innovation: Terrorism, Drones and Supportive Tech (Combating Terrorism Center, report).

Schaus & Johnson (2018) – UAS’ Influences on Conflict Escalation Dynamics (CSIS / JSTOR report).

Tier 2 – High Value Additions (articles/reports)

Good to deepen critique and add contemporary angles:

Christopher Brose (2019) – The New Revolution in Military Affairs (Foreign Affairs).

Steven Metz (2007) – Rethinking Insurgency: The Future of American Strategy (SSI monograph).

Alexandra Evans et al. (2025) – Lessons from the War in Ukraine for Space (with UAS sections) (RAND).

Stimson Center (2015) – Military Utility, National Security and Economics (report).

Tier 3 – Supplementary (books only if needed)

Bring these in only where articles are too thin:

Colin S. Gray (2005) – Another Bloody Century (book, but still the most cited sceptical text).

Paul Scharre (2018) – Army of None (book, widely cited, fills autonomy gap).

Dmitry Adamsky (2023) – Russian Way of Deterrence (book, but excerpts useful for mission command/centralisation).

Benjamin Jensen (2016) – Forging the Sword (book; doctrinal change, OODA loop).




Gray's 2018 book on strategy \parencite{GRAY_2018} notes ``the primary task of subordinate generals must be to organize and ocmmand the timel and approprite necessitites fo combat fo rthe purposes established by the most senior level of military command in their negotiation with civilian political authortyh''. This aligns with \parencite{COHEN_2002}'disucssion on the special friction between the military and civilian actors. Gray says that ``strategy is about the purposes of action while tactics are about actually performing the action in question''. ``no matter what the weapon technlologies wil lbe in the future decades, we know for certain that nothing fundamentaly important and positive in moral terms is going to change''.