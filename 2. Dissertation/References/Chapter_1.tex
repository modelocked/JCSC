\section*{Small States and Security Outcomes: Ireland and the Five Effects Framework}

\subsection*{Introduction}
Small states have long struggled with the dilemma of how to exert influence in a system dominated by larger powers. From Keohane’s seminal work on the “Lilliputians’ dilemmas” \parencite{KEOHANE_1969} to Thorhallsson’s analysis of state size within the EU \parencite{THORHALLSSON_2006}, the literature demonstrates that agency is not absent, but constrained. This essay addresses the research question: \textit{To what extent can small states influence international security outcomes through military and non-military means?} Using a five-effects framework—Niche Specialisation, Organisational Agility, Hybrid Leverage, Soft Power Synergy, and Legitimacy—the argument will evaluate both the enabling mechanisms and the limits on small-state influence. The hypothesis is that small states achieve recognition and marginal influence through these effects, but legitimacy is the centre of gravity that sustains or undermines all others.  \parencite{ROTHSTEIN_1966}

Chapter 2 will present reasons for optimism, drawing on institutionalists, constructivists, and those who stress innovation. Chapter 3 will provide the realist counterpoint, showing how structural constraints and power politics marginalise small-state strategies. Chapter 4 will ground these arguments in comparative cases (Ireland, Estonia, Finland, Denmark, Qatar). The conclusion will argue that legitimacy reconciles these tensions and is indispensable for sustaining influence.

\subsection*{Chapter 2: Reasons for Proposal}
\paragraph{Niche Specialisation.}  
Krepinevich’s analysis of military revolutions emphasises how technological or doctrinal innovation can offer small states opportunities to “steal a march” \parencite{KREPINEVICH_1994}. For Ireland, peacekeeping functions as a niche that has conferred disproportionate diplomatic recognition. Thorhallsson argues that size and alliances allow small states to carve specialised roles within larger structures \parencite{THORHALLSSON_2006}. Similarly, Keohane’s institutionalism shows how small states, embedded within rules-based orders, can magnify their voice through institutional niches \parencite{KEOHANE_1988}. The implication is that small states can convert limited resources into visibility, provided these roles are coherent and embedded.

\paragraph{Organisational Agility.}  
Metz suggests that adaptation is most likely after defeat, but small states may be more open to reform due to vulnerability and limited sunk costs \parencite{METZ_2007}. Estonia’s rapid investment in cyber defence, for example, illustrates how a small state can exploit agility to secure relevance. Constructivists such as Wendt contend that identity and culture shape institutions \parencite{WENDT_1999}; Ireland’s Defence Forces, though resource-constrained, have sought to embed reforms in organisational culture through initiatives such as the High Level Action Plan \parencite{HLAP_2022}. Organisational agility thus represents an avenue for credibility, signalling seriousness to partners despite material weakness.

\paragraph{Hybrid Leverage.}  
Farrell and Newman’s concept of “weaponised interdependence” demonstrates how global networks can be sources of coercion and leverage \parencite{FARRELL_2019}. For small states, network position matters more than sheer size. Ireland, hosting key data infrastructure, gains relevance by shaping EU regulatory frameworks such as GDPR. Nye’s work on power illustrates how hybrid and soft elements converge \parencite{NYE_2008}. Constructivists highlight that narratives and norms can amplify hybrid tactics. For small states, hybrid leverage is less about coercion than about regulatory entrepreneurship.

\paragraph{Soft Power Synergy.}  
Nye’s soft power thesis highlights credibility and attraction as sources of influence \parencite{NYE_2008}. Tonra argues that Europeanisation has amplified Ireland’s foreign policy by embedding its diplomacy within EU structures \parencite{TONRA_1999}. Bailes and Thorhallsson show that small states instrumentalise the EU to project voice \parencite{BAILES_2012}. Constructivists argue that identity and reputation shape international outcomes \parencite{BAILES_2013}. Neutrality, peacekeeping, and development aid constitute elements of Ireland’s soft power, which gain synergy when framed within EU and UN legitimacy.

\paragraph{Legitimacy.}  
Gray emphasises that strategy is ultimately about political consequences, not operational brilliance \parencite{GRAY_2018}. Cohen insists that legitimacy is forged in civil–military dialogue rather than harmony \parencite{COHEN_2002}. Thorhallsson and Keohane show that institutions confer legitimacy on small states \parencite{THORHALLSSON_2006, KEOHANE_1988}. Constructivists such as Wendt highlight that legitimacy is socially constructed, shaping identities and roles \parencite{WENDT_1999}. For Ireland, neutrality remains a core legitimacy marker, anchoring all other effects.

\subsection*{Chapter 3: Reasons Against Proposal}
\paragraph{Niche Specialisation.}  
Gray warns against overstating novelty: continuity and structural constraints dominate \parencite{GRAY_2005}. Waltz argues that small states adapt within structures but cannot shape systemic outcomes \parencite{WALTZ_1979}. Mearsheimer dismisses the transformative potential of institutions, insisting that small states’ niches are marginal unless aligned with great-power interests \parencite{MEARSHEIMER_1994}. Thus, niches risk tokenism rather than influence.

\paragraph{Organisational Agility.}  
Murray contends that culture and doctrine resist change, producing inertia even after shocks \parencite{MURRAY_2001}. Mearsheimer’s offensive realism stresses that great-power competition sets hard limits on small-state reforms \parencite{MEARSHEIMER_2019}. While Ireland may seek reform through the HLAP, resource crises and political caution mean agility is often rhetorical. Organisational adaptation cannot offset structural weakness.

\paragraph{Hybrid Leverage.}  
Betts highlights risks of technological misuse and overreach \parencite{BETTS_1996}. Realists argue that hybrid moves by small states may provoke escalation or retaliation from great powers. Farrell and Newman note that weaponised interdependence cuts both ways: network exposure creates vulnerabilities as well as leverage \parencite{FARRELL_2019}. For Ireland, reliance on transatlantic data flows demonstrates exposure more than influence.

\paragraph{Soft Power Synergy.}  
Realists insist that soft power lacks weight compared to material force \parencite{WALTZ_1979, MEARSHEIMER_2019}. Gray warns that soft power is fragile and collapses under pressure from hard power \parencite{GRAY_2005}. While Ireland leverages neutrality and peacekeeping, critics of neutrality argue that ambiguity erodes coherence. Soft power without delivery risks becoming hollow symbolism.

\paragraph{Legitimacy.}  
Mearsheimer contends that liberal orders fail because nationalism and power politics dominate \parencite{MEARSHEIMER_2019}. Legitimacy, in his view, is rhetorical cover for great-power interests. Critics of Irish neutrality argue that it is increasingly incoherent within EU security commitments \parencite{WHITE_2015}. Gray warns that legitimacy without material capacity is unstable \parencite{GRAY_2018}. Thus, legitimacy is fragile if not backed by coherent resources and policy.

\subsection*{Chapter 4: Comparative Cases}
\paragraph{Ireland.}  
Ireland illustrates both the promise and the limits of small-state influence. Europeanisation has amplified Ireland’s diplomatic voice \parencite{TONRA_1999}, but neutrality is contested and organisational reform faces resource limits \parencite{HLAP_2022, DOD_2025}. Peacekeeping offers a niche, but it depends on sustaining credibility through delivery.

\paragraph{Estonia.}  
Estonia demonstrates organisational agility through its cyber defence niche. By investing early, it gained credibility within NATO and the EU. Yet, its influence is contingent on great-power support and institutional frameworks, not independent capacity.

\paragraph{Finland.}  
Finland combines hard military investment with EU/NATO embeddedness, showing how small states leverage legitimacy and credibility through resilience. Neutrality has been redefined, demonstrating legitimacy’s malleability.

\paragraph{Denmark.}  
Denmark exemplifies selective opt-outs: it maximises influence through EU/NATO engagement but limits commitments where domestic legitimacy is contested. This illustrates agility but also the fragility of soft power when domestic consensus is absent.

\paragraph{Qatar.}  
Qatar shows how wealth can fund niche specialisation (Al Jazeera, mediation), hybrid leverage (energy markets), and soft power. Yet its legitimacy remains fragile, contingent on perceptions of alignment with larger powers. The case demonstrates the universality of the five effects.

\subsection*{Conclusion}
Small states cannot unilaterally shape international security outcomes. However, through niche roles, institutional agility, hybrid leverage, soft power, and legitimacy, they can exert influence disproportionate to their size. The critical synthesis of institutionalist, constructivist, and realist perspectives shows that each effect is conditional. Niches risk tokenism without coherence. Agility is necessary but constrained by resources. Hybrid leverage exposes as much as it empowers. Soft power amplifies but is fragile under pressure. Only legitimacy provides an enduring anchor: it reconciles roles, sustains credibility, and ensures that small states are recognised as relevant actors. For Ireland, legitimacy rooted in neutrality, peacekeeping, and EU/UN embedding remains its centre of gravity. Influence is therefore conditional, relational, and legitimacy-dependent.

GPT Make sure that you Use this qUOTE FROM \parencite{THORHALLSSON_2006} ``ambitious political leaders may form a notion of the size of their state and its capacity for international action based on distinctive national
history or myths''. Neorealists say that if you ensure that your objectives do not exceed your capabilities a tiny state can do extraordinarily well on the international stage. I.e., for any state, it's crucial that an accurate view of your position is taken. Thorhallsson's model to analyse a state's power based on six categories is worth noting. Hence, should Ireland or other small states be able to identify an edge where they could `steal the march', (such as through Estonia's niche specialisation in cyber warfare), they can increase their influence international security outcomes - becoming the `system-affecting' states which Keohane spoke of. Indeed, the idea of a small stat's influence being linked to its systemic role perception is supported by Keohane's 1969 paper \parencite{KEOHANE_1969}. This adds nuances to realists such as Waltz who dismiss small states' influence, deeming them to be passive actors \parencite{WALTZ_1979}. 

\parencite{WALTZ_1979} speaks of the interdependence and integration of states. He speaks of states being able to specialise where they do not fear vulnerability resulting from increased interdependence. This clearly cannot happen in an anarchic international system, because there is no higher order to protect states or adjudicate disputes. Simply put, in anarchic systems, power dominates. It stands to reason that realpolitik would likey dominate in such circumstances. Considering the implications on small powers, it seems that the influence of the small power first falls from its system. Small powers which find themselves in a anarchic region are likely to have their influence diminished. This is because realpolitik dominates. Consider Eamonn de Valera's Irish Free State during the Second World War. The maxim, when elephants are fighting those around them get trampled comes to mind. He realised that when the great powers of the world were at war (chief among which from his perspective was the UK, which only recently acceeded to the free state), the Irish had near-zero influence on the international order. He was acutely aware that sovreignty was vulnerable since the UK could see the free state as a threat from which an attack could be launched. He had to acquiesce and permit the so-called ``Treaty Ports'' to remain as UK bases. Furthermore, discussions regarding neutrality at the time are of note. \parencite{FANNING_2015} notes that de Valera felt strongly foreign policy was fully the purview of the government. Hence, while neutrality was a realpolitik/neorealistic decision to retain sovreignty, it didn't constitute his stance on what Irish foreign policies / culture should be in future. I.e., that this decision could be changed as required. (It is interesting to note that many see this realpolitik decision based on the prevailing international relatoins of the time now as a cultural and uniquely-irish matter). Had de Valera not understood his weakness and poorly played his hand, it is very likely that the UK could have reoccupied Ireland in full. ``The Second World War, then, affirmed rather than altered the order of priorities in de Valera's Ireland: independence first, unity a poor second.'' ``De Valera's dignified rebuke to Churchill - pointing out that if Britain's necessity were admitted as a moral code 'no small nation adjoining a great power could ever hope to ... go its way in peace' ''. This, in Fanning's estimateion ``caught the public imagination and irrevocably identified neutrality with Irish independence''. Then as an example of de Valera's realpolitik, ``There were two reasons, both rooted in raison d'etat, why it was necessary that Irish support for the Allies had to remain buried in secrecy. First, because any public departure from neutrality would expose Irish cities to German air-raids or other retaliation. Second, because for de Valera to have publicly abandoned neutrality would have outraged the philosophy of foreign policy that he himself had created and cultivated so succcessfully since 1920. There was an intrinsic meric, moreover, in 'the secrecy with which d eValera shrouded his ultimate intentions and wishes', because, as Desmond Williams has pointed out, it enabled both belligerent blocs to interpret his statments 'according to their desireds and in a sense favourable to themselves. De Valera in fact appears to never to have told anyone, even in his cabinet, everything that was on his mind' ''

 \parencite{GRAY_2018} make note of Gray's principle 14, "intellgence and deception are permenant features of stragegy". link this to \parencite{FANNING_2015}'s observations of de Valera where only he knew his own thought. He would not share his thoughts and ultimately his strategy with anyone.                                                                                                                                    

 \textcite{DUMAN_2025} The authors suggests peacekeeping’s obsolescence, pushing Ireland toward structural-realist adaptation.
 
\parencite{THORHALLSSON_2006} ``The conceptual framework emphasizes the
importance of domestic and international actors’ assessments of a state’s
action competence and vulnerability, internally and externally''. This emphasises the importance of a small state's legitimacy. 

\parencite{KEOHANE_1969,ROTHSTEIN_1966}  The concept of alliances is interesting for small powers. This is a function of the state's geography, since regionalism and one's immediate neighbours shall be the dominant factor for an alliance) - though can be extended through involvement in international organisations.    Rothstein's assessed that should a small power seek an alliance, it is best to be multilateral. He warns against the unequal partnership with a great-power. This is in agreement with Waltz' (1969) analysis, where a  [basically power dominates. the smaller power is overlooked and can be dragged into conflict at odds with its own interests.] This is noteworthy in the context of the Free State during the Second World War. As reported by \parencite{FANNING_2015}, the UK promised a 32 county Ireland should the Free State enter the war on the side of hte UK. Eamon de Valera prioritised independence of a 26 county coutry over the possiblity of future reunification. As a realist, de Valera would have been acutely aware of hte power dynamics, seeing that Ireland's leverage would wither after the war, while it had all to lose. Hence, unequal alliances are best avoided in order to advance hte influence of a small power. From the perspective of the small power, Rothstein would argue against multilateral alliances involving large powers are disadvantageous. It is interesting therefore to consider NATO. [he's probably right....]. It is further interesting to consider Finland's recent accession to NATO. Finland has a history of taking their sovreignty and defence with great care. Recent Russian aggression has clearly tipped the balance in favorur of alliance - which is precisely where Rothstein predicts that unequal alliances prove worthy of consideration.
                                                                                                                              
Was Saddam's Iraq or Assad's Syria a small power? Is Israel a small power? This needs analysis using Thorhallsson's paper.  \parencite{KEOHANE_1988} ``Contemporary world politics is a matter of wealth and poverty, life and death. The''

\section{\parencite{CARROLL_2023}} "like many small nations in a world of emerging conventional and hybrid threats, ireland is facing these same challenges, thes same strategic choices. However, in some areas, there is a disconnect between the ways and means of IRish defence policy, and indeed, th eends. Solving this disconnect requires the evaluation of the threat environment facing Ireland and the Defence Forces. [...] Cleary highlights that over the course of forty years,Ireland's Defence Forces have only dinimished , while emerging threats and operational committments have only increased, and that doing more with less has consequence for Irish seciryt. In Dan Ayiotis' article therein ``Irish Military Neutrality: A Historical Perspective'' he states ``Maintaining a policy of neutrality and having a military are not mutually exclusive; in fact, the opposite is true. The stronger the military, the more ``neutral'' a state can afford to be. [...] The real origins of Irish neutrality lie in realpolitik and the pragmatism of a small, newly independent, and often cash-strpped state. This pragmatism can be broadly traced through three phases: neutrality as necessity fro mthe end of hte civil war in 1923 untnil 1939, driven by the obligations of hte Anglo Orish Treay and Commonwealth membershio; neutrality as expediency, from the outbreak of hte second world war in 1939 until joinin the United Nations in 1955, as a small nation navigating dire straits as great world powers and ideologies fought for dominance; and neutrality as convenience from 1955 until the present, as a means of exercising an influence on the international stage, espousing multilateralism and enjoying its benefits while retrainign a get out of jail free card shoudl the stat efeel overly committed by multilateralism's requirements. In examining thse three phases, on overall picture emerges of neutrality whih as always meanst whatever Ireland needed ot be at the time ''GPT include Dan's observation in the essay. It correlates exactly with my assessment. It correlates with observations regarding de Valera. Whether it remains as realpolitik or as neorealism is to be determined. 
 
\parencite{FLYNN_2019} states that ``. If small states want to be relevant and influential in peacekeeping,  they need to figure out how to offer force packages that cross a threshold well beyond the  tokenistic or niche nor have them burdened by excessive national political caveats that limit their operational flexibility. This implies land units of at least reinforced company size, and  credible aerial and maritime assets as well''. This is of note in the context of Ireland's shrinking peacekeeping footprint - indeed the possibility that peacekeeping is a dead. Given the impending dissolution of UNSCR 1701, the withdrawal from UNDOF and the withdrawal from Africa (save for a single officer in Uganda), Ireland's peacekeeping footprint has withered significantly from the first decade of the century. This is of note in the context of Ireland identifying its peacekeeping as a legitimate and welcome projection of influence and execution of foreign policy. Does this sound the death knell for a substantial portion of Ireland's international legitimacy? Does it require a revision of Ireland's ways and means of international influence? Since evolving from a poor small state to a wealthy one (following the Celtic Tiger), Ireland appears to have developed notions of her own self importance on the world stage. In that context, Ireland has long traded on the legitimacy derived through perceived neutrality and virtue through peacekeeping. Ireland's political policies have evolved from the pragmatic realpolitik of Eamon de Valera to the structural-realism of today. Indeed, Ireland's structural realism is concealed behind performative liberalism. Staying rich safe in nobel clothes. The public actually believes. Underneath is a state which knows it can't defend itself so it it does the virtuous tokenism of ration-packs and old armoured vehicles. Virtue signalling pragmatism.
                                                                                                                              
\parencite{HIRST_2010} descriptions of Israeli conduct towards Lebanon clearly paint a reaslist perspective, where power was wielded primarily by the military. He states that ``a nation born of the sword was forever going ot live by it''. Indeed, the suggestion by Ben Gurion that the foreign ministry's job is to explain the actions of the defence ministry to the west of hte world underpins this. \parencite[p. 53]{HIRST_2010}. It is of note that such an aggressive perspective is at odds with considerations that Israel is a small state (read small power). As evidenced by the US State Department's (2025) fact sheet \nocite{STATEDEPT_2025}, ``steadfast support for Israel's security has been a cornerstone of American foreign policy for every U.S. Administration since the presidency of Harry S. Truman. The United States and Israel have signed multiple bilateral defense cooperation agreements, to include: a Mutual Defense Assistance Agreement (1952); a General Security of Information Agreement (1982); a Mutual Logistics Support Agreement (1991); and a Status of Forces Agreement (1994)". It is worth considering whether this would attract Rothstein's warning against an alliance with a single great power. While Israel is clearly not a "great power" compared to the U.S., is is possibly not a small state either.
                                                                                                                              
\parencite{ROTHSTEIN_1966} notes that ``Intervention was also bound to cause trouble between the rest of the nonaligned states and the two superpowers; and  as their votes in the UN and their political support on a wide range of issues became increasingly more important than their military contribution, that too became a consideration''. This highlights another aspect of  Thorhallsson's ``perceptual size'' \nocite{THORHALLSSON_2006}, where a small state may have elevated influence due to their votes within an international institution such as the UN. It is of note therefore that the increasing logjammed nature of hte UNSCR diminishes the influence of small states who leverage that multilateral organisation. Data from \parencite{HELLMUELLER_2024} demonstrate that the \textit{establishment of new UN peace operations has collapsed since the 1990s}. During the immediate post-Cold War decade, the Security Council authorised on average five to six new missions per year (1991--1995). This rate declined to roughly two to three new missions per year through the 2000s.Since 2012, however, the creation of new peacekeeping operations (PKOs) has virtually ceased, with the few new mandates established being overwhelmingly \textit{special political missions} (SPMs) or special envoys/advisors rather than robust peacekeeping deployments. The empirical trend is stark: whereas the 1990s represented the high-water mark of UN activism, the 2010s and 2020s have seen the Security Council \textit{almost entirely abandon the authorisation of new PKOs}.This pattern provides strong evidence for the contention that peacekeeping is in structural decline.For small states such as Ireland, which have historically derived disproportionate legitimacy and influence from participation in UN peacekeeping, the erosion of new mandates signals the waning of a core foreign-policy instrument. Where once Ireland could reliably translate its neutrality and peacekeeping contributions into international credibility, the paralysis of the Security Council now undermines this niche. In strategic terms, the \textit{ends} (influence and legitimacy) no longer align with the available \textit{ways and means} (credible peacekeeping deployments). This decline therefore marks not only the twilight of UN peacekeeping but also the need for small states to reconsider alternative avenues for sustaining international relevance. 
                                                                                                                              
GPT: KEEP THIS PARAGRAPH IN FULL Flynn (2019)\nocite{FLYNN_2019} cautions that small states must avoid tokenism to remain relevant in peacekeeping, requiring reinforced company-level contributions with credible air and maritime assets to cross the threshold of meaningful influence. Ireland’s shrinking footprint—complete withdrawal from UNDOF in 2024, near-total exit from African missions (save for minimal support in Uganda), and the planned phaseout of its UNIFIL deployment under UNSCR 1701 by 2027—suggests it has fallen below this threshold, with only around 300 troops remaining in Lebanon as of August 2025 amid a final mandate extension to December 2026. The realist implication is stark: peacekeeping, once Ireland’s primary niche and source of legitimacy, is entering twilight, exacerbated by UN Security Council dysfunction—only 25 resolutions adopted by September 26, 2025, with no new peacekeeping missions since 2014 \parencite{HELLMUELLER_2024}. Quinn (2018) \nocite{QUINN_2018} frames peacekeeping as an expression of Irish identity and a liberal foreign-policy tool, but this is increasingly untenable in a multipolar world where consensus wanes. As de Valera’s wartime neutrality revealed \parencite{FANNING_2015,AYIOTIS_2023}, Ireland’s posture has always been realist, a veil of moralism masking strategic hedging. Today, performative liberalism sustains the illusion of influence, yet beneath it lies structural realism and material weakness, with defense spending stagnant at 0.3 percent of GDP. My recent experience in Gaza aid debates underscores this tension: public figures advocated symbolic gestures like escorting aid ships, yet Israel, a hard realist state, remains unmoved, highlighting that such actions satisfy domestic expectations but lack strategic effect. This risks collapsing Ireland’s legitimacy into symbolic performance. Quinn’s (2018) view of values-based peacekeeping overlooks this realist foundation; I assess, per Waltz (1979), \nocite{WALTZ_1979}that Ireland’s engagement is structurally realist, necessitating a pivot—e.g., to EU cyber niches—to sustain influence amid failing UN frameworks

As De Valera’s wartime neutrality revealed \parencite{FANNING_2015,AYIOTIS_2023}, Ireland’s posture has always been realist: a veil of moralism masking strategic hedging. Today, performative liberalism sustains the illusion of influence, but beneath it lies structural realism and material weakness, with defense spending stagnant at around 0.3 percent of GDP. My own experience of recent debates about Gaza illustrates this tension: while public figures argued for symbolic gestures such as escorting aid ships or deploying civilian observers, the reality is that Israel, as a hard realist state, is unmoved by such symbolism. Gestures may satisfy domestic expectations, but they do not alter outcomes. This underscores the risk that Ireland’s legitimacy, long derived from peacekeeping, is collapsing into symbolic performance without strategic effect. Quinn reports on perceptions regarding Ireland’s participation in peacekeeping as values-based rather than transactional, but I suggest that his analysis underplays the realist and neorealist origins of Ireland’s participation. Drawing on Waltz’s structural realism \parencite{WALTZ_1979}, Ireland’s international engagement is best understood not as liberal altruism but as a structurally realist strategy of survival.

GPT I want to use this paragrpah closely. \nocite{COTTEY_2022}Ireland’s security posture is best interpreted through a neo-realist lens of hedging. COTTEY (2022) underscores that despite the systemic shock of the Ukraine war, Irish national security continues to rest on very low defence spending, limited combat capability, and only cautious EU engagement. This continuity reflects not idealist neutrality but a pragmatic calculation: small states cannot afford unilateralism and therefore hedge by balancing autonomy with tacit alignment to stronger powers. \parencite{FANNING_2015,AYIOTIS_2023} demonstrate that this logic is longstanding—since de Valera’s realist manoeuvres in WWII, Ireland has relied upon covert security dependence on Britain, cloaked in the language of neutrality. While COTTEY does not explicitly frame this as hedging, the evidence of structural free-riding and reliance on great power guarantees illustrates precisely that dynamic. The limit is that neutrality remains politically potent at home, constraining acknowledgement of dependence; the implication is that Irish policy is best understood as a form of security hedging, combining symbolic neutrality with material reliance on others.

\nocite{FLEMING_2015} Fleming shows that even before de Valera, Ireland’s neutrality was less about idealism and more about hedging against Britain. When synthesised with \parencite{FANNING_2015,AYIOTIS_2023}, the pattern becomes clear: Irish foreign policy has consistently involved realist calculation masked by symbolic neutrality. Together with COTTEY (2022), Fleming provides historical evidence that Ireland’s posture is best described as small-state hedging—projecting autonomy while relying on the great power next door.

\parencite{FANNING_2015,AYIOTIS_2023} elucidate that since de Valera's initial realist moves, Ireland has tacitly been involved in an alliance with the Great Power that is the United Kingdom.  This is clear evidence her our hedging.      

Introduction (Chapter 1)
For Colin Gray, strategy is about political ends, not reach or effects. Reach belongs to tactics. Operations are the sequencing of tactics. Strategy exists only when means are aligned to political purpose. From this view, coercive, controlling, or collaborative approaches describe relationships of power, not technical reach. Small states, lacking hard power, can only pursue the collaborative.

Body (For/Against/Discussion)
Collaboration can still matter politically: small states gain visibility and legitimacy by aligning their limited means to multilateral ends. However, the absence of coercive or controlling options reveals a structural ceiling. If strategy is only about political ends, small states face a hard limit. They may contribute tactically or organise operationally, but they cannot set the strategic terms of international security.

For the proposal (collaboration as influence)
Point. Small states can exercise influence through collaborative strategies.
Evidence. Ireland’s long participation in UN peacekeeping shows how modest forces can gain legitimacy and visibility.
Explain. This matters because collaboration aligns limited means to political ends, which Gray argues is the essence of strategy.
Limit. Such influence depends on multilateral recognition and is easily marginalised.
Consequent. Collaboration offers visibility but not decisive power.

Against the proposal (hard ceiling on influence)
Point. Small states cannot pursue coercive or controlling strategies.
Evidence. Without military mass or economic weight, they cannot compel outcomes or dictate terms.
Explain. For Gray, strategy is political ends; tactics and operations deliver reach and effects. Small states cannot translate these into control.
Limit. Even strong niche roles remain dependent on larger actors.
Consequent. Small states face a structural ceiling on their strategic influence.

Synthesis signpost
Taken together, these points show that small states can influence only through collaboration. This influence matters, but it remains conditional and secondary to the power of larger actors.
                                                                                                                                                  