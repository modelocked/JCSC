\chapter{DSS}
% Approach (qual/quant/mixed), cases + justification, data & analysis,
% limits, ethics. Social-science or humanities angle as per guideline.

Reflective Journal – DSS Module

Tuesday – Day 2

Day 1, I came home quite disheartened, feeling that we were, once again, just learning information for the sake of it, not really, you know, let’s do a master’s for the sake of it.

Today then, we were talking, we had Rory Finnegan primarily, and somebody else then from DCU talking about Realpolitik and the realist and liberal approaches and the neoliberalism, and, sorry, neorealism, excuse me, not neoliberalism, and I thought it was interesting, but it wasn’t going any place for me.

And then on the way home, I was talking with Brian and Liam, and I was saying that I was reading Colin Gray’s 2018 book on policy, sorry, on strategy, and I think I was saying that I was understanding that strategy is about consequences, political consequences. Operations are about the movement of resources and whatnot to achieve what is done at the tactical level. The tactical level is where the killing is done, and that can be anywhere by any amount of forces, but ultimately, operations are about putting those people into the position where the killing can happen at the tactical level.

And then he was talking about ends, ways and means, and I think ends are the political piece, that’s the policy. The ways are, I think, the strategy to achieve those, which are long-term, you might never conceive them, and the means are the tactics, the actual tactical outputs that you’re having.

Then I was thinking that, I still was thinking that, and all of this is kind of nonsense really, that it’s going nowhere, what’s the bloody point of listening to it? And then just reading for something else, I was reading Colin Gray’s book, sorry, I was reading Eliot Cohen’s 2002 paper, Eliot Cohen’s 2002 paper called, I can’t remember exactly what it’s called, but any of these papers on about the friction, the friction between the civilians and the military, and that there has to be friction in your job, and he mentioned the operational level, and I was thinking then, trying to square that with Gray’s book, saying well actually, you shouldn’t just be mouthing off at the operational level, like you still don’t understand where the military guys fit in with the strategy, and then I realised that the military strategic thinkers, they have to be, they have to understand the geopolitics, the economics, the culture, all these sorts of things, so that they can influence the political policy, and thereafter can come up with probably, and that’s at maybe the cabinet level, maybe, or maybe at parliament level, and then at the operational level, which might be them engaging with the Department of Defence, that’s definitely frictional, definitely a bounce back of ideas and whatnot, and you just can’t do it effectively unless you know about them.

Then I was reading Gray’s 2005 paper, How Has War Changed Since the End of the Cold War, and he’s on about, I don’t know what he’s on about, but that’s what I have at the moment.and.

Then, the nature of war is about being politics by other means. It's about effecting policy using military tools. Whereas the character of war is something I suppose I'd be more familiar with. GRAY_2018 AND COHEN_2002	

 Draft based on user input — personalise before submission.

\chapter*{Feelings Appendix – Week 2}

The most salient feeling this week was resentment. The DSS block demanded an intensity of engagement that conflicted with my home responsibilities, leaving unfinished domestic tasks and reduced time for exercise. This created a sense of being pulled away from priorities I value, which sharpened my frustration and made me less willing to view the course workload positively.

This mattered because resentment distorts my leadership judgement by encouraging a defensive stance. Instead of approaching learning opportunities with openness, I became more inclined to question the legitimacy of the course structure itself. In a professional context, such a mindset could risk disengagement from institutional demands at precisely the moment when commitment is expected.

Because of this, I adjusted by trying to be more deliberate in linking my contributions to specific authors during syndicate discussions, even when the workload felt overwhelming. This gave me a tangible sense of progress and control, and helped counter the impression that the reading was diffuse and unmanageable.

What I learned is that resentment is best mitigated by small, controlled acts of agency. In leadership terms, this reinforces the value of reframing frustration into a constructive practice — here, the discipline of attribution — which preserves both engagement and personal integrity.

Next step: When resentment surfaces, I will implement a short “grounding action” (such as logging one clear author-idea link) before stepping away, to prevent negative emotion from undermining my overall learning process.

⸻

Would you like me to weave in leadership theory (e.g. emotional intelligence, resilience frameworks) to give this appendix more academic weight, or keep it compact and personal?