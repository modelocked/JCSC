\chapter{My Journey to the Course}

This chapter reflects on the formative experiences that have shaped my professional and personal development to date. It traces my early career, operational service, leadership evolution, and intellectual influences, highlighting what has changed about me, what has remained consistent and what these patterns suggest about my character and leadership style.

My journey to the course began in 2003 when I entered the Cadet School. After twenty one months I was commissioned as an Infantry Officer in the 3rd Infantry Battalion at Stephens Barracks, Kilkenny. From the outset my path was shaped by service and academic choices. At around twenty two years of age I entered a period of rapid maturing. I completed the Gold Gaisce (President's Award), taught myself German through audio materials, travelled and committed to structured learning. This followed a phase of difficult behaviour which proved formative. The period from twenty one to twenty three altered my trajectory for the better. A further decisive moment was the choice to pursue physics rather than arts. That decision enabled me to complete a Master's by research in physics. Although I had no ambition to join the Ordnance Corps, this academic path created that opening.  

Early postings included the External Education and Training Branch in Defence Forces Headquarters. I later completed the Ordnance Young Officers Course and qualified as an Ordnance Technical Officer in 2016. Appointments in the Ordnance School and the Ordnance Company broadened my technical and instructional base. Operational service took me overseas to Lebanon in 2012 and later to Syria as a bomb disposal officer. Syria was formative, not only for technical practice but for team building, mentoring officers and handling personalities. I was fortunate to deploy during a period of high operational tempo, likely performing more live \gls{eod} and \gls{iedd} tasks than peers. These experiences enhanced credibility but also reinforced my scepticism of institutional processes that lagged behind operational reality. With hindsight I see that my own sense of pride in operational activity may have limited empathy for colleagues who did not have the same exposure, which is an important reminder that perspective-taking is part of leadership.  

A further transformation occurred when I was appointed Company Commander overseas in 2021–2022. In my promotion interview I argued that the hardest part of higher command would be managing officers rather than soldiers. The board disagreed, yet in practice I was proved correct. Some officers were highly competent, others incompetent, many disillusioned. Each required a different style of leadership. This was intensely difficult yet confirmed that command is as much about managing personalities and expectations as it is about tactics. While my reflection highlights personal resilience, organisational research stresses that effective mission command depends not only on individual adaptability but also on institutional trust \parencite{BACHMANN_2015}. I also recognise that from the perspective of subordinates, my interventions may not always have appeared consistent, underlining the importance of clarity when balancing support with challenge.  

My leadership style has evolved considerably since I was a young Lieutenant. Then I intervened in every minor transgression, listening too closely to corporals and sergeants. Today I intervene less, focusing on what matters. Silence may represent maturity, however it can also be neglect. Over time I have become more situational in approach. I avoid micromanagement, provide broad guidance and allow autonomy. I encourage initiative, support learning through experience and take pride in the development of others. This situational flexibility aligns with leadership theory on adaptive practice \parencite{HEIFETZ_2002}, though my high assertiveness and low expressiveness \parencite{EMERGENETICS_2025} make balance difficult. A psychometric bias towards a rational analytical preference risks underplaying social dynamics; I must therefore compensate by deliberately weighting relational factors and inviting feedback from those with contrasting styles.  

Another key moment came when I pursued and won a redress of wrongs against the organisation. The process exposed the institution at its worst but also confirmed that integrity can and must be defended even against the system itself. This strengthened my belief that credibility is grounded not only in competence but also in fairness and consistency. However, while personal victory reaffirmed my resolve, it risked deepening cynicism towards the wider organisation. To mitigate this, I must avoid assuming that organisational failings are inevitable and instead contribute to reform through constructive engagement.  

Another formative element of my professional development has been my custodianship of the EOD/IEDD manuals, instructions and standard operating procedures of the Defence Forces. Over the last eight years I have redesigned, rewritten and published every EOD syllabus and course in service. This work has gone beyond academic writing to the supervision of training delivery: as syllabi were turned into courses and those courses began running, I continued to oversee their execution. Even when I was not formally posted to the Ordnance School, I maintained responsibility for modernising instructions and updating procedures. This culminated in the current Defence Forces Explosive Ordnance Disposal Instructions \parencite{DORD_2022}, which represent for me the consolidation of nearly a decade of development.  

In undertaking this work I drew not only on operational experience but also on a body of influential literature that shaped my technical, ethical and leadership approach. These texts have underpinned my thinking on how best to prepare EOD operators for both the technical and human dimensions of their work.  

\begin{itemize}
	\item \nocite{CASTNER_2014} \textit{The Long Walk} — Castner, B.  
	\item \nocite{COCHRANE_2012} \textit{The Development of the British Approach to Improvised Explosive Device Disposal in Northern Ireland} — Cochrane, B.  
	\item \nocite{HUGHES_2017} \textit{Painting the Sand} — Hughes, K.  
	\item \nocite{HUNTER_2009} \textit{Eight Lives Down} — Hunter, C.  
	\item \nocite{HUNTER_2010} \textit{Extreme Risk} — Hunter, C.  
	\item \nocite{OPPENHEIMER_2009} \textit{IRA: The Bombs and the Bullets} — Oppenheimer, A. R.  
	\item \nocite{STYLES_1975} \textit{Bombs Have No Pity} — Styles, G.  
\end{itemize}  

In recent years I have received niche IEDD training through an international project. Prior to the course I was the national expert for this programme. The technical instruction was excellent, yet the formative element was a hostage handling course, which demonstrated that I possess genuine skill in handling persons in crisis. This underscored that leadership is not confined to technical competence but extends into human resilience and empathy under stress.  

A further dimension of my development has been shaped by sustained reading. For over twenty years I have examined the origins of totalitarianism and the ethical causes of violence. This has been a long-term intellectual interest which continues to inform my professional outlook. While deployed to Lebanon in 2021 I extended this focus by reading extensively on jihadist ideology and the rise of ISIS. Together these works form the backbone of my ethical outlook and my understanding of how violence is rationalised by states and movements alike.  

The following books are of particular note in shaping my perspective on the origins of totalitarianism and violence:  
\begin{itemize}
	\item \nocite{ARENDT_1966} \textit{The Origins of Totalitarianism} — Arendt, H.  
	\item \nocite{ARENDT_1970} \textit{On Violence} — Arendt, H.  
	\item \nocite{ARENDT_2006} \textit{Eichmann in Jerusalem} — Arendt, H.  
	\item \nocite{ARENDT_2018} \textit{The Human Condition} — Arendt, H.  
	\item \nocite{GROSSMAN_1996} \textit{On Killing} — Grossman, D.  
	\item \nocite{HITLER_1999} \textit{Mein Kampf} — Hitler, A.  
	\item \nocite{KEEGAN_1976} \textit{The Face of Battle} — Keegan, J.  
	\item \nocite{MARX_2015} \textit{The Communist Manifesto} — Marx, K. \& Engels, F.  
	\item \nocite{MALIA_1994} \textit{The Soviet Tragedy: A History of Socialism in Russia, 1917--1991} — Malia, M.
		\item \nocite{SHIRER_1991} \textit{The Rise and Fall of the Third Reich} — Shirer, W. L.
	\item \nocite{SNYDER_2010} \textit{Bloodlands: Europe Between Hitler and Stalin} — Snyder, T.
	\item \nocite{SOLZHENITSYN_2018} \textit{The Gulag Archipelago} — Solzhenitsyn, A.  
	\item \nocite{STERN_2016} \textit{ISIS: The State of Terror} — Stern, J. \& Berger, J. M.  
\end{itemize}  

At this stage of my career I arrive at the Joint Command and Staff Course with clear priorities. My ambition is to achieve at a level commensurate with my education and professional background yet not at the expense of family. In the past I neglected home life. That is no longer acceptable. Family now comes first. Alongside this I intend to maintain health and fitness, training regularly and sustaining balance. I recognise that the course will be demanding and at times a burden. I worry that workload may encroach on home life or affect mental health. I am also conscious of my scepticism towards ideological perspectives in academic settings. My assertiveness makes silence difficult when I believe something is wrong, yet silence itself can weigh heavily. Personal scepticism carries a risk of disengagement; I should actively cultivate openness without abandoning critical judgement.  

\textbf{Thesis:} My journey demonstrates that technical competence, operational experience and personal resilience must be balanced by family, integrity and the conscious management of cynicism if I am to contribute effectively at strategic level.  

\begin{itemize}
	\item My personal case narrative is subjective; I will validate it through feedback, doctrine and psychometric data.  
	\item Silence taken as maturity can tip into neglect; I will set clear thresholds for when to intervene.  
	\item Cynicism towards institutions risks disengagement; I will reframe it as motivation for constructive reform.  
\end{itemize}  

\textbf{Next step:} During the course I will seek structured peer and mentor feedback on when my silence is seen as support and when it is perceived as neglect, using this to calibrate my situational leadership practice.

In summary, my journey shows clear change in how I lead and what I prioritise: from micromanagement to situational flexibility, and from career-first to family-first. What has stayed the same is my determination, assertiveness, and integrity. This says that I am resilient and principled, but also that I must consciously manage scepticism to ensure empathy and balance. My next step on the course is therefore to seek structured feedback, so that my silence is recognised as support rather than neglect and to calibrate my leadership practice with clarity and perspective.
