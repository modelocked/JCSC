\chapter{Individual Leadership Statement}

This chapter builds directly on my personal development goals by setting out my philosophy of leadership.  It introduces the principles I believe underpin effective command. It explains how my style has evolved. It evaluates the influences that continue to shape me as a leader. The statement is not static:  it acknowledges both strengths and blind spots. I try to highlight how competence, integrity and respect  must be balanced with humility, empathy and openness to feedback.

As a leader I believe that competence, integrity and respect form the foundation of effective command. My philosophy is not based on charisma or rhetoric but on steadiness under pressure, the capacity to adapt and the willingness to uphold ethical standards even when unpopular. Leadership for me is both authoritative and authentic. It is authoritative in providing clear direction, demanding high standards and ensuring accountability. It is authentic in being grounded in humility, service and respect for those I lead.  

My leadership style has evolved considerably. As a young Lieutenant I intervened in every minor transgression and listened too closely to corporals and sergeants. Today I prefer to provide broad guidance, encourage initiative and step back to allow autonomy. Silence is not weakness but judgement. Nevertheless I recognise that I can remain silent when I should intervene. I am highly assertive and this can lead to overbearing behaviour. I am also reserved in large groups which means that what I intend as support is not always perceived that way. Colleagues may see restraint as disengagement, which shows that perception matters as much as intent in leadership.  

My practice has been influenced by a wide body of leadership and professional texts. Carnegie’s classic work on influence, Voss’s negotiation strategies and Peterson’s rules for personal discipline shaped how I engage with people at a personal level. Marquet’s advocacy of distributed control and Patterson et al.’s model of dialogue in high-stakes contexts informed my approach to empowerment and communication. McChrystal and Fick illustrated the burdens of command under operational pressure, while Winters and von Luck exemplified calm dependability in crisis. Together these sources reinforce the value of influence, trust and ethical responsibility over reliance on authority alone.  

Of particular influence were:  
\begin{itemize}
	\item \nocite{BUTLER_1990} \textit{Gender Trouble: Feminism and the Subversion of Identity} — Butler, J. 
	\item \nocite{CARNEGIE_1936} \textit{How to Win Friends and Influence People} — Carnegie, D.  
	\item \nocite{FICK_2006} \textit{One Bullet Away} — Fick, N.  
	\item \nocite{MCCHRYSTAL_2013} \textit{My Share of the Task} — McChrystal, S. A.  
	\item \nocite{MARQUET_2015} \textit{Turn the Ship Around!} — Marquet, L. D.  
	\item \nocite{PATTERSON_2002} \textit{Crucial Conversations} — Patterson, K., Grenny, J., McMillan, R. \& Switzler, A.  
	\item \nocite{PETERSON_2018} \textit{12 Rules for Life} — Peterson, J. B.  
	\item \nocite{VOSS_2017} \textit{Never Split the Difference} — Voss, C. \& Raz, T.  
	\item \nocite{WINTERS_2006} \textit{Beyond Band of Brothers} — Winters, R. D. \& Kingseed, C. C.  
	\item \nocite{HACKWORTH_1990} \textit{About Face} — Hackworth, D. H. \& Sherman, J.  
	\item \nocite{GROSSMAN_1996} \textit{On Killing} — Grossman, D.  
	\item \nocite{KOLENDA_2001} \textit{Leadership: The Warrior’s Art} — Kolenda, C. D. (ed.)  
	\item \nocite{DF_LDR_20233} \textit{Defence Forces Leadership Doctrine} — Óglaigh na hÉireann  
	\item \nocite{MACDONALD_1984} \textit{Company Commander} — MacDonald, C. B.  
	\item \nocite{VON_LUCK_1989} \textit{Panzer Commander} — von Luck, H.  
	\item \nocite{KEEGAN_1976} \textit{The Face of Battle} — Keegan, J.  
	\item \nocite{GUDERIAN_1952} \textit{Panzer Leader} — Guderian, H.  
\end{itemize}

I see myself as situational in approach, willing to adapt to the context and the people involved. I am team oriented and transformational in outlook. I seek out talent, value difference and design jobs to be meaningful and challenging. I take pride in developing others and gain satisfaction from seeing them succeed. I strive to lead by example, maintaining professional standards in discipline, communication and fitness. I value honesty and candour, even when difficult.  

The leaders I admire most are those who remain calm and competent in crisis. Major Dick Winters and Major Hans von Luck exemplify the qualities I aspire to: steady, dependable and respected without flamboyance. I want to be the person colleagues and subordinates can rely upon in difficult moments. Within my professional domain I aim to be the trusted expert in improvised explosive device disposal \gls{iedd}, the individual whose advice is sought when it matters most.  

My Emergenetics profile supports this assessment. It shows high analytical (43\%, 95th percentile) and conceptual (35\%, 77th percentile) preferences, with very low structural (6\%) and social (16\%) thinking. Behaviourally I rank extremely high in assertiveness (95th percentile), very low in expressiveness (15th percentile) and mid-range in flexibility (51st percentile). This combination explains both my strengths and my blind spots. I am rational, visionary and determined but can be disorganised, cynical and intimidating to others. It also shows why I thrive on abstract, big-picture problems yet must deliberately invest effort in relational and structural detail. Recognising these traits allows me to work on balance, ensuring that technical competence does not come at the cost of empathy or patience.  

Ultimately my leadership statement is simple. I intend to lead with competence, integrity and respect, to balance authority with authenticity and to uphold the values of service and family. I accept that I remain a work in progress. I am still learning in spite of myself.  

\textbf{Thesis:} Effective leadership requires balancing authority with authenticity, combining analytical rigour with relational sensitivity, and embedding competence in ethical practice.  

\begin{itemize}
	\item High assertiveness and low expressiveness may lead to misperceptions of intent; implication: seek structured feedback on how my silence or candour is interpreted.  
	\item Strong analytical-conceptual bias risks neglect of detail and relationships; implication: deliberately draw on colleagues with structural and social strengths.  
	\item Reliance on experiential learning provides insight but lacks generalisability; implication: triangulate personal reflection with doctrine and peer-reviewed leadership research.  
\end{itemize}

\textbf{Next step:} Use peer and mentor feedback during the course to refine when to intervene and when to remain silent, calibrating assertiveness with relational awareness.

In summary, my leadership statement reflects both change and continuity. I have moved from micromanagement to situational flexibility. Naturally determination, assertiveness  and integrity remain constant. This says that I am resilient and principled. I must consciously manage scepticism and perception to ensure my intent is not misread.

Having articulated my philosophy of leadership, the next chapter turns to the Command, Leadership and Management (CLEM) module. This provides the academic frameworks, psychometric testing and applied exercises against which my leadership style can be further tested and refined.