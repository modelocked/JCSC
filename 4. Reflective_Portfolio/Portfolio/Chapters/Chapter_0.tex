\chapter{Full Master Narrative}


I began my military career in 2003 when I entered the Cadet School. After 21 months of training, I was commissioned as an Infantry Officer in the 3rd Infantry Battalion at Stephens Barracks in Kilkenny. My career path from the outset was shaped not only by military service but also by academic choices. One of the most decisive was my decision to transfer from a Bachelor of Arts to a Bachelor of Science in Physics. That choice led me to complete a Master’s by Research in Physics. At the time, I had no intention of pursuing a career in the Ordnance Corps --- in fact, I never wanted to become an Ordnance Officer --- but this academic trajectory opened that door.

At around 22 years of age I commenced a period of rapid maturing. I complete the Gold Gaisce (President's Award), began a self-taught audio course in German, began travelling and learning. This happened after a period of significant anti-social behaviour, which was exceptionally formative. Hence, the period 21 - 23 years of age basically changed my life. For the better.

Early postings included service in the External Education and Training Branch in Defence Forces Headquarters. I later completed the Ordnance Young Officers Course, qualifying as an Ordnance Technical Officer in 2016. From there, I took on a series of technical and instructional appointments in the Ordnance School and the Ordnance Company. Operational service took me overseas to Lebanon in 2012 and later to Syria as a Bomb Disposal Officer. Syria in particular left a deep impression on me, as the chaos and nihilism of conflict showed that even when you put in the effort and do the right thing, outcomes can remain illogical or be undone by enemy action.

I also served as Technical Officer in the Army Ranger Wing before returning to Defence Forces Headquarters in the Operations Branch. These experiences gave me a mix of operational, technical, and staff exposure at different levels of the organisation.

A key transformation point in my career came when I was appointed as a Company Commander overseas. During my promotion interview, I had argued that the hardest part of higher command would be managing officers, not soldiers. The board disagreed, but I was later proved right. As a Company Commander, I had officers under my command who ranged from highly competent to completely incompetent, with many disillusioned, sceptical, or disengaged. Managing them was particularly challenging: some needed close supervision, others needed space, and all required something different from me as their leader. It was an intensely difficult but formative experience, teaching me that command at that level is less about tactics and more about balancing personalities, expectations, and human complexity.

Reflecting on my development, I can see how my leadership style has evolved since I was a young Lieutenant. At that stage, I relied heavily on corporals and sergeants and felt obliged to intervene in every small barracks transgression. I believed it was my duty to correct every issue. Today, I no longer see it that way. I speak far less about minor matters, preferring to focus on what is truly important. Silence and restraint, I now understand, can be a sign of maturity. Still, I recognise the risk: sometimes I remain silent when I probably should address an issue.

My leadership style has become situational. I adapt how I lead depending on the group and the context, not because I am inconsistent but because I believe different parts of my personality are appropriate in different circumstances. I avoid micromanagement, preferring to give broad guidance and allow autonomy. I encourage initiative, support learning through experience, and take satisfaction in seeing people grow as a result of my input. I make a deliberate effort to treat subordinates with respect and politeness. That said, I recognise that what I consider support is not always perceived that way. Some expect blind agreement with their assessments or their treatment of others, and when I do not provide that, I have been accused of hypocrisy.

I see myself as both a team-oriented and transformational leader. I actively seek out talent, try to develop colleagues, and value differences among people. I place importance on putting forward a vision and aligning others with it. I design and redesign jobs to make them meaningful and challenging, believing that people should find purpose in their work. Yet I also see myself as situational: I change how I act to match the circumstances, whether leading a technical unit, a company, or a staff group.

The leaders I admire most are those who embody steadiness under pressure. Figures like Major Dick Winters and Major Hans von Luck inspire me because they were cool, calm, collected, and competent under fire. They were not necessarily charismatic or flamboyant, but people respected them and turned to them in crises. That is the style I aspire to: I want to be someone my colleagues and subordinates can turn to in difficult moments, knowing that my input will be dependable and competent. In my specific professional domain, I want to be recognised as the go-to expert in improvised explosive device disposal (IEDD), the trusted authority within the Defence Forces on that subject.

My Emergenetics profile reinforces these tendencies. I am highly Analytical (43 per cent, 95th percentile) and Conceptual (35 per cent, 77th percentile), with much lower preferences for Structural (6 per cent, 13th percentile) and Social (16 per cent, 34th percentile) thinking. Behaviourally, I am extremely Assertive (95th percentile), very low in Expressiveness (15th percentile), and moderate in Flexibility (51st percentile). This profile makes me rational, visionary, forceful, and direct, but often reserved in larger groups. It explains why I thrive on complex, abstract problems but sometimes neglect detail or relational nuance. It also suggests why people may find me intimidating: I tend to drive hard for logical, big-picture solutions, often in a reserved way, but with assertive force when I believe I am right.

My strengths are clear. I am tenacious, able to see tasks through to completion. I have strong staff skills and am regarded as candid and intelligent. My public communication is effective, and I possess high empathy. I take pride in seeking out talent, valuing differences, and helping others to develop.

Yet my weaknesses are just as evident. I have a temper and can be quite angry. I am disorganised, often taking on too much at once and spreading myself too thin. My assertiveness, while a strength, can also make me overbearing; I often struggle to let things go. I am less expressive than most, sometimes too reserved in groups. My social battery is low: I enjoy being around people but quickly need time alone, which can create stress during deployments. I can be cynical, distrusting both individuals and the system. I want to believe in the best in people, but experience has taught me otherwise. I believe strongly in public service, but when institutions fall short, my cynicism grows. I also worry about my resilience in the face of chaos: in the nihilism of war, where logic and fairness no longer apply, I fear I could be affected deeply.

A transformational experience was winning a redress of wrongs against the organisation. It revealed the institution at its worst --- bureaucratic, cynical, and unjust --- but also taught me that fairness and integrity can be defended, even against the system itself. It reinforced my conviction that leaders must sometimes take on their own organisations if they are to uphold the values they claim to represent.

As I begin the Joint Command and Staff Course, my personal development goals are straightforward but important. I want to achieve a grade that reflects my education and input, but not at the expense of my family. Family is now my highest priority. In the past, I did not live that way, but now I intend to. I want this course to enhance my professional growth without undermining my home life. I also want to maintain my fitness throughout the year, exercising three times a week and keeping my body mass steady.

My feelings about the course are mixed. I expect it to be a major burden --- ``a gigantic pain'' --- particularly as I already hold two Master’s degrees and do not have an academic itch to scratch. I worry about the impact on my mental health, personal time, and physical fitness. I fear that the workload will encroach on my home life and undo the balance I have worked hard to create. Another worry is ideological: I am concerned about exposure to radical leftist perspectives within the university environment. My assertiveness makes it difficult for me to stay silent if I feel something is being imposed unfairly. If I speak up, it could cause conflict; if I do not, it will gnaw at me from the inside.

In terms of values, family is paramount. I am determined to prioritise it over everything else, even though in the past I did not. I also remain committed to the values of military and public service, though my cynicism has sometimes eroded my trust in institutions. I strive to always do my best, which makes it difficult for me to consciously lower the bar for this course --- even though I know doing so might protect family time.

In sum, my journey has been defined by technical expertise, operational experience, leadership challenges, and personal evolution. From the Cadet School to bomb disposal in Syria, from the Army Ranger Wing to Defence Forces Headquarters, from youthful micromanagement to situational leadership, from being wronged by the organisation to standing my ground and winning a redress --- all of these experiences have shaped the leader I am today. Now, my focus is on reconciling professional ambition with personal values, particularly family, while striving to embody the calm, competent, and respected leadership that I admire in others.

I thought the presentation was good, a lot of effort involved in it. It was worth doing, I think. I was working with Finola, we worked well together, no issues there. It's good to get back to public speaking, but did it actually teach me anything? Probably not. The reality is, you know, it's probably wasted work, to be honest, really. I don't think I learned a whole lot doing it. I expect I might change my views when I get my results, but yeah, that's how I feel at the moment. The Cranfield University Week with Dr. Dennis Vincent. Okay, you know, high-quality academic stuff, don't get me wrong, and I don't knock his credentials or even the requirement to have some of that deliberate stuff, but I'm not sure how much of it was actually new. It's a lot of fluff, to be honest. Certainly even the leadership stuff, if any of that stuff is new to you at this point in your career, I don't know what the fuck you're doing here, to be honest. But yeah, the models that he implemented were decent, or he introduced us were decent, like the seven S's, then East, and the force field model. Yeah, things like that were, they were okay. Yeah, they were okay, you know, to introduce us to them, but I thought the best part was the last day when we talked about ethics. That was a real discussion, but like, as far as I'm concerned, the, you know, exercise that we're doing during the week were shite by comparison, but we didn't get to do any real discussion on the ethics piece, because we ran out of time. I thought the strategic communications, outsourced strategic communications piece was a disgrace. They were terrible, got nothing out of it. They didn't critique us at all. It was completely bullshit. And I thought the emergenetics personality stuff, that was useful. It was interesting. I'm not sure it was useful, but it's interesting. I also think I was shocked when people were shocked that it gave them, it seems at least one individual, he didn't know any of these things about himself, and I thought that was bizarre, but there you go, it taught me nothing new about myself. If anything, the day was a great way to work as an icebreaker amongst the class. Yeah, Gavin Egerton's leadership mission command thing was, it was interesting, but I'm not sure how much it is really applicable in our organization. Probably the thing that I learned about mission command is that, you know, a leader has to get involved at some point at the lowest level, and people can interpret that as micromanaging, but if you're doing it correctly, you're still getting involved. Doing the literature or the dissertation proposal was a gigantic pain in the hole. Fucking gigantic pain in the hole. Brian and I can meet together and obviously we end up discussing all sorts of things regarding the course. Discussing all sorts of things regarding the presentations as well, and his one was on diversity, and we ended up having debates about diversity within the Defence Forces and generally. And I noticed that Brian's views are very sympathetic and empathetic, and mine are less so.